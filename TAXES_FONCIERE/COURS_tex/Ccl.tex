\chapter*{Conclusion Générale}

Le présent cours consacré aux taxes foncières avait pour objectif principal de doter les inspecteurs du cadastre d’une maîtrise approfondie, à la fois juridique, technique et opérationnelle, des règles applicables à l’assiette de la fiscalité foncière locale au Togo. 
À travers l’étude de la Taxe Foncière sur les Propriétés Bâties (TFPB) et de la Taxe Foncière sur les Propriétés Non Bâties (TFPNB), il a mis en lumière le rôle structurant du foncier dans le financement du développement local et dans l’aménagement du territoire.\\

\noindent L’analyse des fondements juridiques et du champ d’application des taxes foncières a permis de rappeler que ces impôts reposent sur des principes essentiels : territorialité, annualité, réalité de l’assiette et rattachement de l’imposition à la propriété au 1\textsuperscript{er} janvier. 
La distinction entre propriétés bâties et non bâties, loin d’être purement théorique, constitue une clé déterminante pour la sécurisation de l’assiette et la prévention des erreurs d’imposition.\\

\noindent Le cours a également mis en évidence que la performance de la fiscalité foncière dépend moins des taux d’imposition que de la qualité de l’assiette. 
L’identification rigoureuse des biens, la détermination correcte des personnes imposables, le respect des obligations déclaratives et la mise à jour permanente des données cadastrales conditionnent directement le rendement des taxes foncières. 
Dans un système fiscal largement auto-déclaratif, les omissions, les erreurs de qualification et les défauts de déclaration constituent les principales sources de perte de recettes pour les collectivités territoriales.\\

\noindent L’étude détaillée des exonérations, tant permanentes que temporaires, a révélé les tensions existantes entre objectifs sociaux, considérations économiques et impératifs budgétaires. 
Si certaines exonérations répondent à des finalités légitimes, leur champ excessivement large ou leur contrôle insuffisant peut affaiblir durablement la base imposable, en particulier dans les zones urbaines et périurbaines à forte pression foncière. 
À cet égard, le cours a souligné la nécessité d’une approche plus ciblée, fondée sur l’usage effectif des biens, la superficie concernée et la vocation réelle des terrains.\\

\noindent Au cœur de l’ensemble de ces mécanismes se trouve l’inspecteur du cadastre. 
Son rôle dépasse largement la simple collecte de données techniques : il est un acteur central de la justice fiscale, un garant de la fiabilité de l’assiette et un maillon essentiel de la gouvernance foncière locale. 
Par ses constats de terrain, ses évaluations, ses classifications et ses propositions de mise à jour, il contribue directement à la crédibilité de l’administration fiscale et à la capacité des collectivités à mobiliser des ressources propres.\\

\noindent En définitive, la fiscalité foncière apparaît comme un levier stratégique encore insuffisamment exploité au Togo, alors même que son potentiel est considérable. 
La consolidation du cadastre, le renforcement des capacités des inspecteurs, l’amélioration de la coordination entre les services cadastraux, fiscaux et communaux, ainsi que l’adaptation progressive du cadre juridique aux réalités foncières contemporaines constituent des conditions indispensables pour faire de la taxe foncière un instrument efficace de développement local durable.\\

\noindent Ce cours se veut ainsi non seulement un cadre d’apprentissage, mais également un outil de référence pour l’action quotidienne des inspecteurs du cadastre, appelés à jouer un rôle déterminant dans la modernisation de la gestion foncière et dans la construction d’une fiscalité locale plus équitable, plus performante et plus transparente.
