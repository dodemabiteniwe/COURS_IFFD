

%%%%%%%%%%%%%%%%%%%%%%%%%%%%%%%%%%%%%%%%%%%%%%%%%%%%%%%%%%%%%%%% BOURLLION CHAP 1 %%%%%%%%%%%%%%%%%%%%%%%%%%%%%%%%%%%%%%%%%%%%%%

\section{Bref historique des Impôts Directs Locaux (IDL)}
L'histoire des impôts directs locaux remonte à l'époque où les autorités locales ont commencé à financer leurs services publics à travers des prélèvements obligatoires sur les résidents et les activités économiques de leurs territoires.
\begin{itemize}[font=\color{magenta} \Large, label=\ding{230}]
	\item \textbf{Révolution française (1789)} : La Révolution française a marqué une étape clé dans l’évolution des impôts locaux. En 1790, la France a instauré des taxes foncières généralisées pour financer les dépenses des communes, qui étaient les unités locales de gouvernement.
	\item \textbf{Époque précoloniale} :Dans les sociétés précoloniales africaines, les formes d'imposition étaient souvent basées sur des systèmes tribaux ou communautaires. Il ne s'agissait pas d’impôts formels comme ceux observés aujourd'hui, mais plutôt de contributions aux chefs locaux sous forme de biens, de services ou de travail. Ces contributions servaient à financer la protection de la communauté, les cérémonies religieuses, ou les infrastructures de base.
	\item  \textbf{Période coloniale} : Avec l'arrivée des puissances coloniales (notamment la France, le Royaume-Uni, le Portugal et la Belgique), des systèmes fiscaux formels ont été introduits pour financer l'administration coloniale.
	Les impôts locaux ont pris plusieurs formes (Imposition de capitation ou impôt par tête, impôt foncier,travail forcé et corvées)
	\item \textbf{Période postcoloniale}: Après les indépendances dans les années 1950 et 1960, la plupart des pays africains ont hérité des systèmes fiscaux coloniaux, souvent inadaptés aux réalités locales. Avec l'indépendance du Togo en 1960, le pays a hérité des systèmes fiscaux coloniaux, mais a tenté d’adapter ces systèmes aux nouvelles réalités socio-économiques. Les taxes sur la propriété foncière ont été introduites dans les zones urbaines et périurbaines. Cependant, la collecte de ces impôts a toujours été un défi, notamment en raison de l'absence de cadastres bien établis et de la forte informalité des régimes fonciers dans les zones rurales.\\
	
	Bien que la décentralisation ait été introduite dans la législation togolaise pour renforcer l’autonomie des collectivités locales, la mise en œuvre réelle a souvent été retardée ou partielle. En conséquence, la capacité des autorités locales à collecter des impôts directs a été limitée. Cependant, des efforts pour renforcer la gouvernance locale, notamment avec l'organisation des premières élections locales en 2019 après plus de trois décennies, ont donné un nouveau souffle aux initiatives locales, y compris la collecte des impôts.
\end{itemize}

Les impôts directs locaux au Togo sont régis par plusieurs textes législatifs, qui visent à structurer la fiscalité locale dans le cadre de la décentralisation. Voici les principaux: 

\begin{itemize}[font=\color{magenta} \Large, label=\ding{230}]
	\item  \textbf{Loi n°2007-011 du 13 mars 2007 portant décentralisation et libertés locales}\\
	Cette loi constitue l'un des piliers de la décentralisation au Togo. Elle organise l'autonomie des collectivités locales et leur confère des compétences pour la gestion de leurs affaires. En matière fiscale, elle accorde aux collectivités locales le pouvoir de percevoir certaines taxes et impôts pour financer les services publics locaux.
	\item \textbf{ Loi n° 98-006 du 11 février 1998 portant décentralisation au Togo}
	\item \textbf{Loi n°2019-005 du 7 juin 2019 modifiant la loi n°2007-011 du 13 mars 2007 relative à la décentralisation et aux libertés locales}\\
	Cette loi vient actualiser le cadre de la décentralisation, notamment en matière fiscale, à la suite des premières élections locales tenues au Togo après plus de trois décennies. Elle renforce les compétences fiscales des nouvelles municipalités.
	\item \textbf{Code Général des Impôts (CGI)}\\
	Le Code Général des Impôts au Togo, adopté et amendé à plusieurs reprises, comprend des dispositions spécifiques concernant la fiscalité locale. Il regroupe l'ensemble des textes relatifs aux impôts, y compris ceux perçus par les autorités locales.
\end{itemize}
\section{Synthèse des caractéristiques générales des IDL }
\subsection{Types d'impôts directs locaux}
Les principaux impôts directs locaux perçus au profit des collectivités territoriales au Togo incluent :
\begin{itemize}[font=\color{magenta} \large, label=\ding{114}]
	\item La taxe foncière : Elle s'applique aux propriétés bâties et non bâties (terrains). Elle est calculée sur la base de la valeur locative (pour les propriétés bâties) ou la valeur vénale (pour les propriétés non bâties) des biens immobiliers.
	\item La taxe d'habitation : Prélevée sur les occupants des logements, cette taxe depend de la catégorie du logement.
	\item La Patente : \textit{La patente est établie suivant la capacité contributive des redevables appréciée d'après des critères économiques en fonction de l'importance des activités exercées par eux sur le territoire national} (Art. 252 CGI)
	\item La taxe professionnelle unique (TPU) : Elle concerne les petites et moyennes entreprises opérant dans les communes. Elle est simplifiée pour le secteur informel afin de faciliter leur contribution fiscale.
	\item Les taxes sur les marchés : Les collectivités locales perçoivent des taxes sur les activités commerciales dans les marchés locaux.
	\item Les droits d'occupation du domaine public : Ces droits concernent l’utilisation des espaces publics par des particuliers ou des entreprises à des fins commerciales ou industrielles.
\end{itemize}
\subsection{Objectifs des impôts locaux}
Les impôts locaux servent principalement à financer les services publics offerts par les collectivités locales, tels que :
\begin{itemize}[font=\color{magenta} \large, label=\ding{114}]
	\item La gestion des infrastructures locales (routes, marchés, bâtiments publics).
	\item La fourniture de services comme l’eau, l’électricité et les services de santé.
	\item L’éducation (construction et entretien des écoles) et d’autres services de proximité. Ils visent également à renforcer l'autonomie financière des collectivités locales, en leur permettant de générer des revenus propres pour accomplir leurs missions.
\end{itemize}

\noindent  Les impôts directs locaux au Togo constituent un pilier important du financement des collectivités territoriales. Bien que des efforts aient été faits pour améliorer le cadre fiscal local, de nombreux défis persistent, notamment en matière de collecte efficace et d'administration fiscale. Les réformes en cours visent à renforcer la décentralisation et à accroître l'autonomie financière des collectivités locales afin de favoriser un développement local plus équilibré et durable.
\section{Les ressources fiscales des collectivités}

\subsection{L'impact des ressources fiscales dans le budget des collectivités}
\noindent Le graphique figurant ci-après, présente de manière schématique les transferts financiers de l'État au profit des collectivités locales. L'on peut noter la bonne progression à partir de 2021 des contributions après un choc en 2020 en raison du Covid-19. On note également une faible représentation des taxes foncières et qui souligne un potentiel qui reste inexploité.  

\begin{figure}[H]
	\centering
	\begin{subfigure}[b]{0.45\textwidth}
		\centering
		\includegraphics[width=\textwidth]{Recet2019_2023}
		\caption{Évolution des transferts de l'État (2019-2023)}
		\label{fig:evolutrecet}
	\end{subfigure}
	\hfill
	\begin{subfigure}[b]{0.45\textwidth}
		\centering
		\includegraphics[width=\textwidth]{Recet2023}
		\caption{Composition des transferts (2023)}
		\label{fig:comporecet}
	\end{subfigure}
	\caption{Recettes des collectivités}
	\label{fig:Recet}
\end{figure}

En moyenne, la fiscalité directe et indirecte constitue plus de la moitié des ressources
de la section de fonctionnement des collectivités. 

\subsection{Composition de la fiscalité directe locale}
\noindent Les IDL sont habituellement classés en deux catégories:
\begin{itemize}[font=\color{magenta} \large, label=\ding{114}]
	\item  les "impôts ménages" comprenant TH, TFB et TFNB
	\item les "impôts économiques" comprenant la Patente, La TPU
\end{itemize}
Toutefois, les taxes foncières bâtie et non bâtie sont aussi, éventuellement, à la
charge des acteurs économiques.

\subsection{La répartition entre les collectivités}

\begin{table}[H]
	\centering
	\renewcommand{\arraystretch}{1.3}
	\setlength{\tabcolsep}{4pt} % Ajuste l'espacement entre les colonnes
	\begin{tabular}{|p{5cm}|p{5cm}|p{5cm}|}
		\hline
		\textbf{Désignation} & \textbf{Part Etat/OTR/Autres structures} & \textbf{Part collectivités locales} \\ 
		\hline
		Taxe Foncière sur les Propriétés Bâties (TFPB) 
		& 50\% dont 33,33\% à l'État et 16,67\% à l'OTR (article 277 du CGI) 
		& 50\% : 25\% communes, 13\% districts, 7\% FACT (régions et communes), 5\% ANASAP \\ 
		\hline
		Taxe Foncière sur les Propriétés non Bâties (TFPNB) 
		& 50\% dont 33,33\% à l'État et 16,67\% à l'OTR (article 277 du CGI) 
		& 50\% : 25\% communes, 13\% districts, 7\% FACT (régions et communes), 5\% ANASAP \\ 
		\hline
		Taxe d'Habitation (TH) 
		& 0\% (articles 288 et suivants du CGI) 
		& 100\% : 65\% communes, 17\% districts, 10\% FACT (régions et communes), 8\% ANASAP \\ 
		\hline
		Patente 
		& 50\% dont 30\% à l'État, 10\% à l'OTR, 5\% au Fonds spécial pour le développement de l'habitat, 5\% au Fonds national d'apprentissage (article 255 du CGI) 
		& 50\% : 30\% communes, 10\% districts, 5\% FACT (régions et communes), 5\% ANASAP \\ 
		\hline
		Taxe Professionnelle Unique (TPU) et taxes directes assimilées 
		& 10\% à l'OTR 
		& 90\% : 45\% communes, 20\% districts, 15\% FACT (régions et communes), 10\% ANASAP \\ 
		\hline
	\end{tabular}
	\caption{Répartition des produits des impôts locaux  entre l'État, OTR, autres structures, et les collectivités territoriales ({\footnotesize \textsl{Décret 2021-039/PR}}).}
\end{table}
Chaque collectivité, dès lors qu'elle bénéficie d'une part de taxe, doit normalement en voter le taux.

\subsection{Les intervenants de la fiscalité directe locale}
\subsubsection{Une relation tripartite}
Trois intervenants - ou catégories d'intervenants - sont associés à la mise en œuvre de
la fiscalité directe locale :
\begin{itemize}[font=\color{magenta} \large, label=\ding{43}]
	\item les collectivités territoriales, qui votent les taux des IDL ainsi que les
	abattements et exonérations relevant de leurs domaines de compétence ;
	\item les contribuables, à qui incombent des obligations déclaratives et qui s'acquittent
	de leurs impositions auprès de l’État ;
	\item l'État, qui calcule, contrôle, recouvre et reverse les produits issus de la FDL aux
	collectivités bénéficiaires.
\end{itemize}

Dans ce circuit, chacun des intervenants remplit un rôle clairement défini.


\begin{tikzpicture}[
	node distance=1cm and 1cm, 
	every node/.style={align=left, thick, draw, minimum width=4cm, minimum height=1.6cm},
	arrow/.style={->, thick, >=stealth}
	]
	
	% Noeud État avec couleur rouge
	\node (etat) [text=red] 
	{\textbf{ÉTAT} \\ Assiette / Contrôle \\ Recouvrement \\ Conseil \\ Reversement};
	
	% Noeud Collectivités Locales avec couleur verte
	\node (collectivites) [text=green!70!black, below left=of etat] 
	{\textbf{COLLECTIVITÉS LOCALES} \\ Vote des taux \\ Abattements \\ Exonérations};
	
	% Noeud Contribuables avec couleur bleue
	\node (contribuables) [text=blue, below right=of etat] 
	{\textbf{CONTRIBUABLES} \\ 1) Déclarations \\ 2) Paiements};
	
	% Arcs entre les éléments
	\draw[arrow] (etat.south west) -- (collectivites.north east);
	\draw[arrow] (collectivites.north) -- (etat.south west);
	\draw[arrow] (etat.south east) -- (contribuables.north west);
	\draw[arrow] (contribuables.north) -- (etat.south east);
	
\end{tikzpicture}


\subsubsection{Le rôle des collectivités locales}
\begin{itemize} [font=\color{magenta} \large, label=\ding{230}]
	\item \textbf{Le vote des taux :}\\
	En vertu de \textbf{l'article 332} de la loi sur la décentralisation et des libertés locales, la	création	des	impôts	et	taxes	relève	du	domaine	de	la	loi. Le	conseil	local à néanmoins, par	délibération, la possibilité	d'en	fixer 	le	taux	dans	les	conditions	déterminées	par	la	loi	de	finances.	
\end{itemize}

\subsubsection{Le rôle de l'État}

\begin{itemize} [font=\color{magenta} \large, label=\ding{230}]
	\item \textbf{La responsabilité de l'État}\\
	La fiscalité directe locale constitue un domaine particulièrement sensible dans la mesure
	où elle met en présence trois parties aux intérêts divergents : l'administration, le
	contribuable et la collectivité.
	\begin{itemize}
		\item Si le service ne respecte pas la loi fiscale en réclamant au contribuable un montant inférieur à ce qui est légalement dû, la collectivité sera fondée à rechercher sa
		responsabilité.
		\item Si le service donne trop facilement raison à la collectivité, c'est le contribuable qui
		pourra contester l'imposition mise à sa charge.
	\end{itemize}
	
	Les collectivités locales peuvent estimer que les bases imposables aux IDL des
	contribuables installés sur leur territoire sont exonérées à tort ou sous-évaluées et
	réclamer de ce fait une indemnisation à hauteur des pertes qu'elles ont subies et
	quelles que soient par ailleurs les difficultés de l'action de l'administration des finances publiques dans l'exercice de ses misions.
	\item \textbf{Les domaines d'intervention de l'État}\\
	\begin{itemize}[font=\color{magenta} \large, label=\ding{43}]
		\item La mission de conseil
		\item L'assiette et le contrôle des impositions\\
		L'OTR assure l'assiette des IDL, c'est à dire qu'elle calcule les montant des
		impositions à la charge des contribuables.
		\begin{itemize}
			\item les services d'assiette (Cadastre, BOFiC, BOFiP) recensent les redevables et les biens imposables ;
			\item les services en charge des impôts fonciers centralisent les données utiles à la liquidation des IDL (recensement des bases d'imposition, des taux, ...).
		\end{itemize} 
		Les service en charge du contrôle fiscal (services d'assiette,brigades de vérifications, ...) veillent à régulariser toute absence ou insuffisance d'imposition détectée.
		Ces services traitent aussi les demandes en réduction ou exonération présentées par
		les contribuables.
		\item Le recouvrement
		\item Le reversement aux collectivités
		En application des textes, les taxes et impositions perçues sont attribuées mensuellement aux collectivités, à raison d'un douzième de leur montant total, tel qu'il est prévu au budget de l'année en cours.
	\end{itemize}
\end{itemize}

\subsubsection{Le rôle des contribuables}

\begin{itemize}[font=\color{magenta} \large, label=\ding{230}]
	\item \textbf{Les obligations déclaratives}\\
	Pour le calcul de la valeur locative :\\
	\underline{\textbf{	Article 86 CGI} :}\\
	Chaque année, les propriétaires et principaux locataires et en leur lieu et place les gérants d'immeubles, sont tenus de déposer une déclaration auprès de l’administration fiscale, au plus tard le 31 mai de l'année d’imposition. La déclaration doit indiquer les mentions ci-après :
	\begin{enumerate}
		\item les nom et prénoms usuels de chaque locataire, la consistance des locaux qui leur sont loués, le montant du loyer principal et, s'il y a lieu le montant des charges ;
		\item les nom et prénoms usuels de chaque occupant à titre gratuit et la consistance du local occupé
		\item la consistance des locaux occupés par le déclarant lui-même ;
		\item la consistance des locaux vacants ;
		\item la superficie exacte des terrains imposables à la taxe foncière sur les propriétés non bâties.
	\end{enumerate}
	\underline{\textbf{Article 87 CGI }:}\\
	Les constructions nouvelles, reconstructions et additions de constructions ainsi que les 
	changements de consistance ou d'affectation des propriétés bâties et non bâties, sont portés par les propriétaires à la connaissance de l'Administration fiscale dans les quatre (04) mois de leur réalisation définitive.\\
	
	\noindent À défaut de déclaration dans les délais prévus au présent article, les constructions nouvelles, additions de constructions et reconstructions sont imposées dès le 1er janvier de l'année qui suivra celle de leur achèvement présumé.\\
	
	\noindent Pour l'établissement de la taxe d'habitation :\\
	\underline{\textbf{Article 88 CGI:}}\\
	Les contribuables assujettis à la taxe d'habitation doivent souscrire une déclaration sur un imprimé conforme au modèle prescrit par l'Administration fiscale auprès des services des impôts de leur lieu de résidence au plus tard le 15 janvier de l’année d’imposition.\\
	
	\noindent En cas de changement de résidence, le contribuable assujetti doit notifier ce changement, au service des impôts du nouveau lieu de résidence, dans les deux (02) mois de l'entrée de la jouissance desdits locaux.
	\item \textbf{Le paiement}\\
	\underline{\textbf{Article 85 CGI}} : \\
	Les taxes foncières sont dues pour l'année entière à compter du 1 er janvier de l'année de l'imposition.\\
	
	\noindent La taxe foncière est payable spontanément à la caisse du receveur des impôts. Les paiements sont effectués  au moment du dépôt de la déclaration annuelle des immeubles selon un modèle fourni par l’administration fiscale.\\
	
	\noindent Le propriétaire du sol et le locataire sont solidairement responsables du paiement de l'impôt.\\
	
	\noindent\underline{\textbf{ Article 88 CGI}} : \\
	Les contribuables assujettis à la taxe d'habitation doivent souscrire une déclaration sur un 
	imprimé conforme au modèle prescrit par l'Administration fiscale auprès des services des impôts de leur lieu de résidence au plus tard le 15 janvier de l’année d’imposition.\\
	
	\noindent En cas de changement de résidence, le contribuable assujetti doit notifier ce changement, au service des impôts du nouveau lieu de résidence, dans les deux (02) mois de l'entrée de la jouissance desdits locaux.
\end{itemize}

