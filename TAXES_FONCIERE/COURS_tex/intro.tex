\chapter*{Introduction}
La fiscalité foncière représente aujourd'hui’hui l’un des piliers les plus stratégiques de la mobilisation des ressources locales et du financement du développement territorial. Dans le contexte du Togo, elle s’impose non seulement comme un outil de justice fiscale mais aussi comme un instrument d’aménagement du territoire et de renforcement de la gouvernance locale.\\

\noindent À travers la taxe foncière, l’État et les collectivités territoriales affirment le principe d’équité selon lequel la propriété foncière, en tant que richesse stable et visible, doit contribuer au financement des charges publiques.

\section*{La taxe foncière : un impôt de proximité et de responsabilité}

La taxe foncière, qu’elle s’applique aux propriétés bâties (TFPB) ou non bâties (TFPNB), est un impôt réel, direct et annuel, assis sur la valeur cadastrale du bien, indépendamment de la situation personnelle du propriétaire.
Son rôle dépasse la simple contribution financière : elle est le reflet d’une citoyenneté fiscale territoriale, en rapprochant le contribuable de la collectivité qu’il finance. En ce sens, elle favorise la responsabilisation des acteurs locaux et la visibilité de l’impôt au sein de la communauté.\\

\noindent Le cadastre occupe ici une position fondamentale : il fournit la base physique et juridique sur laquelle repose l’assiette de la taxe. L’Office Togolais des Recettes (OTR), à travers la Direction du Cadastre et de la Conservation Foncière, assure la gestion intégrée de cette fiscalité, en veillant à la cohérence entre les données cadastrales, la réglementation fiscale et les procédures administratives.\\
Dans cette chaîne, l’inspecteur du cadastre joue un rôle clé : il relie la donnée foncière à la règle fiscale et garantit la fiabilité de l’imposition.

\section*{Un cours orienté vers la maîtrise des règles et procédures fiscales}

Le présent cours, conçu pour les Inspecteurs du Cadastre formés à l’Institut de Formation Fiscale et Douanière (IFFD), a pour objectif général de permettre une maîtrise approfondie du régime fiscal applicable aux propriétés foncières.\\
Il s’agit d’un cours de spécialisation, venant en complément d’évaluation cadastrale. Ainsi, le présent enseignement ne revient pas sur les aspects techniques de l'évaluation des biens immobiliers, mais s’attache plutôt à développer la compréhension juridique, fiscale et procédurale de la taxe foncière.\\

\noindent Ce cours vise donc à doter les inspecteurs des compétences nécessaires pour :

\begin{itemize}[font=\color{magenta} \Large, label=\ding{114}]
	
\item \textbf{déterminer avec précision le champ d’application de la taxe foncière ;}

\item \textbf{identifier les personnes imposables et comprendre les règles d’assujettissement ;}

\item \textbf{calculer la base imposable et liquider l’impôt conformément au Code Général des Impôts (CGI) ;} 

\item \textbf{appliquer correctement les exonérations légales et temporaires ;}

\item \textbf{assurer un suivi efficace du recouvrement, de l’émission du rôle à la perception effective de la taxe ;}

\item \textbf{et accompagner les contribuables dans le respect de leurs obligations déclaratives et de paiement.}

\end{itemize}

La dimension contentieuse (réclamations, contestations, recours) n’est pas traitée ici, car elle fait l’objet d’un module spécifique consacré au contentieux des taxes foncières.


\section*{L’importance de la maîtrise des règles fiscales dans la performance locale}
La maîtrise des règles et des procédures relatives à la taxe foncière constitue une compétence stratégique et incontournable pour les inspecteurs du cadastre. Elle conditionne directement la performance des administrations fiscales locales et la capacité des collectivités territoriales à mobiliser leurs ressources propres.\\

\noindent En effet, les difficultés récurrentes observées dans le recouvrement des taxes locales au Togo ne découlent pas uniquement du manque de moyens techniques ou logistiques, mais souvent d’une mauvaise interprétation ou application des règles fiscales fondamentales.\\

\noindent  Ces insuffisances se manifestent principalement sur plusieurs aspects clés : 

\begin{itemize}[font=\color{magenta} \Large, label=\ding{230}]
	\item \textbf{La détermination du champ d’application}\\
	
	\noindent La délimitation du champ d’application de la taxe foncière constitue la première étape cruciale de tout processus d’imposition. Or, sur le terrain, des erreurs fréquentes apparaissent dans la distinction entre propriété bâtie et non bâtie, ou encore dans la qualification juridique du bien imposable.\\
	Certains biens sont imposés à tort (par exemple, des infrastructures publiques ou des terrains affectés à un service d’utilité collective), tandis que d’autres échappent à l’impôt faute d’identification correcte.\\
	La méconnaissance des notions de destination principale du bien, de nature juridique de la propriété ou encore du statut d’achèvement des constructions entraîne des faiblesses importantes dans la constitution de la base d’imposition.\\
	
	\noindent Une bonne compréhension du champ d’application permet à l’inspecteur de :\\
	
	\begin{itemize}[font=\color{magenta} \Large, label=\ding{109}]
		\item délimiter avec rigueur les biens imposables au sens du Code Général des Impôts (CGI) ;
		\item éviter la double imposition d’un même immeuble sous plusieurs catégories ;
		\item et garantir l’équité entre contribuables en assurant que tous les biens éligibles soient effectivement soumis à la taxe.
	\end{itemize}
	
   \item \textbf{La qualité du redevable légal}\\
	
   \noindent L’identification correcte de la personne imposable est une autre difficulté récurrente. Dans le contexte foncier togolais, marqué par la coexistence de plusieurs régimes de propriété (pleine propriété, usufruit, concession, bail emphytéotique, copropriété, etc.), il est fréquent que la taxe soit assignée à la mauvaise personne ou qu’aucun redevable ne soit clairement identifié.\\
	Or, selon les principes fiscaux, le redevable légal de la taxe foncière est celui qui bénéficie du droit de propriété ou d’un droit réel assimilé, lui conférant la jouissance économique du bien.\\
	La distinction entre propriétaire apparent, usufruitier, concessionnaire ou locataire de longue durée est donc essentielle pour éviter les erreurs d’imposition et les litiges ultérieurs.\\
	
   \noindent La compétence de l’inspecteur du cadastre doit ainsi s’étendre à :
	\begin{itemize}[font=\color{magenta} \Large, label=\ding{109}]
		\item  la lecture critique des titres fonciers et actes notariés ;
	
	   \item la vérification des mutations ou transferts de propriété ;
	
		\item et la mise à jour régulière des fichiers d’assiette pour refléter la réalité juridique des biens.
	\end{itemize}
	
	\item \textbf{La bonne application des exonérations}\\
	
    \noindent Les exonérations constituent une composante importante du régime fiscal foncier, mais également une source fréquente d’erreurs et d’incohérences administratives.\\
	Elles peuvent être permanentes (biens de l’État, propriétés diplomatiques, édifices religieux, établissements d’enseignement publics, etc.) ou temporaires (constructions neuves, logements sociaux, projets d’intérêt public, etc.).\\
	
	\noindent Cependant, dans la pratique, leur application est souvent marquée par :
	\begin{itemize}[font=\color{magenta} \Large, label=\ding{109}]
	\item une absence de contrôle des conditions d’éligibilité (par exemple, un bien déclaré cultuel mais utilisé à des fins commerciales) ;
	
	\item une prolongation illégale d’exonération temporaire faute de mise à jour du fichier fiscal ;
	
	\item ou encore une interprétation abusive des textes par certains contribuables ou agents.
    \end{itemize}
	La maîtrise des dispositions légales et des circulaires ou notes d’application est donc essentielle pour garantir une application juste et équilibrée des exonérations.\\
	L’inspecteur doit savoir vérifier les conditions d’octroi, la durée de validité et la procédure de retrait, tout en assurant une traçabilité rigoureuse des décisions d’exonération dans les registres cadastraux et fiscaux.\\
	
	\item \textbf{La coordination entre services cadastraux, fiscaux et communaux}\\
	
	\noindent La fiscalité foncière repose sur une chaîne d’acteurs dont l’efficacité dépend de la qualité de la coordination inter-institutionnelle.\\
	Dans le système togolais, le cadastre produit la donnée géographique et technique ; la fiscalité applique la règle juridique et procède à la liquidation ; les collectivités locales bénéficient du produit et participent à la sensibilisation et au recouvrement.\\
	Cependant, des faiblesses subsistent dans la circulation de l’information : retards dans la transmission des fiches d’évaluation, absence d’harmonisation des bases de données, ou doublons dans les fichiers communaux et fiscaux.
	Pour pallier ces lacunes, l’inspecteur du cadastre doit non seulement maîtriser les règles techniques de son domaine, mais aussi comprendre le fonctionnement administratif global du cycle de la taxe foncière.\\
	
	\noindent Il doit être capable de :
	\begin{itemize}[font=\color{magenta} \Large, label=\ding{109}]
	\item participer efficacement aux réunions de coordination inter-services ;
	
	 \item assurer la mise à jour continue des fichiers d’assiette ;
	
	\item et veiller à la concordance entre les données cadastrales, fiscales et financières (budgétaires).
    \end{itemize}
    
	\item \textbf{L’intégration pratique et la pédagogie de terrain}\\
	
	\noindent Un inspecteur du cadastre compétent ne se limite donc pas à la connaissance théorique des textes : il doit être capable de lire, interpréter et appliquer la norme fiscale avec rigueur sur le terrain.\\
	La mise en œuvre correcte des règles suppose des procédures standardisées, sécurisées et traçables, depuis la collecte de la donnée foncière jusqu'à’à l’émission du rôle d’imposition.\\
	C’est pourquoi le présent cours accorde une place prépondérante à la pédagogie active et à la formation par la pratique.\\
	
	\noindent Chaque partie du programme est accompagnée de :
	\begin{itemize}[font=\color{magenta} \Large, label=\ding{109}]
	\item cas concrets inspirés de situations rencontrées dans les communes togolaises ;
	
	\item exercices de simulation sur la détermination du redevable, la liquidation ou l’application d’exonérations ;

	\item et études de rôles d’imposition permettant de comprendre l’articulation entre les services techniques et fiscaux.
    \end{itemize}
	Cette approche pragmatique vise à développer chez les auditeurs une autonomie décisionnelle et une sécurité juridique dans leurs interventions, tout en les préparant à participer activement à la modernisation de la fiscalité foncière au Togo.
	
\end{itemize}

\section*{Contenu et structuration du cours}

Le programme du présent cours est structuré en \textbf{cinq chapitres complémentaires}, correspondant aux principales étapes du cycle de gestion de la taxe foncière au Togo. 
Cette organisation permet une progression logique : des \textit{fondements juridiques et institutionnels} vers la \textit{maîtrise des mécanismes d’imposition, de recouvrement et de contrôle}, tout en s’appuyant sur des \textbf{cas pratiques contextualisés} pour renforcer l’apprentissage.

\subsection*{Chapitre 1 -- Fondements, cadre légal et règles communes}

Ce premier chapitre pose les \textbf{bases juridiques, fiscales et administratives} de la taxe foncière. 
Il définit la nature de cet impôt, ses principes directeurs et son insertion dans l’architecture fiscale nationale. 
Les auditeurs y apprendront à :
\begin{itemize}[font=\color{magenta} \normalsize, label=\ding{109}]
	\item comprendre la \textbf{nature juridique} de la taxe foncière comme impôt réel, local et annuel ;
	\item maîtriser les \textbf{principes constitutionnels et fiscaux} qui fondent son existence (égalité devant l’impôt, territorialité, capacité contributive, etc.) ;
	\item identifier le \textbf{champ d’application matériel et territorial}, c’est-à-dire les biens et zones concernés par la taxe ;
	\item déterminer le \textbf{fait générateur}, la \textbf{période d’imposition} et les \textbf{conditions d’assujettissement} ;
	\item distinguer les \textbf{propriétés imposables, non imposables et exonérées} selon les textes en vigueur.
\end{itemize}
Ce chapitre constitue le socle de référence du cours, indispensable à la compréhension des chapitres suivants.

\subsection*{Chapitre 2 -- Les taxes foncières sur les propriétés bâties}

Le deuxième chapitre est consacré à la \textbf{taxe foncière sur les propriétés bâties (TFPB)}, qui constitue la principale source du produit foncier. 
Les participants y verront comment s’articulent les notions de bien imposable, de redevable légal et de base imposable, à travers :
\begin{itemize}[font=\color{magenta} \normalsize, label=\ding{109}]
	\item la définition et le \textbf{champ d’application fiscal du bâti} (immeubles achevés, constructions mixtes, dépendances, etc.) ;
	\item l’identification des \textbf{personnes imposables} (propriétaire, usufruitier, emphytéote, concessionnaire) et leurs obligations déclaratives ;
	\item les \textbf{règles de liquidation de la taxe}, à partir de la valeur locative cadastrale (VLC) et de la base nette imposable (BNI) ;
	\item l’application des \textbf{taux légaux};
	\item et enfin, la \textbf{gestion des exonérations spécifiques} (bâtiments publics, logements sociaux, constructions neuves, etc.).
\end{itemize}
Des \textbf{cas pratiques de calcul de la TFPB} permettront aux auditeurs de s’exercer sur des situations variées rencontrées dans les communes togolaises.

\subsection*{Chapitre 3 -- Les taxes foncières sur les propriétés non bâties}

Ce chapitre aborde la \textbf{taxe foncière sur les propriétés non bâties (TFPNB)}, qui vise les terrains nus, agricoles ou constructibles. 
Il met l’accent sur les différences fondamentales entre le régime fiscal des terrains et celui des constructions, en abordant :
\begin{itemize}[font=\color{magenta} \normalsize, label=\ding{109}]
	\item la \textbf{définition du terrain non bâti} au sens du Code Général des Impôts ;
	\item la \textbf{classification fonctionnelle} des sols (terrains agricoles, constructibles, spéciaux, zones d’extraction, etc.) ;
	\item la \textbf{détermination du redevable} (propriétaire, exploitant, usufruitier, détenteur d’un droit réel) ;
	\item la \textbf{liquidation de la taxe}, fondée sur des barèmes administratifs ou la valeur vénale du terrain ;
	\item et l’application des \textbf{exonérations propres aux terrains agricoles ou d’utilité publique}.
\end{itemize}
Des \textbf{exemples réels et études de cas} (zones périurbaines, lotissements, terres agricoles) illustreront la diversité des situations d’imposition.





