\documentclass[12pt,a4paper]{article}
\usepackage[utf8]{inputenc}
\usepackage[T1]{fontenc}
\usepackage[french]{babel}
\usepackage{lmodern}
\usepackage{geometry}
\geometry{margin=2.3cm}
\usepackage{amsmath,amssymb}
\usepackage{graphicx}
\usepackage{booktabs}
\usepackage{tabularx}
\usepackage{array}
\usepackage{enumitem}
\usepackage{hyperref}
\hypersetup{colorlinks=true,linkcolor=black,urlcolor=blue}
\setlist[itemize]{left=0pt,label=--,itemsep=3pt,topsep=3pt}
\setlist[enumerate]{itemsep=4pt,topsep=4pt}

% Petites aides typographiques
\newcommand{\code}[1]{\texttt{#1}}
\newcommand{\points}[1]{\hfill\textit{[#1 pts]}}

% Symbole et macro pour la partie entière (floor)
\newcommand{\floor}[1]{\left\lfloor #1 \right\rfloor}

% Raccourcis math
\newcommand{\Tref}{T_{\mathrm{ref}}}
\newcommand{\Caff}{C_{\mathrm{aff}}}
\newcommand{\Ca}{C_{a}}
\newcommand{\VL}{\mathrm{VL}}
\newcommand{\VR}{\mathrm{V_{ret}}}
\newcommand{\VD}{\mathrm{V_{d\acute ecl}}}
\newcommand{\Base}{\mathrm{Base}}
\newcommand{\TFPB}{\mathrm{TFPB}}
\newcommand{\TFPNB}{\mathrm{TFPNB}}
\newcommand{\THs}{\mathrm{TH}}

% Titre
\title{\Large\bfseries Cas Pratique Intégrateur\\
	\large Évaluation cadastrale, champ TFPB/TFPNB, bases et impôts, TH et changements}
\author{Module : Fiscalité foncière}
\date{}

\begin{document}
	\maketitle
	\noindent\textbf{Rappels utiles (méthode simplifiée).}
	\begin{itemize}
		\item \(\VL_{\text{bâti}} = \Tref \times \Caff \times \Ca \times 12\).
		\item \(\VL_{\text{non bâti}} = \text{Superficie} \times \Tref\) (tarif/m\(^2\)).
		\item \(\VD = \) (loyer mensuel total) \(\times 12\). En l'absence de loyer déclaré : \(\VD=\VL\).
		\item \(\VR = \max(\VL,\VD)\).
		\item Base d'imposition : \(\Base = 0{,}5\times \VR\) (bâti) ; \(\Base = 1{,}0\times \VR\) (non bâti).
		\item Impôt estimé : \(\TFPB = \Base \times 7{,}5\%\) (bâti) ; \(\TFPNB = \Base \times 0{,}5\%\) (non bâti).
	\end{itemize}
	
	\medskip
	\noindent\textbf{Indications pour \(\Ca\) (extraits opérationnels) :}
	\[
	\begin{aligned}
		\text{Bâtiment à étage (}n\ge 1\text{)}:\quad 
		&\Ca = 0{,}5\,(n+1) + 0{,}25\cdot \floor{\tfrac{|\text{Surf}-600|}{600}}\\
		\text{Villa/maison simple}:\quad 
		&\Ca = 1 + 0{,}25\cdot \floor{\tfrac{|\text{Surf}-600|}{600}}\\
		\text{Hangar/baraque}:\quad 
		&\Ca = 1 + \floor{\tfrac{|\text{Surf}-150|}{150}}
	\end{aligned}
	\]
	\vspace{0.5em}
	\noindent\textbf{Repères :} Les valeurs de \(\Tref\) (mensuelles pour le bâti, unitaires/m\(^2\) pour le non bâti) et les \(\Caff\) sont \textit{fournies au sujet} et tiennent compte du secteur tarifaire et de l’affectation.
	
	\bigskip
	
	%==================== PARTIE 1 ====================%
	\section*{Partie 1 --- Évaluation cadastrale \& Champ d’application }
	
	\subsection*{Exercice 1 : Identification du régime fiscal et calcul de \(\VL\)}
	On considère trois biens appartenant à un même contribuable.
	
	\medskip
	\noindent\textbf{Données communes} :\\
	\(\Caff\) (référence) : Habitation \(=1{,}00\) ; Commerce \(=1{,}30\) ; Profession libérale \(=1{,}20\).\\
	\(\Tref\) (mensuel, par type et zone) :
	\begin{itemize}
		\item Immeuble à étage (zone urbaine standard) : \(\Tref=70\,000\) F
		\item Commerce de quartier (rez-de-chaussée) : \(\Tref=90\,000\) F
		\item Logement (appartement) : \(\Tref=50\,000\) F
		\item Terrain nu (valeur vénale moyenne par m\(^2\)) : \(\Tref=8\,000\) F/m\(^2\)
	\end{itemize}
	
	\medskip
	\noindent\textbf{Biens :}
	\begin{center}
		\renewcommand{\arraystretch}{1.2}
		\begin{tabularx}{\linewidth}{@{}>{\raggedright\arraybackslash}p{1.8cm} X >{\raggedleft\arraybackslash}p{2.5cm} >{\raggedleft\arraybackslash}p{3.4cm} >{\raggedright\arraybackslash}p{3.2cm}@{}}
			\toprule
			\textbf{Bien} & \textbf{Description} & \textbf{Superficie} & \textbf{Loyer déclaré (mensuel)} & \textbf{Affectation/ \(\Tref\)}\\
			\midrule
			A & Immeuble R+2, 6 appartements (Hedzranawoé) & 720 m\(^2\) & 200\,000 F (global) & Habitation ; \(\Tref=70\,000\) F \\
			B & Terrain nu non loti (Kpalimé périphérie) & 600 m\(^2\) & -- & Terrain ; \(\Tref=8\,000\) F/m\(^2\) \\
			C & Boutique + logement derrière (Agoè-Sogbossito) & 450 m\(^2\) & 90\,000 F (boutique) ; 40\,000 F (logement) & Commerce \(\Tref=90\,000\) F ; Habitation \(\Tref=50\,000\) F \\
			\bottomrule
		\end{tabularx}
	\end{center}
	
	\medskip
	\noindent\textbf{Travail demandé.}
	\begin{enumerate}
		\item Identifier, pour chaque bien, s’il relève de la \(\TFPB\), de la \(\TFPNB\) et/ou de la \(\THs\). 
		\item Déterminer l’unité d’évaluation et calculer \(\Ca\) (utiliser la formule correspondant au type). 
		\item Calculer \(\VL\) (ou valeur vénale pour le non bâti) avec les \(\Tref\) et \(\Caff\) fournis. 
		\item Calculer \(\VD\) et retenir \(\VR=\max(\VL,\VD)\)
	\end{enumerate}
	
	\bigskip
	
	%==================== PARTIE 2 ====================%
	\section*{Partie 2 --- Bases d’imposition, impôts estimés et TH}
	
	\subsection*{Exercice 2 : Base imposable et \(\TFPB/\TFPNB\)}
	À partir de \(\VR\) de la Partie 1, calculer pour chaque bien : \(\Base\), puis \(\TFPB\) ou \(\TFPNB\).
	
	\subsection*{Exercice 3 : Taxe d’habitation (TH)}
	Sur le Bien A, au 1\textsuperscript{er} janvier, 1 appartement est occupé par le propriétaire et 5 sont loués.
	\begin{enumerate}
		\item Qui est redevable de la \(\THs\) et pour quelles unités ?
		\item Sur quelle base la \(\THs\) est-elle liquidée (rappels réglementaires)~? 
		\item Déterminer la \(\THs\) due pour les 5 locataires (faire les hypothèses raisonnables et \emph{explicites} si besoin).
	\end{enumerate}
	
	\bigskip
	
	%==================== PARTIE 3 ====================%
	\section*{Partie 3 --- Mutations et changements}
	
	\subsection*{Exercice 4 : Changement d’affectation et divisions}
	Au 1\textsuperscript{er} juillet, le propriétaire :
	\begin{itemize}
		\item transforme \textbf{deux appartements} du Bien A en \textbf{bureaux (activité libérale)} ;
		\item engage la viabilisation du Bien B en vue de \textbf{division et cession par lots}.
	\end{itemize}
	Questions :
	\begin{enumerate}
		\item Quelles obligations déclaratives (délais et fondement) s’appliquent au changement d’affectation et aux travaux (rappel de l’article 87) ? 
		\item Le \(\Caff\) change-t-il pour les deux nouvelles unités en activité libérale~? Justifier et recalculer \(\VL\) de ces unités en indiquant \(\Tref\), \(\Caff\) et \(\Ca\) pertinents. 
		\item Principe de date d’effet des changements : à partir de quand l’imposition \(\TFPB/\TFPNB/\THs\) est-elle modifiée~? Expliquer le rôle de la règle du 1\textsuperscript{er} janvier.
		\item Pour le terrain B en cours de division : \(\TFPNB\) reste-t-elle due~? Par qui~? Jusqu’à quel événement~? 
	\end{enumerate}
	
\end{document}
