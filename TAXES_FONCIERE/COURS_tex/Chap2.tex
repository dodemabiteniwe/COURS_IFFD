\chapter{La Taxe Foncière sur les Propriétés Bâties (TFPB)}
\section{Introduction}
La Taxe Foncière sur les Propriétés Bâties (TFPB) constitue théoriquement l’un des impôts locaux les plus importants pour les collectivités territoriales. Pourtant, les données fiscales récentes montrent qu’elle demeure faiblement mobilisée, avec un rendement oscillant entre 984 millions et 1,59 milliard entre 2019 et 2024, alors que le produit total des recettes locales dépasse 22 milliards en 2024. Cette situation révèle un écart important entre le potentiel réel de la taxe foncière et son niveau d’exploitation actuel.\\

\noindent Ce faible rendement s’explique en partie par la nature même du système fiscal togolais, qui repose largement sur l’auto-déclaration et l’auto-liquidation. Une construction non déclarée, un changement d’usage non signalé ou une mise à jour non effectuée prive immédiatement la collectivité d’une part de son assiette. La performance de la taxe foncière dépend donc directement de la qualité des déclarations des contribuables, mais surtout de la capacité de l’administration à contrôler, corriger et fiabiliser ces informations.\\

\noindent Dans ce contexte, l’inspecteur du cadastre joue un rôle stratégique : il est le garant de l’exactitude de l’assiette, de l’identification des constructions imposables et de la régularité des mises à jour. Le présent chapitre vise à renforcer cette compétence en présentant les règles essentielles de la TFPB : champ d’application, identification du bâti, obligations déclaratives, redevables, exonérations et principes de liquidation. Il s’agit de fournir les bases techniques et juridiques nécessaires pour améliorer durablement la mobilisation de la taxe foncière dans un système fiscal auto-déclaratif.

\section{Champ d’application de la TFPB}

\subsection{Définition générale de la propriété bâtie}

La Taxe Foncière sur les Propriétés Bâties repose sur une définition précise de ce qu’est un « bâti » au sens fiscal. Une construction est considérée comme une propriété bâtie lorsqu’elle réunit quatre caractéristiques essentielles :

\begin{itemize}[font=\color{magenta} \normalsize, label=\ding{109}]
	\item une fixation au sol à perpétuelle demeure,
	\item une durabilité réelle ou supposée,
	\item un résultat d’un travail humain,
	\item une aptitude à procurer un usage ou un avantage économique.
\end{itemize}

Ces quatre éléments constituent le socle juridique permettant de distinguer les constructions imposables des installations provisoires ou temporaires. Ils doivent être compris de manière fonctionnelle et non strictement architecturale, car l’objectif de la TFPB est de saisir la richesse immobilière stable présente sur le territoire.\\

\noindent Une propriété est dite « bâtie » lorsque sa présence sur le sol présente un caractère permanent. Une maison solidement construite en matériaux définitifs, un immeuble collectif, un atelier industriel ou un bâtiment commercial constituent naturellement des propriétés bâties. Mais la fiscalité foncière ne se limite pas à ces cas évidents : toute installation fixe, durable et issue d’un travail humain peut être considérée comme bâtie dès lors qu’elle est en mesure d’être utilisée ou d’apporter une utilité.

\subsubsection{La fixation au sol}

La fixation au sol est le premier critère de qualification du bâti. Une construction est considérée comme fixée au sol lorsqu’elle est intégrée de manière stable ou permanente à la parcelle, c’est-à-dire :

\begin{itemize}[font=\color{magenta} \footnotesize, label=\ding{109}]
	\item qu’elle repose sur des fondations ou un socle en dur,
	\item qu’elle est scellée au béton, soudée ou maçonnée,
	\item qu’elle ne peut être déplacée sans transformation, destruction ou altération de sa structure.
\end{itemize}

\noindent Ce critère permet de distinguer le bâti des installations simplement posées sur le sol. Ainsi :
\begin{itemize}[font=\color{magenta} \small, label=\ding{219}]
	\item une maison construite en parpaings est clairement fixée au sol ;
	\item un hangar en métal scellé dans des plots béton est également considéré comme bâti ;
	\item à l’inverse, un kiosque posé sur des madriers ou un conteneur reposant sans scellement conservent leur caractère mobile, donc non imposable au titre de la TFPB.
\end{itemize}

\noindent Cette distinction est essentielle dans les contextes urbains et périurbains togolais où l’on rencontre de nombreuses installations hybrides ou semi-mobiles. C’est au regard du mode de fixation que l’inspecteur tranche.

\subsubsection{La durabilité ou permanence}

Une construction doit également présenter une durabilité suffisante. La durabilité renvoie à la vocation du bien à demeurer dans le temps. Ainsi, une construction réalisée en matériaux solides et destinée à un usage prolongé est imposable, même si elle est de faible superficie. À l’inverse, une installation temporaire, un abri de chantier ou une structure conçue pour un événement particulier n'entre pas dans l’assiette de la TFPB.\\
Ce critère est fondamental dans le contexte togolais, où l’on rencontre fréquemment des constructions hybrides. L’inspecteur doit distinguer ce qui est réellement éphémère de ce qui, malgré son apparence, constitue un bien immobilier pérenne.\\

\noindent Ce critère implique que :
\begin{itemize}[font=\color{magenta} \small, label=\ding{219}]
	\item un magasin construit en matériaux définitifs possède un caractère durable ;
	\item une plateforme bétonnée constitue un ouvrage permanent ;
	\item une paillote démontable, à l’inverse, est un ouvrage éphémère destiné à disparaître ou à être déplacé.
\end{itemize}

\noindent La durabilité s’apprécie à la fois par la nature des matériaux (béton, brique, tôle fixée, charpente solide) et par l’intention apparente de permanence.

\subsubsection{Le résultat d’un travail humain}

Ce critère permet d’éliminer de l’assiette les éléments naturels (grottes, rochers, cavités), même lorsqu’ils sont utilisés économiquement. Une construction est un bâti lorsqu’elle résulte d’une intervention humaine volontaire, même minimale.

\noindent Cela inclut :

\begin{itemize}[font=\color{magenta} \small, label=\ding{219}]
	\item les constructions classiques (maisons, immeubles, bureaux),
	\item les ouvrages maçonnés (puits, latrines, garages),
	\item les plateformes ou dallages créés par l’homme,
	\item les dépendances et annexes bâties.
\end{itemize}

\noindent La finalité est de s’assurer que la valeur imposable provient bien d’un investissement humain qui crée une plus-value économique et non d’un état naturel du terrain.

\subsubsection{L’aptitude à l’usage ou à l’avantage économique}

Il s’agit du critère le plus important en fiscalité. L’imposition ne repose pas sur l’occupation réelle, mais sur la capacité du bien à procurer un avantage, même potentiel.

\noindent Une construction est considérée comme bâtie dès qu’elle :

\begin{itemize}[font=\color{magenta} \small, label=\ding{219}]
	\item peut être habitée, louée ou utilisée,
	\item peut accueillir une activité économique (commerce, artisanat, industrie),
	\item procure un bénéfice indirect (stockage, bureaux, usage administratif).
\end{itemize}

\noindent Ce critère implique que :
\begin{itemize}[font=\color{magenta} \small, label=\ding{219}]
	\item un immeuble vide mais en bon état reste imposable ;
	\item un magasin non encore ouvert mais utilisable est taxé ;
	\item un atelier abandonné mais structurellement solide entre dans la TFPB ;
	\item un logement inoccupé n’est pas une cause d’exonération.
\end{itemize}

\noindent Cette notion d’aptitude à l’usage rejoint l’idée, défendue en doctrine, que la TFPB taxe la potentialité économique du bâtiment, et non son exploitation effective. C’est la capacité du bâtiment à produire un revenu ou une utilité qui justifie son imposition.

\subsection{La notion d’achèvement fiscal}

La qualification du bâti dépend aussi de son état d’achèvement fiscal, qui est distinct de l’achèvement technique.
Un bâtiment est considéré comme « achevé » fiscalement dès qu’il peut être utilisé pour l’usage auquel il est destiné, même si :

\begin{itemize}[font=\color{magenta} \small, label=\ding{219}]
	\item les finitions ne sont pas terminées,
	\item l’esthétique n’est pas aboutie,
	\item les installations ne sont pas complètes.
\end{itemize}

\noindent Ainsi :
\begin{itemize}[font=\color{magenta} \small, label=\ding{219}]
	\item une maison sans peinture demeure imposable si elle est habitable ;
	\item un entrepôt sans carrelage ou sans portes intérieures peut être utilisé pour du stockage ;
	\item un local commercial non encore ouvert mais terminé structurellement est imposable.
\end{itemize}

\noindent L’achèvement fiscal répond donc à une logique de fonctionnalité, et non de perfection architecturale.

\noindent Un bâtiment est considéré comme non achevé (donc non imposable) seulement lorsqu’il est objectivement inhabitable ou inutilisable : absence de toiture, murs non terminés, absence d'accès, structure non stabilisée.\\

\noindent La frontière entre achevé et non achevé étant parfois ténue, c’est l’inspecteur du cadastre qui doit exercer un jugement professionnel fondé sur l’observation directe et les principes administratifs internes.

\subsection{Les constructions assimilées aux propriétés bâties}

Certaines installations ne constituent pas des bâtiments au sens strict, mais elles sont fiscalement assimilées au bâti car elles présentent les caractéristiques suivantes : permanence, fixation au sol, usage économique, contribution à la valeur du bien.\\

\noindent Il s’agit notamment :

\begin{itemize}[font=\color{magenta} \footnotesize, label=\ding{109}]
	\item des piscines maçonnées,
	\item des terrasses et plateformes bétonnées,
	\item des garages et dépendances fixes,
	\item des hangars soudés ou scellés,
	\item des blocs sanitaires en dur,
	\item des chambres froides ou installations industrielles fixées.
\end{itemize}

\noindent Ces constructions constituent ce que l’on appelle l’annexe bâtie, qui doit être intégrée dans la base imposable du bien principal. Leur omission peut générer des écarts de VLC importants.

\subsection{Le local comme unité d’évaluation fiscale}

L’unité sur laquelle repose la taxation est le local. Un bâtiment peut contenir plusieurs locaux différents selon leur usage : habitation, commerce, atelier, bureau, stockage. Chaque local doit être identifié, décrit et classé séparément afin d'assurer une évaluation juste.\\

\noindent Cette approche garantit une meilleure prise en compte de la diversité des usages au sein d’un même bâtiment, ce qui est indispensable dans un contexte où les constructions mixtes (boutique en façade, chambres à l’arrière ou à l’étage, atelier sur la même parcelle) sont fréquentes.\\

\noindent Un bâtiment peut contenir un ou plusieurs locaux, chacun pouvant relever d’une affectation différente : habitation, commerce, atelier, bureau, local technique, espace cultuel, etc.\\

\noindent Le classement par local permet d’obtenir une assiette fiable, précise et conforme aux principes de la fiscalité réelle.

\subsection{Les bâtiments à usage mixte}

Dans de nombreux cas, un même bâtiment présente plusieurs affectations. L’administration fiscale exige alors une ventilation des surfaces selon les usages réellement constatés. Ainsi, une boutique au rez-de-chaussée et un appartement à l'étage doivent être évalués indépendamment.\\
Cette ventilation est indispensable pour appliquer les bons coefficients d’affectation et les bons taux d’imposition. Elle permet également de distinguer les parties éventuellement exonérées de celles qui ne le sont pas, notamment lorsque des institutions religieuses, éducatives ou sanitaires exploitent des espaces mixtes.\\

\noindent L’inspecteur doit donc ventiler les surfaces et évaluer chaque local séparément pour éviter les erreurs d’assiette.\\


\noindent La détermination du champ d’application de la TFPB repose sur une analyse rigoureuse et intégrée des éléments constitutifs du bâti, de son achèvement fiscal, de son usage réel et de son intégration dans la parcelle. Il s’agit d’une étape fondatrice qui conditionne tout le reste de la chaîne fiscale. La précision de l’assiette dépend directement de la capacité de l’inspecteur à reconnaître, qualifier et documenter les constructions, qu’elles soient évidentes, partiellement achevées ou assimilées au bâti.\\
En maîtrisant ces principes, l’inspecteur du cadastre garantit l’équité, l’efficacité et la sécurité juridique de la fiscalité foncière bâtie.


\section{Personnes imposables et obligations déclaratives}

La détermination de la personne imposable constitue un élément central du régime de la taxe foncière. Elle garantit l’efficacité du recouvrement et la sécurité juridique du rôle d’imposition. Dans un système fiscal largement fondé sur l’auto-déclaration et l’auto-liquidation, comme c’est le cas au Togo, la compréhension des règles d’assujettissement est essentielle : un bien mal rattaché à un redevable, ou une mutation non déclarée, entraîne une perte d’assiette et fragilise la mobilisation des recettes. Cette section expose les situations les plus courantes, mais aussi les configurations complexes rencontrées sur le terrain.

\subsection{Redevable au 1\textsuperscript{er} janvier : principe majeur}

En matière de taxe foncière, l’imposition obéit au principe de l’annualité : 

\begin{quote}
	\textit{La taxe foncière est due pour l’année entière par celui qui est propriétaire au 1\textsuperscript{er} janvier.}
\end{quote}

Ce principe stabilise l’assiette, sécurise l’élaboration des rôles et facilite le recouvrement. Il implique que toute modification intervenue après cette date n’a d’effet qu’à compter de l’année suivante.

Ainsi :
\begin{itemize}
	\item une vente intervenue en février n’affecte pas la taxe de l’année : le vendeur reste redevable pour toute l’année ;
	\item un acheteur ne devient redevable qu’au 1\textsuperscript{er} janvier de l’année suivante ;
	\item une démolition en cours d’année ne supprime pas la taxe due pour l’année en cours.
\end{itemize}

Ce principe met en évidence l’importance, pour l’administration, de disposer d’une information foncière mise à jour. Les mutations non déclarées constituent une cause fréquente d’incohérence entre la matrice cadastrale et la réalité du terrain. L’inspecteur du cadastre a donc la responsabilité de repérer et actualiser ces situations.

\subsection{Démembrement de propriété : l’usufruitier comme redevable}

Le démembrement de propriété constitue l’une des situations les plus sensibles en matière d’identification du redevable de la taxe foncière. Au Togo, comme dans la plupart des systèmes de droit civil, la propriété peut être divisée en deux composantes : la nue-propriété et l’usufruit. Le Code Général des Impôts (art. 260) intègre explicitement les usufruitiers parmi les personnes imposables, consacrant ainsi un principe fondamental :

\begin{quote}
	\textit{La taxe foncière doit être supportée par celui qui bénéficie de la jouissance économique du bien.}
\end{quote}


Cette règle, simple en apparence, emporte pourtant de nombreuses conséquences pratiques et juridiques.


\paragraph{pourquoi l’usufruitier est redevable ?}

\noindent En droit civil, la nue-propriété n’emporte aucun avantage immédiat pour celui qui la détient : le nu-propriétaire ne peut ni utiliser le bien, ni en tirer des revenus, ni en exercer une maîtrise économique concrète. À l’inverse, l’usufruitier dispose de l’usage et des fruits : il peut occuper le bien, le louer, le céder temporairement ou l’exploiter dans un cadre économique. Ainsi, même si le nu-propriétaire demeure le titulaire ultime du droit de propriété, c’est bien l’usufruitier qui concentre la valeur économique du bâtiment.\\

\noindent C’est pour cette raison que l’administration fiscale doit systématiquement inscrire l’usufruitier au rôle d’imposition, même si le nu-propriétaire prend volontairement en charge les dépenses d’entretien ou les obligations déclaratives. Sur le plan fiscal, seul l’usufruitier est considéré comme détenteur du revenu potentiel du bien ; la contribution doit donc reposer sur lui. Ce principe joue dans toutes les configurations : succession, donation avec réserve d’usufruit, partage familial, convention d’occupation, ou même usufruit établi en fait lorsque l’usage réel du bien révèle une jouissance exclusive et durable.

\paragraph{Spécificités administratives et identification du redevable}

\noindent Dans la pratique, l'administration fiscale doit inscrire systématiquement l’usufruitier au rôle, quels que soient les arrangements familiaux ou les prises en charge informelles. L’expérience montre que les successions non formalisées, les donations avec réserve d’usufruit ou les situations d’occupation exclusive exigent une attention particulière de l’inspecteur du cadastre.\\

\noindent Pour sécuriser l’assiette, il est indispensable de vérifier :

\begin{itemize}[font=\color{magenta} \footnotesize, label=\ding{109}]
	\item les actes notariés ou documents successoraux, notamment en présence d’un conjoint survivant ;

	\item la jouissance réelle du bien, lorsque des situations d’usufruit de fait apparaissent (occupation exclusive, exploitation régulière ou perception de loyers).
\end{itemize}
L’administration privilégie toujours la réalité d’usage sur les déclarations incomplètes ou les situations de fait non régularisées.\\

\noindent Ce principe de jouissance économique a également des conséquences sur les exonérations et les droits dérivés du bien : c’est l’usufruitier qui peut en bénéficier, à la condition que l’usage du bien corresponde effectivement aux conditions prévues par la loi. En cas de fin d’usufruit, ces avantages ne se transmettent pas automatiquement au nu-propriétaire, car l’exonération repose sur une situation personnelle et sur un usage effectif.

\subsection{Baux de longue durée et concessions domaniales}

Plusieurs formes juridiques confèrent à leur titulaire des droits réels assimilables à la propriété. Elles modifient le redevable légal de la taxe foncière et doivent être maîtrisées par l’inspecteur.

\paragraph{Baux emphytéotiques (18 à 99 ans).}
Ils confèrent au preneur un droit réel, transmissible et durable. Le redevable de la taxe est l’emphytéote.

\paragraph{Baux à construction.}
Le preneur, qui édifie des constructions et en détient la jouissance, devient redevable de la taxe.

\paragraph{Concessions domaniales.}
Certaines autorisations d’occupation du domaine public, notamment celles créant des droits réels, rendent le concessionnaire redevable si des constructions en dur sont édifiées. 

Dans ces situations, la fiscalité foncière suit la réalité du droit réel plutôt que la simple titularité administrative du terrain.



\subsection{Copropriété, indivision et personnes morales}

Les formes collectives de détention immobilière constituent un domaine particulièrement délicat pour l’administration fiscale. Contrairement à la pleine propriété individuelle, où le redevable est clairement identifié, la copropriété, l’indivision et les personnes morales introduisent une pluralité de sujets de droit pouvant partager la détention ou la jouissance du bien. L’inspecteur du cadastre doit, dans ces situations, s’appuyer sur les règles juridiques, les textes fiscaux et les pratiques administratives afin de garantir la sécurité du rôle d’imposition.

\subsubsection{La copropriété : imposition du syndicat des copropriétaires}

Lorsqu’un immeuble est soumis au régime de copropriété, l’administration fiscale n’a pas à déterminer la part exacte de chaque détenteur privatif. La taxe foncière porte sur l’ensemble immobilier considéré comme une unité économique. Le redevable légal est donc le \textbf{syndicat des copropriétaires}, représenté par le syndic désigné.\\

\noindent Ce mécanisme repose sur l’idée que le syndicat constitue une entité capable de recevoir les notifications et d’exécuter les obligations fiscales au nom de tous les copropriétaires. La contribution individuelle selon les tantièmes relève d’un rapport interne à la copropriété, sans incidence pour l’administration fiscale. La doctrine résume bien ce principe :

\begin{quote}
	« L’administration impose l’ensemble immobilier ; la répartition entre copropriétaires est un rapport privé qui n’affecte ni l’assiette ni le recouvrement. »
\end{quote}

\noindent Dans le contexte togolais, ce principe revêt une importance particulière, car de nombreuses copropriétés fonctionnent de manière informelle, sans syndic professionnel. L’inspecteur doit alors identifier la personne ou le groupe assurant la gestion effective de l’immeuble afin de garantir la réception des notifications.

\subsubsection{L’indivision : solidarité fiscale et choix administratif}

L’indivision, fréquente au Togo en raison des successions familiales non liquidées, constitue l’une des sources de complication les plus importantes pour l’identification du redevable. Contrairement à la copropriété, l’indivision n’a pas de personnalité morale propre : elle regroupe des individus exerçant un droit simultané et non fractionné sur le bien.\\

\noindent Afin d’éviter une dilution de responsabilité, l’administration dispose d’une liberté d’inscription :

\begin{itemize}[font=\color{magenta} \footnotesize, label=\ding{109}]
	\item inscription collective de tous les indivisaires ;
	\item inscription d’un seul d’entre eux, avec la mention \textit{« et indivisaires »}.
\end{itemize}

\noindent Dans tous les cas, la règle essentielle est la \textbf{solidarité fiscale} : chaque indivisaire peut être poursuivi pour la totalité de la taxe. Ce mécanisme protège l’assiette et assure l’efficacité du recouvrement. Comme l’expliquent Schmidt et Kornprobst, l’impôt foncier repose sur une « unité d’assiette » incompatible avec un éclatement des obligations entre co-détenteurs.\\

\noindent Plusieurs difficultés pratiques sont fréquentes au Togo : absence de titre formel lors de successions coutumières, pluralité d’occupants, conflits intra-familiaux ou impossibilité d’identifier un représentant unique. L’inspecteur doit, dans ces situations, déterminer l’occupant principal ou l’ayant droit administrativement identifiable afin de garantir l’efficacité de la notification.

\subsubsection{Les personnes morales : une unité de redevabilité autonome}

Les sociétés civiles, entreprises commerciales, coopératives, associations ou fondations peuvent être titulaires de droits réels sur des immeubles bâtis. Dans ce cas, la règle est simple : \textbf{la personne morale elle-même est redevable}, indépendamment de ses membres, associés ou dirigeants.\\

\noindent Ce système présente plusieurs avantages pour l’administration fiscale :

\begin{itemize}[font=\color{magenta} \footnotesize, label=\ding{109}]
	\item existence d’un interlocuteur unique,
	\item stabilité juridique de la personnalité morale,
	\item meilleure capacité contributive et traçabilité administrative.
\end{itemize}

\noindent La nature juridique de la personne morale (associative, civile, commerciale) n'a pas d'incidence sur l’assujettissement : seule compte la détention d’un \textbf{droit réel} sur l’immeuble. Comme le souligne Maurin, une personne morale exerçant un droit réel détient une maîtrise économique suffisante pour justifier l’imposition.\\

\noindent Certaines situations particulières exigent toutefois une vigilance accrue : immeubles appartenant à des sociétés dissoutes, constructions pérennes édifiées par des associations sur le domaine public, ou encore locaux exploités dans le cadre de baux commerciaux à longue durée. L’inspecteur doit alors distinguer clairement l’occupant contractuel du titulaire réel du droit foncier.\\

\noindent Copropriété, indivision et personnes morales exigent une maîtrise fine des règles juridiques et fiscales. Contrairement à la propriété individuelle, ces formes collectives reposent sur une multiplicité d’acteurs dont la responsabilité doit être clarifiée afin d’assurer la sécurité du rôle et l’efficacité du recouvrement. Dans toutes ces situations, l’administration applique un principe directeur : \textit{imposer l’entité la plus stable, la plus structurée et la plus aisément identifiable}, afin de préserver l’intégrité de l’assiette foncière.



\subsection{Obligations déclaratives du propriétaire}
Les articles 86 et 87 du LPF structurent les obligations des propriétaires, locataires principaux et gérants d’immeubles.\\

\noindent Dans un système auto-déclaratif comme le Togo, ces obligations déterminent la qualité de l’assiette fiscale.
\subsubsection{Déclaration annuelle obligatoire (Art. 86 LPF)}

L'article 86 du Livre des Procédures Fiscales institue une déclaration annuelle obligatoire à la charge des propriétaires, principaux locataires et gérants d'immeubles. Cette déclaration, à déposer au plus tard le 31 mai, constitue l'un des mécanismes essentiels du système auto-déclaratif togolais.\\

\noindent Elle doit indiquer :
\begin{itemize}[font=\color{magenta} \footnotesize, label=\ding{109}]
	\item l'identité des locataires, la consistance des locaux loués et les loyers pratiqués ;
	\item l'identité des occupants à titre gratuit ainsi que les locaux qu'ils occupent ;
	\item les locaux occupés par le déclarant lui-même ;
	\item les locaux vacants ;
	\item la superficie exacte des terrains imposables.
\end{itemize}

\paragraph{Utilité fiscale et administrative.}
Cette déclaration permet à l'administration de vérifier l'usage réel des locaux, de contrôler la consistance du bâti, d'actualiser les bases d'imposition et d'affiner la valeur locative cadastrale. Elle constitue une source d'information essentielle pour maintenir la cohérence entre l'assiette déclarée et la réalité du terrain.

\paragraph{Intérêt pour le cadastre.}
Le dispositif s'inscrit dans la logique du \textit{cadastre permanent}, tel que conceptualisé par la doctrine. L'administration confronte les informations déclarées aux constats de terrain, détecte les omissions, procède aux rectifications nécessaires et actualise la valeur locative cadastrale en fonction des transformations ou changements d'usage constatés.

\paragraph{Enjeux dans le contexte togolais.}
La déclaration annuelle revêt une importance accrue en raison des nombreuses occupations informelles, des changements d'usage non déclarés, de la sous-déclaration des loyers et des transformations non signalées. L'inspecteur du cadastre doit donc vérifier, croiser et corriger les données fournies, afin d'assurer la fiabilité de l'assiette foncière et l'efficacité de la taxe.

\subsubsection{Déclaration des constructions nouvelles et changements d’affectation (Art. 87 LPF)}

L'article 87 du Livre des Procédures Fiscales impose aux propriétaires une obligation déclarative spécifique dans un délai de quatre mois pour tout événement modifiant la consistance, la nature ou l’usage des immeubles. Cette obligation complète la déclaration annuelle prévue à l’article 86 et garantit une mise à jour continue de l’assiette foncière. \\ 
À défaut de déclaration, les constructions ou transformations sont réputées achevées et imposables au 1\textsuperscript{er} janvier de l’année suivant leur achèvement présumé.

\paragraph{A. Construction nouvelle.}
La construction nouvelle désigne tout bâtiment réalisé sur une parcelle jusque-là dépourvue de structure permanente. \\
\textit{Exemples} : édification d’une maison d’habitation, d’un bâtiment commercial ou d’un atelier artisanal. \\ 
Conséquences fiscales :  
\begin{itemize}[font=\color{magenta} \small, label=\ding{219}]
	\item fixation d’une nouvelle valeur locative cadastrale (VLC) ;
	\item perte possible d’exonérations temporaires en cas de non-déclaration.
\end{itemize}

\paragraph{B. Reconstruction.}
La reconstruction concerne la réédification d’un bâtiment existant, qu’il ait été démoli volontairement ou détruit accidentellement.  \\
\textit{Exemples }: remplacement d’une bâtisse vétuste par une structure moderne, reconstruction après sinistre. \\ 
Conséquences fiscales :  
\begin{itemize}[font=\color{magenta} \small, label=\ding{219}]
	\item recalcul de la VLC sur la base de la nouvelle structure ;
	\item taxation d’office possible en cas d’omission.
\end{itemize}

\paragraph{C. Extension.}
L’extension correspond à l’ajout d’un volume bâti augmentant la surface ou la capacité d’usage d’un bâtiment existant.\\  
\textit{Exemples} : ajout de chambres, construction d’un magasin accolé, agrandissement d’un atelier industriel. \\ 
Conséquences fiscales :  
\begin{itemize}[font=\color{magenta} \small, label=\ding{219}]
	\item intégration des extensions dans l’unité d’évaluation ;
	\item augmentation de la surface pondérée et de la VLC ;
\end{itemize}

\paragraph{D. Changement d’usage.}
Le changement d’usage correspond à la transformation de la destination du local sans modification physique substantielle.\\  
\textit{Exemples :} conversion d’une chambre en boutique, transformation d’une habitation en cabinet médical, reconversion d’un atelier en logement.  \\
Conséquences fiscales :  
\begin{itemize}[font=\color{magenta} \small, label=\ding{219}]
	\item modification du groupe d’affectation et des coefficients ;
	\item application éventuelle d’un taux d’imposition différent ;
\end{itemize}

\paragraph{E. Changement de consistance.}
Le changement de consistance englobe les modifications structurelles affectant la configuration du bâtiment sans constituer une extension au sens strict.  \\
\textit{Exemples :} surélévation d’un niveau, ajout d’une terrasse maçonnée, construction d’une annexe isolée, transformation d’un garage en pièce d’habitation. \\ 
Conséquences fiscales :  
\begin{itemize}[font=\color{magenta} \small, label=\ding{219}]
	\item modification des surfaces et recalcul de la surface pondérée ;
	\item   réévaluation de la VLC ;
\end{itemize}


\noindent En cas de non-déclaration dans les quatre mois, la transformation est réputée achevée et imposable au 1\textsuperscript{er} janvier de l’année suivante. L’administration peut alors fixer les bases d’office et, le cas échéant, appliquer des pénalités. Ce mécanisme vise à prévenir les tentatives de dissimulation ou de retard volontaire dans la déclaration des nouvelles constructions ou transformations.

\subsubsection{Portée du système auto-déclaratif}

Le système fiscal togolais repose largement sur l’auto-déclaration, ce qui signifie que l’administration n’a connaissance des constructions, des mutations ou des changements d’usage qu’à travers les informations volontairement transmises par les contribuables. En pratique, l’absence de déclaration entraîne presque automatiquement l’absence d’imposition et une perte directe de recettes pour les collectivités territoriales. Cette caractéristique engendre de nombreuses omissions : constructions achevées mais non signalées, usages modifiés sans notification, mutations juridiques non rapportées ou exonérations maintenues alors que les conditions n’en sont plus remplies.\\

\noindent Dans ce contexte, le rôle de l’inspecteur du cadastre devient essentiel. Il lui appartient de vérifier la sincérité des déclarations reçues, de confronter ces informations avec la réalité observée sur le terrain, d’identifier les transformations non déclarées, d’apprécier la cohérence des loyers déclarés avec les pratiques du marché, et, au besoin, de procéder à des rectifications ou inscriptions d’office afin de sécuriser l’assiette. Comme le rappellent Schmidt et Kornprobst, « le contrôle de l’assiette foncière est la véritable clé du rendement de la taxe ». Dans un système auto-déclaratif, ce principe prend une importance fondamentale : la performance de la taxe foncière dépend directement de la capacité de l’administration à vérifier, corriger et fiabiliser la donnée foncière.


\section{Base imposable et liquidation de la TFPB}

La liquidation de la Taxe Foncière sur les Propriétés Bâties (TFPB) repose sur deux éléments fondamentaux : la \textbf{valeur locative cadastrale (VLC)}, déterminée par le Cadastre, et la \textbf{base nette imposable (BNI)}, établie par l’Administration fiscale conformément aux règles du Code Général des Impôts (CGI). L’article 270 du CGI fixe le principe général selon lequel les propriétés bâties sont imposées en fonction de leur VLC, diminuée d’un abattement forfaitaire de 50\,\%.

La présente section décrit le processus complet d’établissement de l’assiette foncière et de liquidation de la TFPB, en intégrant les procédures d’identification, de délimitation et de détermination de la consistance cadastrale, essentielles à une évaluation correcte.

\subsection{La Valeur Locative Cadastrale (VLC)}

\subsubsection{Définition juridique (Art. 270 et 271 CGI)}

Selon l’article 270 du CGI, la VLC représente le \textit{loyer théorique annuel} que le bien pourrait produire au 1\textsuperscript{er} janvier de l’année d’imposition. Cette valeur est diminuée de 50\,\% afin de tenir compte des charges de gestion, d’assurances, d’entretien, de réparations et d’amortissement.  
L’article 271 définit les méthodes d’établissement de la VLC : recours aux baux authentiques ou loyers réels, comparaison, ou appréciation directe fondée sur la valeur vénale.

La VLC est une valeur administrative, objective et indépendante des loyers réellement perçus.

\subsubsection{ Identification et Délimitation}

Avant toute évaluation, l’inspecteur procède à l’identification et à la délimitation du bien. Cette phase garantit l’exhaustivité et la fiabilité de l’assiette fiscale.

\paragraph{Procédure d’identification}

L’identification consiste à associer une réalité physique (terrain + constructions) à une référence administrative cadastrale. Elle repose sur :
\begin{itemize}
	\item les déclarations du propriétaire (permis de construire, déclaration d’achèvement) ;
	\item les opérations de terrain (levées, vérifications, récolements) ;
	\item l’analyse d’images satellitaires, orthophotos ou données SIG (télédétection), permettant de détecter les omissions et constructions non déclarées.
\end{itemize}

Toute nouvelle construction identifiée conduit à la mise à jour du fichier propriétaire et du plan cadastral (attribution de référence parcellaire, modification de la matrice littérale).

\paragraph{Procédure de délimitation et mesure de la consistance}

La délimitation graphique distingue le \textbf{sol imposable à la TFPB} (sol dépendant du bâti) du sol restant imposable à la TFPNB.  
L’inspecteur mesure ensuite la consistance physique du bâtiment : surfaces pondérées, nombre de niveaux, matériaux, équipements, annexes. Ces éléments alimentent la fiche d’évaluation cadastrale et déterminent directement la VLC.

\subsubsection{Caractéristiques techniques de la VLC}

La VLC dépend de :
\begin{itemize}
	\item la surface pondérée ;
	\item l’état du bâtiment ;
	\item l’usage ou affectation ;
	\item la catégorie cadastrale ;
	\item les coefficients réglementaires (situation, affectation, entretien).
\end{itemize}

Elle constitue la base technique transmise par le Cadastre au service de liquidation.

\subsection{Détermination de la Base Nette Imposable (BNI)}

\subsubsection{Abattement légal}

L'impôt foncier, étant un impôt sur le revenu (le revenu théorique ou potentiel du bien), doit porter sur la richesse réelle restante après déduction des charges nécessaires à sa production. L'abattement est une déduction forfaitaire appliquée à la Valeur Locative Cadastrale (VLC).\\
 Il vise à couvrir l'ensemble des frais incompressibles que le propriétaire devrait supporter pour maintenir le bien en état de location et le gérer : frais d'entretien et de réparation, primes d'assurance, frais de gestion et de recouvrement, amortissement pour dépréciation (usure normale). L'Administration Fiscale recherche la Base Nette Imposable (BNI), c'est-à-dire le revenu foncier théorique net, et non le revenu brut (VLC)
 
 \subsubsection{Taux Fixé par le CGI (50\%)}
 Le taux d'abattement est strictement encadré par le Code Général des Impôts (CGI). Il s'agit d'une disposition de liquidation qui s'impose à l'Administration. Il est impératif de souligner que l'abattement est forfaitaire.\\
Cela signifie :
\begin{itemize}
	\item il s'applique indépendamment des frais réels supportés par le propriétaire;
	\item il ne se cumule pas avec la déduction des frais réels justifiés. 
\end{itemize}

\subsection{Taux d’imposition (Art. 275 CGI)}

L’article 275 du CGI fixe le taux légal de la TFPB à :

\[
7,5\,\% \ \text{du revenu net cadastral}
\]

Certaines collectivités peuvent appliquer un taux propre lorsque la loi leur confère une marge de modulation.

\subsection{Liquidation de la TFPB}

La liquidation est donc :

\[
\text{TFPB} = \text{BNI} \times 7,5\%
\]

ou encore, compte tenu de la BNI = VLC × 50\,\% :

\[
\text{TFPB} = \text{VLC} \times 50\% \times 7,5\%
\]

\subsection{Atelier pratique : Calcul complet de la TFPB}

\paragraph{Données.}
\begin{itemize}
	\item VLC Habitation : 650\,000 XOF
	\item VLC Commerce : 350\,000 XOF
\end{itemize}

\paragraph{Étape 1 : Calcul de la BNI.}
\[
\text{BNI}_H = 650000 \times 0.50 = 325000 \ \text{XOF}
\]
\[
\text{BNI}_C = 350000 \times 0.50 = 175000 \ \text{XOF}
\]

\[
\text{BNI}_{\text{Totale}} = 325000 + 175000 = 500000 \ \text{XOF}
\]

\paragraph{Étape 2 : Liquidation de la TFPB.}
\[
\text{TFPB} = 500000 \times 7,5\% = 37500 \ \text{XOF}
\]

\subsection{Paiement et solidarité (Art. 85 LPF)}

Selon l’article 85 du LPF :
\begin{itemize}
	\item la taxe est due pour l’année entière à compter du 1\textsuperscript{er} janvier ;
	\item elle est payable spontanément en deux échéances : 31 mai et 31 octobre ;
	\item le propriétaire du sol et le locataire titulaire d’un droit réel sont solidairement responsables du paiement.
\end{itemize}


\section{Exonérations spécifiques à la TFPB}


Les exonérations constituent des limites au principe d’imposition des propriétés bâties. Elles doivent être maniées avec rigueur, car leur mauvaise application entraîne des pertes fiscales importantes pour les communes et une rupture du principe d’égalité devant l’impôt.\\

\noindent Le CGI distingue :\\

\begin{itemize}[font=\color{magenta} \small, label=\ding{229}]
\item les exonérations permanentes (Art. 261) : attachées à la nature ou à l’usage du bien ;

\item les exonérations temporaires (Art. 262 à 267) : attachées à la construction ou à la transformation du bien.
\end{itemize}
L’inspecteur du cadastre doit en connaître les subtilités afin de sécuriser l’assiette.

\subsection{Exonérations permanentes (Art. 261 CGI)}

Les exonérations permanentes s’appliquent de plein droit, mais uniquement lorsque toutes les conditions légales sont strictement remplies. Dans la pratique, elles sont souvent mal comprises, ce qui entraîne des omissions injustifiées dans les rôles d’imposition.\\

\noindent Nous étudions ci-dessous chaque cas prévu par le CGI.\\


\begin{itemize} [font=\color{magenta} \Large, label=\ding{182}]
	\item \textbf{{\large Certaines propriétés publiques}}
\end{itemize}

L’article 261 du CGI exonère de la taxe foncière les propriétés appartenant à l’État, aux préfectures, aux communes ainsi qu’aux établissements et organismes publics, dès lors que ces immeubles sont \textbf{affectés à un service public ou d’utilité générale} et \textbf{non productifs de revenus}.  
Le régime étant de nature dérogatoire, son application doit être strictement encadrée.  
Trois conditions cumulatives doivent être remplies.

\begin{itemize} [font=\color{magenta} \normalsize, label=\ding{43}]
	\item \textbf{Première condition : appartenance à une personne publique.}
\end{itemize}

\noindent L’exonération ne concerne que les immeubles appartenant à l’État, aux préfectures, aux communes ou aux établissements et organismes publics.  
Elle s’articule avec le principe de territorialité de la taxe foncière : un immeuble appartenant à une collectivité territoriale n’est exonéré que lorsqu’il est situé sur le territoire de cette même collectivité. Ainsi, un immeuble relevant de la préfecture du Golfe et situé dans le Golfe est intégralement exonéré.  
En revanche, lorsqu’un immeuble public est situé hors du territoire de la collectivité propriétaire, il demeure redevable de la \textbf{part de taxe revenant à la collectivité d’implantation}, dès lors que celle-ci est de même nature.\\  
Par exemple, un immeuble appartenant à une commune et situé sur le territoire d’une autre commune doit acquitter la part de taxe revenant à cette dernière.

\begin{itemize} [font=\color{magenta} \normalsize, label=\ding{43}]
	\item \textbf{Deuxième condition : affectation à un service public ou à une utilité générale.}
\end{itemize}

\noindent L’exonération suppose que l’immeuble soit effectivement utilisé pour une mission de service public ou d’utilité collective.  
Cette affectation doit être réelle, actuelle et exclusive, ce qui exclut les bâtiments désaffectés, sous-utilisés ou utilisés à des fins mixtes.  
La nature dérogatoire de l’exonération impose également d’en limiter la portée : les organismes publics ayant un \textbf{caractère industriel ou commercial} (EPIC ou établissements exerçant des activités marchandes) ne peuvent pas en bénéficier, même lorsqu’ils sont investis d’une mission d’intérêt général.  
Seule la partie strictement nécessaire au service public peut être exonérée.


\begin{itemize} [font=\color{magenta} \normalsize, label=\ding{43}]
	\item \textbf{Troisième condition : absence totale de revenus.}
\end{itemize}

\noindent Pour bénéficier de l’exonération, le bien ne doit générer \textbf{aucun revenu}, quelle qu’en soit l’importance.  
Tout immeuble produisant un revenu, même modeste, entre dans le champ d’application de la taxe foncière : location d’une salle, boutiques annexes, mise à disposition d’espaces contre redevance, etc.  
L’exonération cesse également lorsque la perception de revenus est accessoire mais réelle.  
Ainsi, un centre public comprenant des boutiques louées doit être partiellement imposé pour les surfaces génératrices de revenus.

\paragraph{Réflexion doctrinale:}
Une difficulté classique réside dans la distinction entre \textit{absence de recherche de profit} et \textit{absence de revenus}.  
La doctrine fiscale enseigne que c’est l’absence de \textbf{revenus effectifs}, et non l’intention non lucrative, qui conditionne l’exonération.  
Un immeuble public générant des recettes, même si celles-ci sont réaffectées au financement ou à l’amélioration du service public, perd son caractère exonéré.  
Ainsi, un marché public louant des stands à des commerçants doit être imposé, même si les revenus perçus financent l’entretien du marché.  
L’exonération suppose donc une absence totale de revenus, jointe à une affectation exclusive au service public.


\begin{itemize} [font=\color{magenta} \Large, label=\ding{183}]
	\item \textbf{{\large Installations portuaires, fluviales ou aéroportuaires faisant l’objet d’une concession d’outillage public}}
\end{itemize}


Le deuxième alinéa de l’article 261 du CGI exonère de la taxe foncière les installations situées dans les ports maritimes, fluviaux ou aériens ainsi que sur les voies de navigation intérieure, dès lors qu’elles font l’objet d’une \textbf{concession d’outillage public} accordée par l’État et qu’elles sont exploitées conformément au cahier des charges.  
Cette exonération s’analyse à travers trois conditions cumulatives : localisation, concession et exploitation réglementaire.


\begin{itemize} [font=\color{magenta} \normalsize, label=\ding{43}]
	\item \textbf{Première condition : localisation dans un port, un aéroport ou une voie navigable.}
\end{itemize} 


\noindent L’exonération ne bénéficie pas à l’ensemble des immeubles situés dans une zone portuaire ou aéroportuaire ; elle vise exclusivement les installations techniques qui participent directement au fonctionnement du service public du transport : grues, portiques, hangars de manutention, silos, tours de contrôle, pipelines, etc.  
Un bâtiment administratif ou commercial établi dans l’enceinte portuaire ne peut en aucun cas être exonéré à ce titre.  
L’inspecteur doit donc apprécier l’usage réel de l’installation et non sa simple présence dans la zone portuaire ou aéroportuaire.


\begin{itemize} [font=\color{magenta} \normalsize, label=\ding{43}]
	\item \textbf{Deuxième condition : existence d’une concession d’outillage public.}
\end{itemize}

\noindent L’exonération ne s’applique qu’aux équipements appartenant à l’État et confiés à un concessionnaire dans le cadre d’un contrat d’outillage public.  
Une concession d’outillage public se définit comme un mécanisme par lequel l’État délègue l’exploitation d’équipements techniques indispensables au chargement et déchargement, à la manutention, au stockage ou à la sécurité des opérations portuaires ou aéroportuaires.  
Les immeubles appartenant à des opérateurs privés et ne faisant pas l’objet d’une concession accordée par l’État sont exclus du champ de l’exonération.

\begin{itemize} [font=\color{magenta} \normalsize, label=\ding{43}]
	\item \textbf{Troisième condition : exploitation conforme au cahier des charges.}
\end{itemize}

\noindent Les installations ne sont exonérées que si elles sont exploitées conformément aux obligations contractuelles fixées dans le cahier des charges.  
Toute activité commerciale indépendante ou accessoire, telle que des boutiques, bureaux, parkings payants ou entrepôts loués à des tiers, entraîne l’imposition des surfaces concernées.  
Une réaffectation de l’installation à un usage privé ou commercial met fin à l’exonération, même si l’infrastructure demeure dans la zone portuaire.

\paragraph{Synthèse doctrinale.}
L’exonération repose sur la contribution directe de l’installation à l’exploitation technique du service public du transport.  
Comme le souligne la doctrine, \textit{« ce n’est pas la localisation qui fonde l’exonération, mais la fonction de l’outillage public »}.  
Toute installation qui ne participe pas strictement et exclusivement au fonctionnement technique du port, de l’aéroport ou de la voie navigable doit être imposée.


\begin{itemize} [font=\color{magenta} \Large, label=\ding{184}]
	\item \textbf{{\large Ouvrages établis pour la distribution de l’eau potable ou de l’énergie électrique}}
\end{itemize}

Le troisième cas d’exonération prévu à l’article 261 du CGI concerne \textit{« les ouvrages établis pour la distribution de l’eau potable ou de l’énergie électrique et appartenant à l’État, à des préfectures ou à des communes »}.  
Cette disposition vise les infrastructures techniques indispensables au fonctionnement des réseaux publics de distribution.  
Comme il s’agit d’une exonération dérogatoire, son application doit être strictement limitée aux ouvrages expressément définis par la loi.

\begin{itemize} [font=\color{magenta} \normalsize, label=\ding{43}]
	\item \textbf{Première condition : la nature de l’ouvrage.}
\end{itemize}

\noindent Le terme \textit{ouvrage} doit être interprété dans un sens fonctionnel. Il désigne les installations techniques participant directement à la production, au transport ou à la distribution de l’eau et de l’électricité : stations de pompage, châteaux d’eau, réservoirs, conduites de distribution, transformateurs, postes électriques, lignes haute tension, bornes publiques, etc.  
Cette exonération ne concerne donc ni les bâtiments administratifs des opérateurs publics, ni les bureaux, magasins, ateliers, dépôts ou garages, même s’ils appartiennent à la même entité.  
De même, les bases logistiques, centres de facturation ou espaces commerciaux annexes ne peuvent être assimilés à des ouvrages techniques et demeurent imposables.

\begin{itemize} [font=\color{magenta} \normalsize, label=\ding{43}]
	\item \textbf{Deuxième condition : propriété publique.}
\end{itemize}

\noindent Les ouvrages doivent appartenir à l’État, à une préfecture ou à une commune. La propriété est l’élément déterminant de l’exonération.  
Lorsque la gestion de l’eau ou de l’électricité est confiée à une société privée dans le cadre d’une concession ou d’un partenariat public-privé, l’exonération ne s’applique qu’aux ouvrages demeurant la propriété de la personne publique.  
Inversement, les installations construites et conservées en pleine propriété par une société concessionnaire privée ne bénéficient pas de l’exonération, même si elles participent au service public.  
Les immeubles appartenant à des sociétés d’État ou à des établissements publics à caractère commercial ou industriel sont, pour leur part, imposables lorsqu’ils ne constituent pas des ouvrages techniques.

\begin{itemize} [font=\color{magenta} \normalsize, label=\ding{43}]
	\item \textbf{Troisième condition : affectation exclusive à la distribution.}
\end{itemize}

\noindent L’ouvrage doit être affecté exclusivement à la mission technique de distribution de l’eau ou de l’électricité.  
Toute utilisation mixte, détournée ou accessoire est de nature à remettre en cause l’exonération, au moins pour les surfaces concernées.  
Ainsi, une station de pompage comprenant un local commercial, un bureau administratif, un atelier privé ou un espace loué à un tiers perd l’exonération pour ces parties.  
L’affectation doit être réelle, continue et exclusive ; une installation désaffectée ou inutilisée doit être réintégrée dans l’assiette de la taxe foncière.

\paragraph{Synthèse doctrinale.}
L’exonération repose sur la nécessité de protéger les infrastructures essentielles du réseau public, tout en évitant les dérives consistant à étendre indûment l’exonération à des immeubles annexes qui ne participent pas directement à la mission technique.  
Comme le rappelle la doctrine, \textit{« un opérateur public ne peut étendre à l’ensemble de son patrimoine immobilier une exonération qui ne vise que les ouvrages du réseau lui-même »}.  
L’inspecteur doit donc distinguer strictement les installations techniques des bâtiments de gestion ou d’exploitation commerciale, vérifier l’usage réel des ouvrages et rétablir l’imposition chaque fois que les conditions légales ne sont pas réunies.



\begin{itemize} [font=\color{magenta} \Large, label=\ding{185}]
	\item \textbf{{\large Édifices servant à l’exercice public des cultes}}
\end{itemize}

Le quatrième cas d’exonération prévu à l’article 261 du CGI concerne « les édifices servant à l’exercice public des cultes ».  
Il s’agit d’une exonération ancienne, fondée sur la nécessité de préserver la liberté de culte et de soutenir les activités religieuses non lucratives.  
Comme toute mesure dérogatoire, elle doit être interprétée strictement.  
Trois éléments structurent son analyse : la définition de l’édifice cultuel, la nature de l’affectation et les limites de l’exonération.

\begin{itemize} [font=\color{magenta} \normalsize, label=\ding{43}]
	\item \textbf{Première condition : la nature de l’édifice cultuel.}
\end{itemize}

\noindent L’édifice doit être spécifiquement destiné à l’exercice public d’un culte organisé : églises, mosquées, temples, synagogues, chapelles ouvertes au public.  
Le critère déterminant est l’usage cultuel : le bâtiment doit accueillir régulièrement les fidèles pour la prière, les rites et les cérémonies religieuses.  
Un oratoire privé, une salle de réunion ou un bâtiment simplement dénommé « centre religieux » ne saurait être exonéré s’il n’abrite pas un culte public effectif.

\begin{itemize} [font=\color{magenta} \normalsize, label=\ding{43}]
	\item \textbf{Deuxième condition : l’affectation réelle, exclusive et continue.}
\end{itemize}

\noindent L’édifice doit être utilisé exclusivement et de manière continue pour la pratique du culte.  
Toute utilisation mixte ou accessoire de nature non cultuelle remet en cause l’exonération pour les surfaces concernées.  
Ainsi, un bâtiment religieux transformé en école privée, en centre de conférence ou en entrepôt ne peut plus bénéficier de l’exonération.  
Une chapelle désaffectée ou un édifice ne servant qu’occasionnellement au culte doit également être imposé.

\begin{itemize} [font=\color{magenta} \normalsize, label=\ding{43}]
	\item \textbf{Les limites de l’exonération.}
\end{itemize}


\noindent L’exonération est strictement cantonnée à l’édifice servant au culte.  
Les dépendances et annexes ne bénéficient pas automatiquement de l’exonération.  
Doivent notamment être imposés :
\begin{itemize} [font=\color{magenta} \small, label=\ding{219}]
	\item les logements de fonction des responsables religieux ;
	\item les salles polyvalentes louées à des tiers ;
	\item les écoles privées confessionnelles productives de revenus ;
	\item les boutiques, ateliers, parkings payants et locaux commerciaux attenants ;
	\item les maisons paroissiales affectées à l’habitation.
\end{itemize}
Dans le cas de salles multifonctionnelles utilisées à la fois pour des cérémonies religieuses et pour des activités lucratives, seule la partie strictement cultuelle peut être exonérée, lorsque la distinction est matériellement possible.

\paragraph{Synthèse doctrinale.}
L’exonération vise l’usage du bâtiment et non l’entité religieuse propriétaire.  
Comme le souligne la doctrine, « l’exonération des édifices cultuels ne s’attache ni à la religion ni au statut du propriétaire, mais à la fonction cultuelle exclusive de l’immeuble ».  
Toute activité génératrice de revenus ou tout usage non cultuel fait obstacle à l’exonération.  
Il appartient à l’inspecteur du cadastre de vérifier l’usage réel des bâtiments religieux et de rétablir l’imposition dès que les conditions strictes posées par la loi cessent d’être remplies.


\begin{itemize} [font=\color{magenta} \Large, label=\ding{186}]
	\item \textbf{{\large Immeubles à usage scolaire et universitaire non productifs de revenus}}
\end{itemize}

Le cinquième cas d’exonération prévu à l’article 261 du CGI concerne « les immeubles à usage scolaire et universitaire non productifs de revenus fonciers ».  
Cette disposition vise à soutenir la mission d’éducation, considérée comme essentielle à l’intérêt général.  
Son application repose toutefois sur deux conditions strictes : l’affectation de l’immeuble à l’enseignement et l’absence totale de revenus.

\begin{itemize} [font=\color{magenta} \normalsize, label=\ding{43}]
	\item \textbf{Première condition : affectation à un usage scolaire ou universitaire.}
\end{itemize}


\noindent L’immeuble doit être utilisé pour une activité éducative : enseignement général ou technique, formation professionnelle, enseignement supérieur, bibliothèques publiques, laboratoires pédagogiques ou centres universitaires.  
L’usage doit être réel et effectif.  
Un bâtiment simplement déclaré « scolaire » mais utilisé principalement pour des réunions, des services administratifs non liés à l’enseignement ou d’autres activités doit être imposé.

\begin{itemize} [font=\color{magenta} \normalsize, label=\ding{43}]
	\item \textbf{Deuxième condition : absence totale de revenus fonciers.}
\end{itemize}

\noindent L’exonération n’est applicable que si l’immeuble ne génère aucun revenu.  
Dès qu’un bâtiment scolaire comporte des locaux loués ou exploités à des fins lucratives — boutiques, cantines concédées, salles de réunion mises en location, logements de fonction payants — ces parties deviennent imposables.  
L’absence de revenus est le critère déterminant : un établissement éducatif poursuivant un but non lucratif peut être exonéré, mais un établissement privé payant opérant dans une logique commerciale ne le peut pas.

\begin{itemize} [font=\color{magenta} \normalsize, label=\ding{43}]
	\item \textbf{Distinction selon le statut des établissements.}
\end{itemize}

\noindent Le statut juridique de l’établissement n’est pas décisif ; seule compte la fonction et l’absence de revenus.
\begin{itemize}
	\item \textbf{Établissements publics :} exonérés de plein droit lorsque les locaux servent exclusivement à l’enseignement.
	\item \textbf{Établissements privés non lucratifs :} peuvent bénéficier de l’exonération si aucune activité génératrice de revenu n’est exercée.
	\item \textbf{Établissements privés lucratifs :} toujours imposables en raison de leur activité commerciale fondée sur la perception de frais de scolarité ou d’autres revenus.
\end{itemize}

\paragraph{Synthèse doctrinale.}
Cette exonération repose sur deux critères cumulatifs : l’usage éducatif exclusif et l’absence totale de revenus.  
Comme le souligne la doctrine, « la finalité éducative ne suffit pas : l’absence de revenus est la condition déterminante de l’exonération ».  
Ainsi, une école publique utilisée exclusivement pour l’enseignement sera exonérée, tandis qu’une école privée payante ou un établissement scolaire comportant des locaux loués sera imposé, au moins sur les surfaces concernées.  
Il appartient à l’inspecteur du cadastre de vérifier l’usage réel des immeubles scolaires, d’identifier les dépendances génératrices de revenus et de procéder, le cas échéant, à une imposition partielle.


\begin{itemize} [font=\color{magenta} \Large, label=\ding{187}]
	\item \textbf{{\large Immeubles affectés à des œuvres d’assistance médicale ou sociale}}
\end{itemize}

Le sixième cas d’exonération prévu à l’article 261 du CGI concerne « les immeubles affectés à des œuvres d’assistance médicale ou sociale ».  
Cette disposition vise à favoriser les institutions œuvrant dans l’intérêt général en matière de santé, de solidarité et de protection des personnes vulnérables.  
Comme toute exonération dérogatoire, elle doit être strictement interprétée et appliquée avec rigueur.

\begin{itemize} [font=\color{magenta} \normalsize, label=\ding{43}]
	\item \textbf{Première condition : la nature de l’œuvre.}
\end{itemize}

\noindent L’établissement doit poursuivre une mission d’assistance médicale ou sociale.  
Relèvent notamment de cette catégorie : les dispensaires gratuits, centres d’accueil d’enfants vulnérables, maisons de retraite non lucratives, centres nutritionnels, foyers d’accueil, établissements médico-sociaux gérés par des œuvres caritatives ou humanitaires.  
En revanche, les cliniques privées, cabinets médicaux lucratifs ou structures hospitalières commerciales, même s’ils dispensent des soins, ne remplissent pas la condition d’assistance et demeurent imposables.

\begin{itemize} [font=\color{magenta} \normalsize, label=\ding{43}]
	\item \textbf{Deuxième condition : affectation exclusive à l’assistance.}
\end{itemize}

\noindent L’immeuble doit être affecté exclusivement et de manière continue à une activité d’assistance.  
Toute utilisation mixte ou partielle pour des activités non assistancielles entraîne l’imposition des surfaces concernées.  
Ainsi, lorsqu’une organisation humanitaire utilise un bâtiment à la fois pour un dispensaire gratuit et pour des bureaux administratifs à vocation commerciale, seule la partie strictement dédiée à l’assistance peut être exonérée.  
Les logements du personnel, lorsqu’ils constituent un avantage ou une source indirecte de revenus, n’entrent pas dans le champ de l’exonération.

\begin{itemize} [font=\color{magenta} \normalsize, label=\ding{43}]
	\item \textbf{Troisième condition : absence totale de revenus.}
\end{itemize}

\noindent L’exonération est subordonnée à l’absence de tout revenu foncier.  
Sont donc exclus du bénéfice de l’exonération les immeubles ou parties d’immeubles générant des loyers, redevances, paiements pour hébergement, exploitations commerciales ou activités lucratives annexes.  
Ainsi, un centre médical ou social percevant des frais d’hébergement, une contribution financière obligatoire ou des revenus accessoires doit être imposé pour la partie concernée.  
L’absence de lucidité dans le contrôle de ces revenus peut entraîner des pertes importantes pour l’assiette fiscale.

\paragraph{Synthèse doctrinale.}
Cette exonération repose sur deux critères décisifs : la finalité d’assistance et l’absence de revenus.  
Comme le souligne la doctrine, « l’assistance sociale ne se confond pas avec la prestation médicale marchande : seule l’activité non lucrative ouvre droit à l’exonération ».  
L’inspecteur du cadastre doit donc distinguer précisément les activités assistancielles des activités commerciales, vérifier l’usage réel de l’immeuble, identifier les surfaces productives de revenus et, lorsque nécessaire, procéder à une imposition partielle.


\begin{itemize} [font=\color{magenta} \Large, label=\ding{188}]
	\item \textbf{{\large Bâtiments et installations servant aux exploitations rurales pour loger les animaux ou serrer les récoltes}}
\end{itemize}

Le septième cas d’exonération prévu à l’article 261 du CGI concerne « les bâtiments et installations qui servent aux exploitations rurales pour loger les animaux ou serrer les récoltes ».  
Cette exonération vise à alléger les charges pesant sur les activités agricoles essentielles.  
Elle doit toutefois être interprétée strictement, car elle ne s’applique qu’à des bâtiments présentant une affectation agricole exclusive.

\begin{itemize} [font=\color{magenta} \normalsize, label=\ding{43}]
	\item \textbf{Première condition : l’immeuble doit servir à une exploitation rurale.}
\end{itemize}

\noindent L’exonération ne concerne que les bâtiments utilisés dans le cadre d’une exploitation agricole, pastorale ou agro-rurale effective : élevage, cultures vivrières ou arbustives, maraîchage, plantations, etc.  
Un bâtiment simplement situé en zone rurale ne saurait être exonéré s’il n’est pas rattaché à une activité agricole réelle.  
Une exploitation devenue inactive ou transformée en activité commerciale entraîne la perte de l’exonération.

\begin{itemize} [font=\color{magenta} \normalsize, label=\ding{43}]
	\item \textbf{Deuxième condition : usage exclusif pour loger les animaux ou serrer les récoltes.}
\end{itemize}

\noindent Le CGI identifie deux usages exonérés : le logement des animaux (étables, bergeries, poulaillers, porcheries, écuries) et le serrage des récoltes (greniers, hangars à grains, silos, magasins agricoles).  
L’usage doit être strictement agricole.  
Ne peuvent être exonérés : les ateliers, garages, dépôts de matériel, boutiques, espaces de vente, locaux mixtes ou bâtiments servant partiellement à des activités commerciales ou d’habitation.  
Lorsqu’un bâtiment combine plusieurs usages, l’exonération ne peut porter que sur la partie effectivement et exclusivement agricole.


\begin{itemize} [font=\color{magenta} \normalsize, label=\ding{43}]
	\item \textbf{Limites de l’exonération.}
\end{itemize}

\noindent L’exonération ne s’étend ni aux habitations des exploitants ou des ouvriers agricoles, ni aux bâtiments servant à la transformation des produits (pressage, broyage, concassage), ni aux installations louées à des tiers.  
Les bâtiments agricoles désaffectés ou reconvertis à des usages non agricoles doivent être réintégrés dans l’assiette de la taxe foncière.  
Une vigilance particulière s’impose pour les bâtiments polyvalents : la moindre utilisation non agricole fait obstacle à l’exonération.

\paragraph{Synthèse doctrinale.}
La doctrine rappelle que l’exonération agricole « ne s’attache pas à la localisation rurale de l’immeuble, mais à sa fonction strictement agricole ».  
Ainsi, une étable ou un grenier à grains est exonéré, mais une maison d’habitation ou un hangar servant également de boutique doit être imposé.  
Il appartient à l’inspecteur du cadastre de vérifier l’usage réel des bâtiments, d’identifier les activités génératrices de revenus et, le cas échéant, de procéder à une imposition partielle des surfaces non agricoles.

\begin{itemize} [font=\color{magenta} \Large, label=\ding{189}]
	\item \textbf{{\large Immeubles servant exclusivement à l’habitation et effectivement habités par leurs propriétaires, conjoints, ascendants ou descendants directs}}
\end{itemize}

Le huitième cas d’exonération prévu à l’article 261 du CGI concerne « les immeubles servant exclusivement à l’habitation et effectivement habités par leurs propriétaires, les conjoints, les ascendants ou les descendants directs ».  
Le texte précise expressément que cette exonération ne vise que \textbf{l’habitation principale unique}.  
Il s’agit d’une exonération de nature sociale, dont l’application doit être strictement encadrée.  
Quatre conditions cumulatives doivent être réunies.

\begin{itemize} [font=\color{magenta} \normalsize, label=\ding{43}]
	\item \textbf{Première condition : usage exclusivement résidentiel.}
\end{itemize}

\noindent L’immeuble doit servir uniquement à l’habitation.  
La présence d’activités commerciales, artisanales, de bureaux, de boutiques ou d’ateliers fait obstacle à l’exonération.  
Un immeuble à usage mixte, comportant par exemple un commerce au rez-de-chaussée ou un atelier de stockage, ne peut bénéficier de l’exonération que pour la partie strictement résidentielle lorsque celle-ci est clairement identifiable et distincte.  
Toute utilisation détournée ou polyvalente remet en cause le caractère exclusif exigé par la loi.

\begin{itemize} [font=\color{magenta} \normalsize, label=\ding{43}]
	\item \textbf{Deuxième condition : occupation effective de l’immeuble.}
\end{itemize}

\noindent L’habitation doit être effectivement occupée : il ne peut s’agir d’un logement secondaire, vacant, occasionnel ou simplement déclaré comme habitation principale.  
L’occupation doit être réelle, permanente et constatée sur place (visites, enquêtes, consommations d’eau et d’électricité, informations de voisinage).  
Un immeuble en construction, inachevé ou laissé libre ne peut pas bénéficier de l’exonération tant qu’il n’est pas effectivement habité.

\begin{itemize} [font=\color{magenta} \normalsize, label=\ding{43}]
	\item \textbf{Troisième condition : occupation par les personnes légalement autorisées.}
\end{itemize}

\noindent L’immeuble doit être habité par :
\begin{itemize}
	\item le propriétaire ;
	\item son conjoint ;
	\item ses ascendants directs ;
	\item ou ses descendants directs.
\end{itemize}
Toute occupation par une personne extérieure à ce cercle — même gratuitement — fait tomber l’exonération.  
Ainsi, un logement occupé par un frère, une sœur, un cousin, un ami ou un locataire ne peut en aucun cas être exonéré.  
L’exonération est strictement réservée à la cellule familiale directe.

\begin{itemize} [font=\color{magenta} \normalsize, label=\ding{43}]
	\item \textbf{Quatrième condition : caractère unique de l’habitation principale.}
\end{itemize}

\noindent Le propriétaire ne peut bénéficier de l’exonération que pour une seule habitation principale.  
S’il possède plusieurs maisons qu’il occupe alternativement, une seule peut être exonérée.  
Les résidences secondaires, maisons de vacances, logements du village ou appartements complémentaires demeurent imposables.  
En cas d’habitation multiple au sein d’un même couple ou d’une même famille, l’administration doit déterminer la résidence principale selon des critères objectifs (lieu de travail, durée d’occupation, factures de consommation, présence familiale).

\paragraph{Synthèse doctrinale.}
Cette exonération repose sur une logique de justice sociale et vise à protéger la résidence principale du contribuable, non l’ensemble de son patrimoine immobilier.  
Comme l’énonce la doctrine, « l’exonération s’attache à l’usage résidentiel exclusif et à l’occupation familiale directe, non à la simple qualité de propriétaire ».  
Toute activité commerciale, tout revenu foncier, toute occupation par un tiers ou tout cumul de logements principaux fait obstacle à l’exonération.  
L’inspecteur du cadastre doit donc vérifier avec rigueur l’usage réel du logement et procéder à une imposition dès que l’une des conditions légales n’est plus remplie.


\begin{itemize} [font=\color{magenta} \Large, label=\ding{190}]
	\item \textbf{{\large Autres exonérations permanentes}}
\end{itemize}

Outre les catégories précédemment étudiées, l’article 261 du CGI prévoit trois exonérations permanentes supplémentaires. Leur portée, bien que plus restreinte, doit être interprétée avec rigueur afin d’éviter des extensions abusives du champ exonéré.

\begin{itemize} [font=\color{magenta} \normalsize, label=\ding{43}]
	\item \textbf{Bâtiments et installations des chemins de fer de l’État.}
\end{itemize}

\noindent Sont exonérés les bâtiments et installations appartenant aux chemins de fer de l’État lorsqu’ils sont affectés directement au fonctionnement du service public ferroviaire : gares techniques, postes de signalisation, ateliers de maintenance et infrastructures nécessaires à la sécurité et à la circulation des trains.  
L’exonération ne s’étend pas aux bâtiments administratifs, commerciaux ou aux locaux annexes ne participant pas directement à l’exploitation du réseau.  
Toute activité génératrice de revenus ou toute utilisation détournée de ces bâtiments entraîne l’imposition des surfaces concernées.

\begin{itemize} [font=\color{magenta} \normalsize, label=\ding{43}]
	\item \textbf{Immeubles appartenant à des États étrangers et affectés à la résidence officielle de leurs missions diplomatiques et consulaires.}
\end{itemize}

\noindent Cette exonération découle des principes de droit international et repose sur la réciprocité diplomatique.  
Elle vise exclusivement les immeubles appartenant à des États étrangers et servant de résidence officielle à leurs chefs de mission diplomatique ou consulaire accrédités auprès du gouvernement togolais.  
Elle ne couvre ni les locaux commerciaux dépendant de ces missions, ni les logements du personnel, ni les immeubles acquis à des fins patrimoniales.  
Toute affectation non diplomatique ou usage lucratif fait obstacle à l’exonération.

\begin{itemize} [font=\color{magenta} \normalsize, label=\ding{43}]
	\item \textbf{Immeubles servant exclusivement à l’activité des sociétés mutualistes agréées.}
\end{itemize}

\noindent Sont exonérés les immeubles appartenant aux sociétés mutualistes reconnues par l’autorité de tutelle, dès lors qu’ils sont affectés exclusivement à leurs activités de prévoyance, de solidarité ou d’assistance.  
Les mutualités exerçant des activités commerciales annexes — telles que la location de salles, l’exploitation de boutiques, pharmacies ou services payants — ne peuvent bénéficier de l’exonération pour les surfaces concernées.  
L’inspecteur du cadastre doit donc distinguer précisément les activités mutualistes proprement dites des activités économiques accessoires afin de garantir une imposition correcte.


\subsection{Exonérations temporaires (Articles 262 à 267 du CGI)}

Les exonérations temporaires constituent un mécanisme d’incitation fiscale destiné à encourager l’investissement immobilier et la modernisation du parc bâti.  
Elles concernent notamment la construction de nouveaux bâtiments, les reconstructions, les additions de constructions, la conversion de bâtiments ruraux et la régularisation de lotissements.  
En contrepartie, ces exonérations exigent le respect strict des obligations déclaratives prévues par le Livre des Procédures Fiscales.  
L’analyse des articles 262 à 267 permet d’en mesurer la portée, les finalités et les limites pratiques.

\begin{itemize} [font=\color{magenta} \Large, label=\ding{182}]
	\item \textbf{{\large Exonérations liées aux constructions nouvelles, reconstructions et additions (Art. 262 CGI)}}
\end{itemize}

L’article 262 institue deux régimes distincts d’exonération selon l’usage que le propriétaire destine au bâtiment à compter de son achèvement.

\begin{itemize} [font=\color{magenta} \normalsize, label=\ding{43}]
	\item \textbf{Exonération de deux ans pour les immeubles à usage commercial, industriel ou professionnel.}
\end{itemize}

\noindent Cette exonération vise à soutenir l’investissement productif. Elle allège la pression fiscale au moment où les entreprises réalisent leurs plus lourds investissements : construction d’ateliers, entrepôts, bureaux, locaux professionnels ou unités industrielles.  
L’objectif est de favoriser l’implantation de nouvelles activités génératrices d’emplois et de valeur économique.\\

\noindent La nature productive de ces immeubles entraîne en général une capacité rapide de génération de revenus. Une exonération prolongée créerait un déséquilibre fiscal au détriment des collectivités locales.  
Le législateur limite donc l’avantage à deux ans, période jugée suffisante pour permettre à l’exploitation de démarrer. \\ 
L’usage doit être strictement économique dès l’achèvement : tout local utilisé à des fins d’habitation ou laissé vacant sans motif professionnel peut invalider l’exonération.  
Dans le cas d’immeubles mixtes (ex. commerce en rez-de-chaussée et logement à l’étage), seule la partie réellement dédiée à l’activité commerciale ou professionnelle bénéficie de l’exonération, ce qui impose à l’inspecteur du cadastre de ventiler précisément les surfaces.

\begin{itemize} [font=\color{magenta} \normalsize, label=\ding{43}]
	\item \textbf{Exonération de cinq ans pour les immeubles affectés à l'habitation.}
\end{itemize}


\noindent Il s’agit de l’exonération la plus favorable du CGI. Elle s’inscrit dans une politique d’encouragement du logement, dans un contexte où la croissance démographique exerce une forte pression sur le marché immobilier.  
Elle soutient autant l’auto-construction des ménages que les opérations menées par les promoteurs immobiliers.\\

\noindent L’exonération profite directement à la politique sociale du logement en réduisant la charge fiscale qui pèse sur les nouveaux propriétaires.  
Cependant, cet avantage doit être strictement réservé aux immeubles effectivement affectés à l’habitation dès leur achèvement.  
En cas de changement d’usage ultérieur — passage au commerce, stockage, location professionnelle — l’exonération devient caduque pour la période restante. \\
 
\noindent La condition déclarative est fondamentale : un immeuble non déclaré dans les délais perd intégralement le bénéfice de l’exonération, même si toutes les autres conditions sont remplies.  
Le rôle de l’administration est donc d’éviter que des immeubles achevés depuis plusieurs années soient « régularisés » tardivement pour bénéficier abusivement de l'exonération.

\begin{itemize} [font=\color{magenta} \normalsize, label=\ding{43}]
	\item \textbf{Exonération de cinq ans pour les immeubles acquis par les établissements financiers (hypothèques, dations).}
\end{itemize}

\noindent Cette exonération concerne les immeubles récupérés par :
\begin{itemize}
	\item les banques,
	\item les établissements financiers agréés,
	\item les entreprises publiques à caractère économique.
\end{itemize}

\noindent Ces immeubles sont généralement issus de saisies ou de dations en paiement et se retrouvent temporairement dans le patrimoine de l’établissement créancier.\\

\noindent L’objectif fiscal est de faciliter la revente ou la mise en location transitoire des immeubles saisis, afin de permettre aux établissements financiers de recouvrer leurs créances sans supporter une charge fiscale supplémentaire.  
Cette exonération évite que les immeubles saisis perdent de la valeur faute d'entretien ou de repreneur et soutient la fluidité du marché immobilier.\\
  
\noindent Cependant, elle ne doit pas devenir une niche permettant aux établissements financiers de conserver durablement des immeubles ou d’exercer une activité de location commerciale déguisée.  
Toute utilisation autre que la revente ou la gestion conservatoire (location temporaire pour éviter la dégradation du bien) supprime immédiatement l’exonération.\\
  
\noindent La durée de cinq ans est ferme et ne peut être prolongée, même en cas de difficultés de revente ou de litiges affectant la disponibilité du bien.  
L’inspecteur doit donc vérifier annuellement que le bien est réellement en attente de cession et qu’il n’est ni transformé, ni exploité commercialement.

\begin{itemize} [font=\color{magenta} \Large, label=\ding{183}]
	\item \textbf{{\large Conversions de bâtiments ruraux en habitations locatives ou en usines (Art. 263 CGI)}}
\end{itemize}

L’article 263 du CGI prévoit une exonération temporaire de cinq ans pour les bâtiments ruraux transformés en habitations locatives ou en usines.  
Cette mesure vise à encourager la modernisation du bâti rural et la reconversion de structures agricoles anciennes vers des usages répondant aux besoins actuels du logement et de l’activité économique.

\begin{itemize} [font=\color{magenta} \normalsize, label=\ding{43}]
	\item \textbf{Existence préalable d’un bâtiment rural.}
\end{itemize}

\noindent La conversion doit porter sur un bâtiment antérieurement affecté à un usage agricole (étable, grenier, hangar, dépendance rurale).  
Une reconstruction totale ne constitue pas une conversion au sens de la loi et relève du régime des constructions nouvelles (Art. 262 CGI).

\begin{itemize} [font=\color{magenta} \normalsize, label=\ding{43}]
	\item \textbf{Transformation réelle et fonctionnelle.}
\end{itemize}

\noindent Les travaux doivent modifier de manière substantielle l’affectation du bâtiment pour permettre son nouvel usage.  
Une simple réparation ou remise en état légère ne suffit pas : le bâtiment doit être effectivement utilisable comme habitation locative ou comme unité de production au moment de la déclaration.

\begin{itemize} [font=\color{magenta} \normalsize, label=\ding{43}]
	\item \textbf{Nouvel usage : habitation locative ou activité industrielle.}
\end{itemize}

\noindent L’exonération est réservée aux bâtiments devenus :
\begin{itemize}
	\item des logements destinés à la location ;
	\item ou des usines ou ateliers de transformation.
\end{itemize}
Les habitations principales du propriétaire, les usages commerciaux simples ou les activités non industrielles ne sont pas éligibles.\\

\noindent L’exonération s’applique pour cinq ans à compter de l’année suivant l’achèvement des travaux de conversion. Une vigilance particulière s’impose pour éviter les assimilations frauduleuses entre conversion et construction nouvelle ainsi que les changements d’usage non déclarés.


\begin{itemize} [font=\color{magenta} \Large, label=\ding{45 }]
	\item \textbf{{\large Absence d’exonération pour les terrains à usage commercial ou industriel (Art. 264 CGI)}}
\end{itemize}

L’article 264 CGI exclut explicitement du bénéfice des exonérations temporaires les terrains affectés à un usage commercial ou industriel ainsi que les terrains utilisés pour la publicité commerciale (panneaux, affiches, écrans).  
Cette exclusion repose sur l’idée que ces terrains génèrent un revenu potentiel immédiat et ne justifient pas une période d’allègement fiscal.  
Ainsi, tout terrain destiné à des activités productives ou promotionnelles est imposable dès l’année suivant son affectation, indépendamment de l’état d’avancement des constructions éventuelles.

\begin{itemize} [font=\color{magenta} \Large, label=\ding{45}]
	\item \textbf{{\large Régularisation des lotissements irréguliers (Art. 265 CGI)}}
\end{itemize}

Les immeubles construits sur des lotissements irréguliers ne bénéficient de l’exonération prévue à l’article 262 qu’à compter de l’année où le lotissement devient régulier.  
Cette règle vise à décourager les constructions dans des zones non conformes aux normes d’urbanisme.  
Elle encourage également les propriétaires à engager rapidement les démarches de régularisation foncière, condition indispensable à l’accès aux avantages fiscaux.  
L’exonération ne saurait être appliquée rétroactivement pour les années antérieures à la régularisation.


\begin{itemize} [font=\color{magenta} \Large, label=\ding{45}]
	\item \textbf{{\large Déclarations obligatoires pour bénéficier des exonérations (Art. 266 CGI)}}
\end{itemize}

L’article 266 établit un principe fondamental : \textbf{l’exonération n’est accordée que si les déclarations obligatoires sont déposées dans les délais}.  
Le propriétaire doit :
\begin{itemize}[font=\color{magenta} \normalsize, label=\ding{52}]
	\item déclarer l’ouverture des travaux dans les quatre mois suivant leur commencement ;
	\item déclarer la fin des travaux dans les quatre mois suivant leur achèvement.
\end{itemize}

\noindent Toute déclaration doit préciser la nature du bâtiment, sa destination, sa superficie et sa référence cadastrale, et être accompagnée d’un plan ou croquis.  
À défaut de déclaration dans les délais :
\begin{itemize}[font=\color{magenta} \normalsize, label=\ding{247}]
	\item l’immeuble est imposé dès le 1\textsuperscript{er} janvier suivant son achèvement ;
	\item la cotisation est majorée jusqu’à cinq fois le montant dû pour l’année de découverte.
\end{itemize}

\noindent Cette sanction vise à dissuader les omissions volontaires qui constituent une importante source de perte d’assiette foncière.

\begin{itemize} [font=\color{magenta} \Large, label=\ding{45}]
	\item \textbf{{\large Déclarations tardives et fraction résiduelle d’exonération (Art. 267 CGI)}}
\end{itemize}

Une déclaration tardive ouvre droit uniquement à la fraction de l’exonération restant à courir à partir de l’année suivant sa production.  
Ainsi, un immeuble déclaré plusieurs années après son achèvement ne peut bénéficier que de la portion d’exonération non encore consommée. \\
 
\noindent L’année suivant immédiatement l’achèvement ne peut jamais être exonérée en cas de déclaration tardive.  
Cette règle garantit l’équité fiscale, préserve l’assiette des années antérieures et encourage le respect strict des obligations déclaratives.  
Elle renforce le lien entre avantage fiscal et transparence cadastrale.

\section*{Conclusion}

L’étude de la taxe foncière sur les propriétés bâties (TFPB) met en évidence un impôt à la fois essentiel pour les finances locales et techniquement exigeant dans son application.  
La détermination du champ d’application, l’identification du redevable, le respect des obligations déclaratives, la maîtrise de la valeur locative cadastrale, la liquidation de l’impôt et l’application rigoureuse des régimes d’exonération constituent autant d’étapes où l’expertise de l’inspecteur du cadastre est déterminante.\\

\noindent La TFPB repose sur une logique fondée sur la richesse immobilière construite : elle évalue la potentialité économique du bâti, son usage et la valeur locative qui en découle.  
Son rendement dépend de la qualité du relevé cadastral, de la fiabilité des déclarations, de la détection des omissions et de la mise à jour permanente du fichier immobilier.  
La complexité de cet impôt provient notamment de la diversité des usages du bâti, des situations juridiques multiples (usufruit, emphytéose, indivision) et de la variété des exonérations permanentes et temporaires prévues par le CGI.\\

\noindent Cependant, la TFPB ne représente qu’une composante du dispositif foncier.  
À côté du bâti, l’assiette fiscale comprend également les terrains non bâtis, qui constituent dans de nombreuses communes togolaises un gisement fiscal considérable mais encore largement sous-exploité.  
Alors que la TFPB repose sur l’analyse du bâti, la taxe foncière sur les propriétés non bâties (TFPNB) obéit à une logique distincte : celle de la potentialité foncière, de la pression urbaine et de la valorisation du sol indépendamment de toute construction.\\

\noindent La compréhension de la TFPB offre ainsi les bases nécessaires pour aborder la TFPNB, mais cette dernière requiert une réflexion propre concernant :
\begin{itemize}[font=\color{magenta} \large, label=\ding{219}]
	\item la qualification du terrain ;
	\item les usages possibles du sol ;
	\item les mécanismes d’incitation à la mise en valeur ;
	\item le rôle de la fiscalité dans la gestion et la planification urbaines.
\end{itemize}

\noindent Cette transition ouvre le chapitre suivant, consacré à la taxe foncière sur les propriétés non bâties (TFPNB), dont les principes, les objectifs et les modalités de mise en œuvre diffèrent profondément de ceux applicables au bâti.


