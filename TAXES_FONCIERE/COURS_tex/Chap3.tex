\chapter{Taxe Foncière sur les Propriétés Non Bâties (TFPNB)}
La Taxe Foncière sur les Propriétés Non Bâties (TFPNB) constitue, avec la taxe sur les propriétés bâties (TFPB), l’un des piliers du système de fiscalité foncière locale. 
À la différence de la TFPB, centrée sur la valeur locative des constructions, la TFPNB appréhende la \textbf{valeur intrinsèque du sol}, indépendamment de toute édification : 
valeur agricole, valeur spéculative, valeur extractive ou encore valeur d’agrément. 
Elle taxe ainsi une richesse foncière ``à l’état pur'', dont la valorisation dépend autant du contexte économique que des documents de planification urbaine.\\

\noindent Dans le contexte togolais, où l’expansion urbaine est rapide, où la pression foncière est forte et où de vastes portions du territoire demeurent insuffisamment mises en valeur, la TFPNB devient un outil stratégique pour les collectivités. 
Elle est à la fois :
\begin{itemize}[font=\color{magenta} \large, label=\ding{229}]
	\item un \textbf{instrument économique}, en incitant à la mise en valeur des terrains ;
	\item un \textbf{instrument urbanistique}, en luttant contre la rétention foncière et la spéculation ;
	\item un \textbf{instrument fiscal}, en élargissant une assiette encore largement sous-exploitée.
\end{itemize}

\noindent Pour l’inspecteur du cadastre, la TFPNB nécessite une maîtrise combinée du \textbf{droit fiscal}, du \textbf{droit foncier} et du \textbf{droit de l’urbanisme}. 
La qualification d’un terrain dépend en effet autant du Code Général des Impôts (CGI) que des documents d’aménagement (PDAU, PDA, zonage, servitudes). 
Comme le soulignent la doctrine, la fiscalité immobilière repose sur  la juste appréciation de la richesse foncière disponible, indépendamment des choix individuels d’exploitation . 
De son côté, Maurin rappelle que le cadastre constitue la preuve première de l’usage réel ou potentiel du sol, et qu’il conditionne la sécurité de l’assiette foncière.\\

\noindent L’objectif de ce chapitre est de présenter une vision complète, opérationnelle et doctrinalement cohérente du régime de la TFPNB : 
définition juridique, champ d’application, classification fiscale des terrains, méthodes de valorisation, mécanismes de liquidation, exonérations et rôle de l’inspecteur du cadastre. 
Il s’agit d’offrir aux auditeurs une compréhension approfondie de la fiscalité du non bâti, afin de garantir une administration fiable, équitable et techniquement fondée de cette composante essentielle de l’impôt foncier togolais.


\section{Fondements juridiques et champ d’application de la TFPNB}

\subsection{Fondements juridiques de la TFPNB}

La Taxe Foncière sur les Propriétés Non Bâties (TFPNB) repose sur la taxation de la richesse foncière « à l’état pur », c’est-à-dire de la valeur du sol indépendamment de toute construction. 
Deux principes justifient son existence : (i) la terre constitue un capital patrimonial qui conserve une valeur économique, même non exploitée ; 
(ii) le sol possède une potentialité économique (agricole, spéculative, extractive, d’agrément) qui fonde son imposition.\\

\noindent La fiscalité du non bâti vise à appréhender la rente foncière latente, indépendamment des choix individuels d’exploitation. 
Maurin souligne que le cadastre établit la preuve première de l’usage réel ou potentiel du sol, condition essentielle à la sécurité de l’assiette foncière.\\

\noindent Au Togo, l’encadrement légal de la TFPNB découle principalement des articles 258 et 259 du Code Général des Impôts (CGI), ce dernier définissant de manière précise son champ d’application matériel.

\subsection{Définition légale des propriétés non bâties selon l’article 259 du CGI}

L’article 259 du CGI dispose que :
\begin{quote}
	« Sont imposables au titre des propriétés non bâties les immeubles urbains constitués par des terrains situés dans l'étendue d'une agglomération déjà existante ou en voie de formation et compris dans les limites des plans de lotissement régulièrement approuvés et les terrains qui, se trouvant en dehors du périmètre des agglomérations visées ci-dessus, sont destinés à l'établissement de constructions lorsque ces dernières ne se rattachent pas à une exploitation agricole. »
\end{quote}

\noindent Cette définition permet de distinguer deux grandes catégories de terrains imposables :

\begin{itemize} [font=\color{magenta} \normalsize, label=\ding{43}]
	\item \textbf{Les terrains situés dans des agglomérations existantes ou en voie de formation.}
\end{itemize}

\noindent Il s’agit des terrains inclus dans un périmètre urbain constitué ou en expansion, et notamment des parcelles comprises dans un plan de lotissement régulièrement approuvé.  
Ces terrains, même non exploités, entrent automatiquement dans l’assiette de la TFPNB dès lors qu’ils ne supportent aucune construction imposable à la TFPB.


\begin{itemize} [font=\color{magenta} \normalsize, label=\ding{43}]
	\item \textbf{Les terrains situés hors agglomération mais destinés à la construction.}
\end{itemize}

\noindent Sont imposables les terrains ruraux ou périurbains destinés à accueillir des constructions qui ne se rattachent pas à une exploitation agricole.  
Il peut s’agir de terrains destinés à des habitations, à des activités industrielles ou artisanales, ou encore à des projets de lotissement privé.

\noindent Les terrains affectés exclusivement à une exploitation agricole ne relèvent pas du champ d’application de l’article 259 et obéissent à des régimes spécifiques d’évaluation.

\subsection{Distinction entre propriété bâtie et non bâtie : un enjeu central d’assiette}

La détermination du champ d’application de la TFPNB suppose une distinction rigoureuse entre propriétés bâties et non bâties. 
Selon les critères analysés au chapitre précédent, un bâtiment est considéré comme bâti lorsqu’il est fixé au sol, présente une durabilité, résulte d’un travail humain et est apte à procurer un avantage économique.

Ainsi :
\begin{itemize}[font=\color{magenta} \normalsize, label=\ding{219}]
	\item un conteneur posé sur le sol ne constitue pas un bâti ;
	\item un hangar démontable non scellé maintient le terrain dans la catégorie non bâtie ;
	\item une paillote démontable ne retire pas un terrain de l’assiette de la TFPNB ;
	\item un kiosque métallique scellé peut constituer un bâti et faire basculer le terrain dans l’assiette de la TFPB.
\end{itemize}

\noindent Toute erreur de qualification entre bâti et non bâti entraîne une erreur d’assiette, affectant la sécurité juridique de l’imposition et les recettes communales.

\subsection{Notion d’agglomération et rôle du droit de l’urbanisme}

L’article 259 renvoie implicitement au droit de l’urbanisme pour apprécier les notions suivantes :
\begin{itemize}[font=\color{magenta} \normalsize, label=\ding{70}]
	\item « agglomération existante »,
	\item « agglomération en voie de formation »,
	\item « plan de lotissement régulièrement approuvé ».
\end{itemize}

\noindent L’inspecteur du cadastre doit donc s’appuyer sur les documents d’aménagement :
\begin{itemize}[font=\color{magenta} \normalsize, label=\ding{70}]
	\item Plans Directeurs d’Aménagement Urbain (PDAU),
	\item Plans de Détail d’Aménagement (PDA),
	\item plans de lotissement approuvés,
	\item arrêtés municipaux de délimitation des zones urbaines.
\end{itemize}

\noindent Le zonage constitue donc la première source de qualification fiscale .  
Ainsi :
\begin{itemize}[font=\color{magenta} \normalsize, label=\ding{229}]
	\item un terrain situé en zone U (urbaine) est présumé constructible et donc imposable à la TFPNB ;
	\item un terrain situé en zone A (agricole) ne peut être taxé comme constructible tant qu’un changement de vocation n’est pas acté ;
	\item un terrain situé en zone N (naturelle ou protégée) peut être imposable à un niveau réduit selon son usage et les servitudes applicables.
\end{itemize}

\noindent La TFPNB dépend donc étroitement de la cartographie urbaine et des décisions d’aménagement du territoire.

\subsection{Rôle de l’Inspecteur du Cadastre dans la détermination du champ d’application}

L’inspecteur du cadastre joue un rôle \textbf{probatoire, analytique et décisionnel}.  
Il lui revient de :
\begin{itemize}[font=\color{magenta} \normalsize, label=\ding{43}]
	\item vérifier la réalité physique du terrain (visites, photographies, levés, SIG) ;
	\item confronter cette réalité aux documents d’urbanisme (PDAU, PDA, zonage) ;
	\item qualifier le terrain : agricole, constructible, spécial, dépendance non intégrante ;
	\item inscrire ou mettre à jour la classification dans la matrice cadastrale ;
	\item détecter les omissions : terrains en friche non déclarés, réserves foncières non imposées, terrains présentés abusivement comme agricoles.
\end{itemize}

\noindent La fiscalité des terrains repose sur l’observation objective du sol, et non sur la seule déclaration du propriétaire.  
La TFPNB impose ainsi une vigilance accrue et une parfaite maîtrise des règles de qualification foncière.

\section{Classification fiscale des terrains}

La classification fiscale des terrains constitue une étape déterminante dans l’établissement de l’assiette de la Taxe Foncière sur les Propriétés Non Bâties (TFPNB). 
Elle conditionne la méthode d’évaluation, la base imposable, l’application d’abattements éventuels ainsi que le taux d’imposition. 
La classification cadastrale est l’acte fondamental de l’administration foncière : elle confère à un terrain sa nature fiscale et structure l’ensemble de l’assiette . 
La juste application des taxes foncières repose d’abord sur une répartition motivée et cohérente des catégories de biens.\\

\noindent La classification repose sur trois séries d’éléments : les critères juridiques du Code Général des Impôts, les documents d’urbanisme et l’observation effective du terrain réalisée par l’inspecteur du cadastre.

\subsection{Les terrains agricoles}

\paragraph{Définition fiscale et juridique.\\}
Un terrain est classé agricole lorsqu’il est effectivement exploité (cultures vivrières ou industrielles, pâturages, plantations) ou lorsqu’il présente par nature une vocation agricole.  
La classification peut également découler du zonage en zone agricole (zone A) dans les documents d’urbanisme.\\

\noindent Selon la doctrine, l’usage réel prime en principe sur la vocation potentielle, sauf en cas de spéculation manifeste ou de changement de vocation décidé par les autorités locales.

\paragraph{Enjeux fiscaux.\\}

 Dans le système fiscal togolais, les terrains agricoles bénéficient en principe d’une exonération permanente de la taxe foncière sur les propriétés non bâties, dès lors qu’ils sont effectivement exploités à des fins agricoles. 

\noindent Une mauvaise classification peut avoir des conséquences importantes : la qualification d’un terrain à vocation constructible comme terrain agricole entraîne une exonération injustifiée et une perte directe de recettes pour la collectivité, tandis que la qualification erronée d’un terrain effectivement agricole comme terrain urbain ou constructible conduit à une imposition indue et à un risque élevé de contestation.\\

\noindent L’inspecteur du cadastre doit vérifier l’exploitation réelle, rechercher les indices de friche volontaire ou de rétention foncière, et confronter la situation observée aux documents d’urbanisme.


\subsection{Les terrains constructibles}

\paragraph{Définition.}
Un terrain est constructible lorsqu’il est situé dans une zone ouverte à l’urbanisation (zones U ou zones urbanisables), lorsqu’il est inclus dans un plan de lotissement approuvé ou lorsqu’il est destiné à recevoir une construction non agricole, conformément à l’article 259 du CGI.

\noindent Le caractère constructible repose sur la vocation juridique du sol et non sur son usage actuel.

\paragraph{Usage réel versus usage potentiel.}
En zone urbaine, le principe dominant est celui de l’usage potentiel : un terrain temporairement cultivé peut demeurer constructible si les documents d’urbanisme lui confèrent cette vocation.  
En zone rurale, l’usage réel agricole peut primer tant que le zonage n’a pas été modifié.

\paragraph{Importance du zonage urbanistique.}
Les documents d’urbanisme (PDAU, PDA, plans de lotissement) déterminent la vocation légale des sols.  
Un terrain classé constructible peut toutefois voir sa valeur corrigée en présence de servitudes (inconstructibilité partielle, risques naturels, protection environnementale).

\subsection{Les terrains à usage spécial}

\paragraph{Exemples.}
Relèvent de cette catégorie les terrains dont la valeur découle d’une activité spécifique, notamment :
\begin{itemize}[font=\color{magenta} \normalsize, label=\ding{219}]
	\item carrières, sablières, mines à ciel ouvert ;
	\item marais salants, étangs d’exploitation ;
	\item terrains industrialo-portuaires non bâtis.
\end{itemize}

\paragraph{Particularité fiscale.}
Ces terrains sont évalués sur la base d’un revenu d’extraction, distinct de la valeur vénale ou du revenu agricole.  
L’évaluation mobilise souvent des forfaits de production ou des estimations de rendement.  
La fiscalité des terrains spéciaux doit appréhender la ressource extraite plutôt que la simple superficie imposable.

\subsection{Les dépendances non intégrantes}

\paragraph{Typologie.}
Il s’agit des terrains attenants à un bâtiment mais ne constituant pas une dépendance fonctionnelle indispensable.  
Exemples :
\begin{itemize}[font=\color{magenta} \normalsize, label=\ding{219}]
	\item cours et jardins d’agrément non essentiels à l’usage du bâti ;
	\item parcs privés détachés ;
	\item réserves foncières contiguës non utilisées.
\end{itemize}

\paragraph{Critère de dépendance fonctionnelle.}
Une dépendance est intégrante lorsqu’elle est indispensable au fonctionnement du bâtiment principal (cour technique, aire de manœuvre, zone de stockage).  
À défaut, elle relève de la TFPNB.

\subsection{Principes directeurs de la classification}

La classification fiscale doit respecter trois grands principes :

\begin{enumerate}
	\item \textbf{Principe de légalité} : conformité stricte aux catégories définies par les articles 258 et 259 du CGI.
	\item \textbf{Principe de vocation urbanistique} : priorité au zonage officiel pour déterminer la vocation légale du sol.
	\item \textbf{Principe de réalité physique} : l’observation terrain (visites, SIG, levés) demeure l’élément décisif de l’appréciation.
\end{enumerate}

\noindent La classification constitue ainsi la pierre angulaire de l’assiette de la TFPNB, une erreur pouvant multiplier ou diviser la base imposable de manière significative.


\section{Détermination de la base imposable et liquidation de la TFPNB}

La Taxe Foncière sur les Propriétés Non Bâties (TFPNB) repose sur un principe fondamental : 
l’assiette imposable est constituée par la \textbf{valeur vénale du terrain au 1\textsuperscript{er} janvier} de l’année d’imposition. \\

\noindent À la différence de la taxe sur les propriétés bâties, qui taxe la valeur locative, la TFPNB appréhende la richesse foncière brute, indépendamment de l’usage ou de l’exploitation réelle du terrain.  
Les articles 272, 273, 274 et 276 du Code Général des Impôts (CGI) définissent de manière complète ce régime d’assiette et de liquidation.

\subsection{L’assiette légale : la valeur vénale du terrain (Article 272 CGI)}

L’article 272 du CGI énonce :
\begin{quote}
	« Les propriétés non bâties sont imposables à raison de leur valeur vénale au 1\textsuperscript{er} janvier de l'année d'imposition. »
\end{quote}

\noindent Ce principe fait de la TFPNB un impôt véritablement patrimonial.  
Le terrain est taxé d’après sa valeur sur le marché foncier, quelle que soit sa mise en valeur effective.  
Ainsi, un terrain en friche ou non viabilisé demeure imposable sur sa valeur économique potentielle.  
Cette logique, conforme à la doctrine, vise la « rente foncière latente » attachée au sol.

\subsection{Détermination de la valeur vénale : actes translatifs et estimation directe (Article 273 CGI)}

L’article 273 précise le mode de détermination de la valeur vénale :
\begin{quote}
	« La valeur vénale résulte des actes translatifs récents de propriété [...] ou, à défaut, d'une estimation directe. »
\end{quote}

\noindent Deux méthodes sont prévues :

\paragraph{1. La méthode comparative.}
Elle se fonde sur les ventes récentes portant sur des terrains similaires situés dans la même zone ou un lotissement comparable. \\ 
Cette méthode repose sur l’idée que le marché constitue la meilleure expression de la valeur réelle du sol.  
Elle est utilisée prioritairement lorsque des transactions fiables et récentes sont disponibles.

\paragraph{2. L'estimation directe.}
En l’absence de références de marché, l’administration procède à une estimation tenant compte :
\begin{itemize}[font=\color{magenta} \normalsize, label=\ding{219}]
	\item du zonage urbanistique et de la vocation légale du sol ;
	\item de l’état d’équipement du secteur (voirie, réseaux, services urbains) ;
	\item de la configuration physique du terrain (géométrie, enclavement, pente) ;
	\item des servitudes administratives ou contraintes techniques ;
\end{itemize}
et, le cas échéant, des barèmes internes élaborés par les services du cadastre.  
L’évaluation doit être motivée, documentée et capable de résister au contrôle contentieux.

\subsection{Compétence du cadastre dans la fixation de la valeur vénale (Article 274 CGI)}

L’article 274 dispose :
\begin{quote}
	« La valeur vénale est déterminée par les services chargés du cadastre. »
\end{quote}

\noindent Cette compétence exclusive signifie que les services du cadastre :
\begin{itemize}[font=\color{magenta} \normalsize, label=\ding{45}]
	\item maîtrisent les méthodes d’évaluation foncière ;
	\item établissent les valeurs retenues, sous forme de notes ou rapports d’évaluation ;
	\item assurent la mise à jour régulière des valeurs, au rythme des évolutions de marché et des modifications d’urbanisme.
\end{itemize}

\noindent Le cadastre est un instrument vivant : il reflète la situation juridique, technique et économique du sol et doit être continuellement actualisé. 

\subsection{Liquidation de la taxe : application du taux unique (Article 276 CGI)}

L’article 276 fixe le taux applicable à la TFPNB :
\begin{quote}
	« Le taux de la taxe foncière sur les propriétés non bâties est fixé à 0,5\% de la valeur vénale. »
\end{quote}

\noindent La taxe due est donc déterminée par la formule :

\begin{tcolorbox}[
	colback=gray!5,
	colframe=primarycolor,
	boxrule=0.8pt,
	arc=3mm,
	left=6pt,
	right=6pt,
	top=6pt,
	bottom=6pt,
	title=\textbf{Formule de liquidation de la TFPNB}
	]
	\[
	\text{TFPNB due}
	=
	\text{Valeur vénale au 1\textsuperscript{er} janvier}
	\times 0,5\%
	\]
\end{tcolorbox}



\noindent Ce taux unique présente trois avantages : stabilité, lisibilité et égalité fiscale.  
Contrairement à d’autres impôts locaux, il ne peut être modulé par les collectivités, ce qui contribue à la prévisibilité des recettes.

\subsection{Enjeux administratifs et fiscaux}

La simplicité apparente de la liquidation contraste avec la technicité de l’évaluation de la valeur vénale.  
Une erreur de classification (par exemple, un terrain constructible évalué comme agricole) ou une évaluation insuffisamment justifiée peut compromettre :

\begin{itemize}
	\item l'équité de l’impôt ;
	\item la sécurité juridique de la commune et de l’OTR ;
	\item la fiabilité du rôle d’imposition.
\end{itemize}

Il est à noté que la qualité de la liquidation dépend moins du calcul que de la rigueur de l’assiette .  
C’est pourquoi la mission du cadastre est déterminante : fixer une valeur juste, objective, fondée sur la loi et sur la réalité du marché foncier.

\section{Exonérations applicables à la TFPNB}

Les exonérations prévues en matière de Taxe Foncière sur les Propriétés Non Bâties (TFPNB) répondent à des logiques institutionnelles, économiques et sociales.  
Elles soustraient du champ de l’impôt certaines catégories de terrains en raison de leur statut public, de leur contribution à l’intérêt collectif ou de leur utilité agricole.  
Toutefois, ces exonérations peuvent également provoquer une contraction significative de l’assiette foncière des collectivités locales, notamment dans les zones urbaines et périurbaines où la valeur du sol est élevée.\\  

La présente section analyse les exonérations prévues par les articles 268 et 269 du Code Général des Impôts (CGI), en évaluant leur portée, leurs effets et leurs limites sur le rendement de la TFPNB.

\subsection{Exonérations prévues à l’article 268 du CGI}

L’article 268 dresse une liste de six catégories de terrains exonérés de la TFPNB.  
Ces exonérations sont permanentes et s’appliquent automatiquement dès lors que les terrains remplissent les conditions légales.  

\begin{itemize} [font=\color{primarycolor} \Large, label=\ding{182}]
	\item \textbf{{\large Terrains et voies de communication appartenant à l’État, aux préfectures, aux communes et aux établissements publics, non productifs de revenus}}
\end{itemize}

\noindent L’article 268, 1° dispose que sont exonérés :
\begin{quote}
	« les terrains et voies de communications appartenant à l’État, aux préfectures, aux communes, aux établissements et organismes publics, affectés ou non à l’usage public mais non productifs de revenus. »
\end{quote}

Cette exonération vise les terrains publics affectés à des usages administratifs, institutionnels ou collectifs.  
Elle repose sur une double logique : éviter que l’État ou les collectivités territoriales s’imposent mutuellement et préserver les biens destinés au service public.\\

Sur le plan fiscal, cette mesure présente toutefois une portée très large.  
Elle inclut non seulement les infrastructures et espaces effectivement utilisés, mais également les réserves foncières publiques non mises en valeur et parfois situées dans des zones urbaines à forte potentialité économique.  
Cette situation peut entraîner une contraction importante de l’assiette foncière locale, les communes disposant souvent de terrains non exploités qui, s’ils étaient imposables, constitueraient une source substantielle de recettes.\\

L’absence de revenus constitue la condition essentielle de l’exonération.  
Un terrain public qui génère un revenu (location, concession, exploitation commerciale) doit, en principe, sortir du champ de cette exonération, ce qui impose à l’administration une vigilance particulière quant aux usages réels du foncier public.

\begin{itemize} [font=\color{primarycolor} \Large, label=\ding{183}]
	\item \textbf{{\large Pépinières et jardins d’essai créés par l’administration ou par les sociétés d’intérêt collectif agricole.}}
\end{itemize}


\noindent L’article 268, 2° exonère les pépinières et jardins d’essai établis par l’administration ou par les sociétés d’intérêt collectif agricole, ainsi que les sociétés de prévoyance, lorsqu’ils sont utilisés dans un objectif de sélection et d’amélioration des plants.  
Cette exonération poursuit un but de soutien aux politiques publiques agricoles, à la recherche variétale et à la modernisation des techniques culturales.\\

\noindent Elle se justifie également par la faible valeur économique des terrains affectés à ces usages, l’essentiel de leur richesse provenant non pas du sol lui-même mais de l’activité scientifique ou agricole qui s’y déroule. \\ 

\noindent Cependant, l’absence de critères opérationnels précis dans le CGI peut entraîner des difficultés de contrôle.  
Certaines parcelles peuvent être présentées comme « jardins d’essai » sans véritable activité probante, dans le seul but de bénéficier de l’exonération.  
Une meilleure formalisation administrative — notamment l’exigence d’une attestation d’activité ou d’un plan de programme agricole — renforcerait l’efficacité et la transparence de ce régime.


\begin{itemize} [font=\color{primarycolor} \Large, label=\ding{184}]
	\item \textbf{{\large Terrains à usage scolaire.}}
\end{itemize}

\noindent L’article 268, 3° prévoit l’exonération des terrains affectés à un usage scolaire.  
Cette mesure concerne aussi bien les établissements publics que les établissements privés reconnus par les autorités compétentes.  
Elle vise à soutenir le développement des infrastructures éducatives en réduisant la charge fiscale afférente au foncier nécessaire à leur activité.\\

\noindent Sur le plan fiscal, cette exonération se comprend : l’enseignement représente un service public essentiel et les terrains qui lui sont affectés ne génèrent généralement pas de revenus patrimoniaux. \\ 

\noindent Toutefois, la portée de l’exonération peut devenir problématique dans certaines situations.  
Dans les zones urbaines, de grandes réserves foncières appartenant à des écoles privées peuvent être exonérées même lorsqu’elles ne sont pas utilisées pour l’activité éducative.  
Le CGI ne précise pas de seuil d’utilisation minimale, ni ne distingue les terrains strictement nécessaires au fonctionnement de l’établissement des terrains excédentaires servant de réserves foncières ou de supports à des projets immobiliers spéculatifs.\\

\noindent Cette absence de distinction limite la capacité de l’administration à taxer des terrains à forte valeur vénale et entraîne, de ce fait, une perte potentiellement significative de recettes pour les collectivités locales.

\begin{itemize} [font=\color{primarycolor} \Large, label=\ding{185}]
	\item \textbf{{\large Sols et terrains passibles de la taxe foncière sur les propriétés bâties.}}
\end{itemize}

\noindent L’article 268, 4° exclut du champ de la TFPNB les sols et terrains déjà passibles de la taxe foncière sur les propriétés bâties (TFPB).  
Cette exonération repose sur un principe fondamental de la fiscalité foncière : l’interdiction de la double imposition d’un même bien pour une même période.\\

\noindent En effet, la valeur du sol supportant une construction est intégrée dans la valeur locative cadastrale du bâtiment et, à ce titre, imposée via la TFPB.  
Soumettre ce même terrain à la TFPNB reviendrait à taxer deux fois une seule et même richesse foncière.\\

\noindent Toutefois, cette règle soulève des difficultés pratiques lorsqu’un bâtiment n’occupe qu’une partie réduite d’une grande parcelle.  
Le CGI togolais ne précise pas explicitement si la fraction de terrain excédentaire, non indispensable à l’usage du bâti, doit être regardée comme une dépendance intégrante ou comme un terrain non bâti imposable à la TFPNB.  
Cette absence de précision peut conduire à des situations de sous-imposition, notamment dans les zones urbaines où de vastes parcelles partiellement construites conservent une valeur vénale élevée.

\begin{itemize} [font=\color{primarycolor} \Large, label=\ding{186}]
	\item \textbf{{\large Terrains cultivés ou effectivement utilisés pour les cultures maraîchères, florales, fruitières ou la production de plants et semis.}}
\end{itemize}

\noindent L’article 268, 5° exonère les terrains cultivés ou effectivement utilisés, au 1\textsuperscript{er} janvier de l’année d’imposition, pour des cultures maraîchères, florales ou fruitières, ainsi que pour la production de plants et de semis.  
Cette exonération vise à soutenir l’agriculture intensive de proximité et à encourager des activités contribuant directement à la sécurité alimentaire et à l’économie locale.\\

\noindent Sur le plan fiscal, la justification repose sur le fait que ces terrains génèrent un revenu d’exploitation relativement modeste et ne relèvent pas d’une logique de rente foncière ou de spéculation immobilière.  
Ils sont ainsi assimilés à des instruments de production plutôt qu’à des actifs patrimoniaux.\\

\noindent Néanmoins, cette exonération présente des limites importantes.  
La notion d’« utilisation effective » est délicate à apprécier et peut donner lieu à des pratiques de contournement, notamment en zone périurbaine.  
Certains propriétaires peuvent maintenir une culture minimale ou temporaire dans le seul but d’échapper à la TFPNB, alors même que le terrain présente une forte vocation constructible.  
L’absence de critères quantitatifs ou de durée minimale d’exploitation affaiblit le rendement potentiel de la taxe et impose un contrôle rigoureux par les services cadastraux.

\begin{itemize} [font=\color{primarycolor} \Large, label=\ding{187}]
	\item \textbf{{\large Voies des chemins de fer de l’État.}}
\end{itemize}

\noindent L’article 268, 6° prévoit l’exonération des voies des chemins de fer appartenant à l’État.  
Cette disposition s’inscrit dans la logique classique de neutralité fiscale des grandes infrastructures publiques de transport, affectées à un usage collectif et non destinées à produire une rente foncière.\\

\noindent Cette exonération est pleinement justifiée pour les voies effectivement utilisées à des fins ferroviaires, qui sont par nature impropres à toute valorisation alternative.  
Toutefois, des interrogations apparaissent lorsque certaines emprises ferroviaires sont désaffectées, partiellement abandonnées ou situées dans des zones urbaines à forte valeur foncière.\\

\noindent Le CGI ne distingue pas entre les voies en exploitation et les terrains ferroviaires inutilisés ou devenus sans affectation fonctionnelle.  
Cette absence de distinction peut conduire à une exonération durable de terrains qui, en réalité, pourraient être valorisés ou réaffectés, constituant ainsi une source potentielle de recettes fiscales pour les collectivités locales.



\subsection{Exonération des immeubles ruraux (Article 269 du CGI)}

L’article 269 du Code Général des Impôts dispose de manière générale que :
\begin{quote}
	« Sont exemptés de la taxe les immeubles ruraux. »
\end{quote}

Cette exonération vise, dans son esprit initial, à protéger l’économie agricole et à éviter une pression fiscale excessive sur les terres rurales, souvent caractérisées par des rendements incertains et une faible rentabilité monétaire. 
Elle s’inscrit ainsi dans une logique sociale et économique de soutien aux populations vivant principalement de l’exploitation du sol.
\vspace{0.5cm}
\begin{itemize} [font=\color{magenta} \normalsize, label=\ding{45}]
	\item \textbf{Analyse juridique et fiscale.}
\end{itemize}

\noindent Sur le plan juridique, la notion d’« immeuble rural » n’est pas définie avec précision par le CGI. 
Elle repose implicitement sur des critères de localisation (hors agglomération) et d’usage agricole ou assimilé. 
Cette indétermination normative laisse à l’administration une large marge d’appréciation, mais elle fragilise également la sécurité juridique et la cohérence de l’assiette foncière.\\

\noindent Fiscalement, l’exonération se justifie pleinement lorsque les terrains concernés sont :
\begin{itemize}[font=\color{magenta} \normalsize, label=\ding{52}]
	\item effectivement exploités à des fins agricoles ou pastorales ;
	\item de superficie raisonnable au regard des capacités d’exploitation ;
	\item détenus par des exploitants dont le revenu dépend principalement de l’activité agricole.
\end{itemize}

\noindent Dans ces conditions, l’exonération permet de préserver l’équité fiscale et d’éviter une taxation disproportionnée de terres dont la valeur économique repose davantage sur le travail que sur le capital foncier.
\vspace{0.5cm}
\begin{itemize} [font=\color{magenta} \normalsize, label=\ding{45}]
	\item \textbf{Constat socio-fiscal et impact sur les recettes locales.}
\end{itemize}

\noindent Toutefois, l’observation du terrain révèle une réalité sensiblement différente de celle présumée par le législateur. 
Dans de nombreuses zones dites rurales ou périurbaines, les immeubles ruraux exonérés sont détenus non par de petits exploitants agricoles, mais par des personnes aisées résidant en milieu urbain.\\

\noindent Il s’agit fréquemment :
\begin{itemize}[font=\color{magenta} \normalsize, label=\ding{55}]
	\item de grandes superficies acquises à des fins de réserve foncière ;
	\item de terrains faiblement ou symboliquement cultivés ;
	\item de patrimoines fonciers détenus dans une logique de spéculation ou d’anticipation de l’urbanisation.
\end{itemize}

\noindent Cette situation engendre un double déséquilibre fiscal.  
D’une part, des propriétaires disposant d’un important capital foncier échappent totalement à l’impôt local.  
D’autre part, les collectivités territoriales, souvent elles-mêmes à caractère rural ou semi-rural, se trouvent privées de ressources fiscales indispensables à leur développement.\\

\noindent L’exonération généralisée des immeubles ruraux contribue ainsi à :
\begin{itemize}[font=\color{magenta} \normalsize, label=\ding{247}]
	\item réduire significativement le potentiel de recettes foncières locales ;
	\item limiter la capacité d’investissement des communes rurales (voirie, écoles, centres de santé) ;
	\item accentuer la dépendance financière de ces collectivités vis-à-vis des transferts de l’État.
\end{itemize}
\vspace{0.5cm}
\begin{itemize} [font=\color{magenta} \normalsize, label=\ding{45}]
	\item \textbf{Limites structurelles}
\end{itemize}

\noindent L’exonération prévue à l’article 269 souffre de deux insuffisances majeures.  
Premièrement, elle ne distingue pas entre les terrains effectivement agricoles et les terrains simplement classés comme ruraux sans exploitation réelle.  
Deuxièmement, elle ne prend pas en compte la superficie des biens exonérés, permettant l’exonération de très vastes domaines fonciers sans contrepartie fiscale.\\

\noindent En l’absence de mécanisme de plafonnement ou de requalification, cette exonération peut devenir un facteur de rétention foncière et de spéculation, en contradiction avec les objectifs d’aménagement du territoire et de justice fiscale.
\vspace{0.5cm}
\begin{itemize} [font=\color{magenta} \normalsize, label=\ding{45}]
	\item \textbf{Orientations}
\end{itemize}
\noindent Afin de préserver l’objectif social de l’exonération tout en renforçant la soutenabilité financière des collectivités locales, plusieurs orientations peuvent être envisagées :

\begin{enumerate}
	\item \textbf{Limiter l’exonération aux terrains effectivement agricoles}, en subordonnant le bénéfice de l’exonération à la preuve d’une exploitation réelle et continue.
	\item \textbf{Introduire un plafonnement de superficie exonérée}, au-delà duquel les terrains seraient soumis à la TFPNB, même en milieu rural.
	\item \textbf{Différencier le régime selon le profil du propriétaire}, en maintenant l’exonération pour les exploitants agricoles et en la restreignant pour les propriétaires non exploitants résidant en milieu urbain.
	\item \textbf{Renforcer le rôle du cadastre}, en confiant explicitement aux inspecteurs la mission d’identifier les terrains ruraux à forte vocation spéculative ou urbaine.
	\item \textbf{Mettre en place un mécanisme de requalification progressive}, permettant le basculement automatique vers un régime imposable dès lors que le terrain perd son caractère agricole effectif.
\end{enumerate}

\noindent Une telle évolution permettrait de concilier équité fiscale, développement rural et mobilisation durable des ressources foncières locales, tout en renforçant la crédibilité et l’efficacité de la TFPNB.


\section*{Conclusion}

La Taxe Foncière sur les Propriétés Non Bâties (TFPNB) occupe une place stratégique dans la fiscalité locale togolaise, bien que son rendement actuel demeure très inférieur à son potentiel réel. 
À la différence de la taxe sur les propriétés bâties, la TFPNB repose sur une logique patrimoniale pure : elle appréhende le sol en tant que capital foncier, indépendamment de son exploitation effective. 
À ce titre, elle constitue un instrument essentiel de lutte contre la rétention foncière, la spéculation et l’occupation inefficiente de l’espace.\\

\noindent L’analyse du champ d’application, de la classification fiscale des terrains et de la détermination de la base imposable a mis en évidence le rôle central du cadastre. 
L’identification rigoureuse des propriétés non bâties, leur qualification correcte au regard des règles d’urbanisme et l’évaluation objective de leur valeur vénale conditionnent directement la justesse de l’impôt et la crédibilité de l’action fiscale. 
Dans ce domaine, l’inspecteur du cadastre ne se limite pas à une fonction technique : il agit comme un garant de l’équité fiscale et de la sécurisation de l’assiette foncière.\\

\noindent L’étude des exonérations applicables à la TFPNB révèle toutefois des limites structurelles importantes. 
Si certaines exonérations répondent à des objectifs légitimes — soutien à l’agriculture, neutralité fiscale des biens publics, encouragement des activités d’intérêt général — leur champ excessivement large réduit sensiblement les ressources des collectivités territoriales, en particulier dans les zones rurales et périurbaines. 
Le décalage observé entre l’esprit du législateur et la réalité socio-foncière, marquée par la détention de vastes superficies rurales par des propriétaires urbains aisés, souligne la nécessité d’un encadrement plus ciblé et plus équitable de ces exonérations.\\

\noindent Ainsi, la TFPNB apparaît moins comme une taxe accessoire que comme un levier structurant de l’aménagement du territoire et du financement du développement local. 
Son efficacité repose avant tout sur la qualité de l’assiette, la rigueur des opérations cadastrales, la coordination entre les services du cadastre, de l’urbanisme et de la fiscalité, ainsi que sur une adaptation continue des règles aux dynamiques foncières contemporaines.\\

\noindent En définitive, la maîtrise des règles applicables à la TFPNB constitue une compétence essentielle pour l’inspecteur du cadastre. 
Elle conditionne non seulement la mobilisation durable des ressources locales, mais également la crédibilité de l’administration fiscale et la confiance des contribuables dans l’équité du système foncier.



