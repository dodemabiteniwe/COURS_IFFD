\section{Champ d’application de la TFPB}

\subsection*{(Version rédigée en paragraphes, avec puces équilibrées, enrichissement doctrinal et administratif)}

\subsection{Définition générale de la propriété bâtie}

La Taxe Foncière sur les Propriétés Bâties repose sur une définition précise de ce qu’est un « bâti » au sens fiscal. Une construction est considérée comme une propriété bâtie lorsqu’elle réunit quatre caractéristiques essentielles :

\begin{itemize}
	\item une fixation au sol à perpétuelle demeure,
	\item une durabilité réelle ou supposée,
	\item un résultat d’un travail humain,
	\item une aptitude à procurer un usage ou un avantage économique.
\end{itemize}

Ces quatre éléments constituent le socle juridique permettant de distinguer les constructions imposables des installations provisoires ou temporaires. Ils doivent être maîtrisés parfaitement par l'inspecteur du cadastre car toute erreur dans leur appréciation fausse l’assiette fiscale.

\subsection{Développement détaillé des quatre éléments constitutifs du bâti}

\paragraph{1. La fixation au sol}

La fixation au sol est le premier critère de qualification du bâti. Une construction est considérée comme fixée au sol lorsqu’elle est intégrée de manière stable ou permanente à la parcelle, c’est-à-dire :

\begin{itemize}
	\item qu’elle repose sur des fondations ou un socle en dur,
	\item qu’elle est scellée au béton, soudée ou maçonnée,
	\item qu’elle ne peut être déplacée sans transformation, destruction ou altération de sa structure.
\end{itemize}

Ce critère permet de distinguer le bâti des installations simplement posées sur le sol. Ainsi :
\begin{itemize}
	\item une maison construite en parpaings est clairement fixée au sol ;
	\item un hangar en métal scellé dans des plots béton est également considéré comme bâti ;
	\item à l’inverse, un kiosque posé sur des madriers ou un conteneur reposant sans scellement conservent leur caractère mobile, donc non imposable au titre de la TFPB.
\end{itemize}

Cette distinction est essentielle dans les contextes urbains et périurbains togolais où l’on rencontre de nombreuses installations hybrides ou semi-mobiles. C’est au regard du mode de fixation que l’inspecteur tranche.

\paragraph{2. La durabilité ou permanence}

La durabilité renvoie à la capacité de la construction à demeurer dans le temps. Une construction est considérée comme durable lorsqu’elle :

\begin{itemize}
	\item présente une structure solide ou semi-solide,
	\item est conçue pour persister plusieurs années,
	\item n’est pas destinée à être démontée rapidement,
	\item ne relève pas d’un usage strictement temporaire (chantier, foire, événement).
\end{itemize}

Ce critère implique que :
\begin{itemize}
	\item un magasin construit en matériaux définitifs possède un caractère durable ;
	\item une plateforme bétonnée constitue un ouvrage permanent ;
	\item une paillote démontable, à l’inverse, est un ouvrage éphémère destiné à disparaître ou à être déplacé.
\end{itemize}

La durabilité s’apprécie à la fois par la nature des matériaux (béton, brique, tôle fixée, charpente solide) et par l’intention apparente de permanence.

\paragraph{3. Le résultat d’un travail humain}

Ce critère permet d’éliminer de l’assiette les éléments naturels (grottes, rochers, cavités), même lorsqu’ils sont utilisés économiquement. Une construction est un bâti lorsqu’elle résulte d’une intervention humaine volontaire, même minimale.

Cela inclut :

\begin{itemize}
	\item les constructions classiques (maisons, immeubles, bureaux),
	\item les ouvrages maçonnés (puits, latrines, garages),
	\item les plateformes ou dallages créés par l’homme,
	\item les dépendances et annexes bâties.
\end{itemize}

La finalité est de s’assurer que la valeur imposable provient bien d’un investissement humain qui crée une plus-value économique et non d’un état naturel du terrain.

\paragraph{4. L’aptitude à l’usage ou à l’avantage économique}

Il s’agit du critère le plus important en fiscalité. L’imposition ne repose pas sur l’occupation réelle, mais sur la capacité du bien à procurer un avantage, même potentiel.

Une construction est considérée comme bâtie dès qu’elle :

\begin{itemize}
	\item peut être habitée, louée ou utilisée,
	\item peut accueillir une activité économique (commerce, artisanat, industrie),
	\item procure un bénéfice indirect (stockage, bureaux, usage administratif).
\end{itemize}

Ce critère implique que :
\begin{itemize}
	\item un immeuble vide mais en bon état reste imposable ;
	\item un magasin non encore ouvert mais utilisable est taxé ;
	\item un atelier abandonné mais structurellement solide entre dans la TFPB ;
	\item un logement inoccupé n’est pas une cause d’exonération.
\end{itemize}

Cette notion d’aptitude à l’usage rejoint l’idée, défendue en doctrine, que la TFPB taxe la potentialité économique du bâtiment, et non son exploitation effective.

\subsection{La notion d’achèvement fiscal}

La qualification du bâti dépend aussi de son état d’achèvement fiscal, qui est distinct de l’achèvement technique.
Un bâtiment est considéré comme « achevé » fiscalement dès qu’il peut être utilisé pour l’usage auquel il est destiné, même si :

\begin{itemize}
	\item les finitions ne sont pas terminées,
	\item l’esthétique n’est pas aboutie,
	\item les installations ne sont pas complètes.
\end{itemize}

Ainsi :
\begin{itemize}
	\item une maison sans peinture demeure imposable si elle est habitable ;
	\item un entrepôt sans carrelage ou sans portes intérieures peut être utilisé pour du stockage ;
	\item un local commercial non encore ouvert mais terminé structurellement est imposable.
\end{itemize}

L’achèvement fiscal répond donc à une logique de fonctionnalité, et non de perfection architecturale.

Un bâtiment est considéré comme non achevé (donc non imposable) seulement lorsqu’il est objectivement inhabitable ou inutilisable : absence de toiture, murs non terminés, absence d'accès, structure non stabilisée.

La frontière entre achevé et non achevé étant parfois ténue, c’est l’inspecteur du cadastre qui doit exercer un jugement professionnel fondé sur l’observation directe et les principes administratifs internes.

\subsection{Les constructions assimilées aux propriétés bâties}

Certaines installations ne constituent pas des bâtiments au sens strict, mais elles sont fiscalement assimilées au bâti car elles présentent les caractéristiques suivantes : permanence, fixation au sol, usage économique, contribution à la valeur du bien.

Il s’agit notamment :

\begin{itemize}
	\item des piscines maçonnées,
	\item des terrasses et plateformes bétonnées,
	\item des garages et dépendances fixes,
	\item des hangars soudés ou scellés,
	\item des blocs sanitaires en dur,
	\item des chambres froides ou installations industrielles fixées.
\end{itemize}

Ces constructions constituent ce que l’on appelle l’annexe bâtie, qui doit être intégrée dans la base imposable du bien principal. Leur omission peut générer des écarts de VLC importants.

\subsection{Le local comme unité d’évaluation fiscale}

L’unité d’évaluation de la TFPB est le local, c’est-à-dire l’espace clos et couvert affecté à un usage principal. Cette unité est fondamentale pour :

\begin{itemize}
	\item structurer les relevés cadastraux,
	\item calculer la surface pondérée,
	\item appliquer les coefficients d’affectation,
	\item ventiler les usages mixtes,
	\item attribuer les exonérations,
	\item déterminer les taux applicables.
\end{itemize}

Un bâtiment peut contenir un ou plusieurs locaux, chacun pouvant relever d’une affectation différente : habitation, commerce, atelier, bureau, local technique, espace cultuel, etc.

Le classement par local permet d’obtenir une assiette fiable, précise et conforme aux principes de la fiscalité réelle.

\subsection{Les bâtiments à usage mixte}

Dans les villes togolaises, les bâtiments à usage mixte sont fréquents : boutique au rez-de-chaussée, habitation à l’étage, local de stockage à l’arrière, etc.

La règle fiscale est claire :

\textbf{Chaque partie du bâtiment est imposée selon son usage propre.}

Ainsi, un même bâtiment peut être, simultanément :

\begin{itemize}
	\item habitation (coefficient habitation),
	\item commerce (coefficient commercial),
	\item bureau (coefficient administratif),
	\item atelier (coefficient industriel).
\end{itemize}

L’inspecteur doit donc ventiler les surfaces et évaluer chaque local séparément pour éviter les erreurs d’assiette.

\subsection{Constructions exclues de la TFPB}

Ne relèvent pas de la TFPB :

\begin{itemize}
	\item les constructions provisoires ou démontables,
	\item les structures mobiles (kiosques posés, conteneurs non fixés),
	\item les baraques de chantier,
	\item les ouvrages dépourvus d’utilité autonome (mur isolé, fosses non fonctionnelles),
	\item les bâtiments en ruine ou totalement impropres à l’usage.
\end{itemize}

Toutefois, dans la pratique, de nombreuses installations prétendument « provisoires » sont en réalité permanentes. Une vérification rigoureuse est donc indispensable.

\subsection{Cas togolais particuliers}

Plusieurs situations sont typiques du terrain togolais :

\begin{itemize}
	\item les kiosques métalliques scellés : souvent déclarés mobiles, mais fiscalement imposables ;
	\item les cuisines extérieures en dur : fréquemment omises dans le relevé cadastral ;
	\item les latrines maçonnées : doivent être intégrées car elles augmentent la valeur du bien ;
	\item les plateformes industrielles : très répandues dans les zones d'activité, imposables au titre d’annexe bâtie ;
	\item les bâtiments « abandonnés » mais utilisables : continuent d’être imposés tant que l’usage est possible.
\end{itemize}

Ces réalités montrent l’importance du travail en terrain de l’inspecteur.
