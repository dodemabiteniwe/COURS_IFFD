\chapter{Fondements, cadre légal et règles communes}
\section*{Introduction}
Les taxes foncières constituent l’une des composantes essentielles de la fiscalité immobilière togolaise. Elles incarnent la traduction concrète du principe selon lequel la propriété foncière — source de richesse durable, localisée et stable — doit contribuer à l’effort collectif de financement public. Conformément aux dispositions du Code Général des Impôts (CGI, 2025) et du Livre des Procédures Fiscales (LPF), les taxes foncières sont des impôts directs, réels et annuels perçus principalement au profit des collectivités territoriales, avec gestion centralisée par l’Office Togolais des Recettes (OTR).\\

\noindent Ce premier chapitre a pour objectif de poser les fondements juridiques, économiques et techniques de la fiscalité foncière, en définissant son cadre légal, ses caractéristiques, ses acteurs, ainsi que les règles communes applicables aux propriétés bâties et non bâties.

\section{Définition, nature et caractéristiques juridiques}

\subsection{Définition générale et fondement légal}

Les \textbf{taxes foncières} constituent, selon le \textit{Code Général des Impôts du Togo (CGI, 2025)}, des \textbf{impôts directs, réels et annuels} établis sur la détention ou la propriété d’un bien immobilier. 
Elles visent non pas le revenu effectivement perçu par le propriétaire, mais le \textit{revenu potentiel} que ce bien serait susceptible de produire dans des conditions normales de location.

\noindent Elles comprennent deux grandes catégories définies par le législateur :
\begin{itemize}
	\item la \textbf{Taxe Foncière sur les Propriétés Bâties (TFPB)}, applicable aux constructions et dépendances fixées au sol ;
	\item la \textbf{Taxe Foncière sur les Propriétés Non Bâties (TFPNB)}, applicable aux terrains nus, sols agricoles ou à bâtir.
\end{itemize}

\noindent La taxe foncière constitue  un impôt de détention du capital foncier, frappant la propriété immobilière indépendamment de l’exploitation qui en est faite. 
Elle repose donc sur la \textbf{valeur du patrimoine} et non sur le flux de revenus. Cette conception est reprise par l'article 257 du \textbf{CGI}, qui précise :
\begin{quote}
	\og  Il est institué au profit des budgets de l’Etat et des collectivités locales une taxe foncière 
	due sur les propriétés bâties et non bâties. \fg{}
\end{quote}

Ainsi, la taxe foncière découle directement du \textbf{droit de propriété} et de la \textbf{domination économique} qu’il confère sur un bien immeuble, indépendamment de son usage ou de sa rentabilité effective.

\subsection{Fondement doctrinal : impôt de quotité et instrument d’équilibre territorial}

Dans la doctrine fiscale, la taxe foncière appartient à la catégorie des \textbf{impôts de quotité} ou \textbf{impôts réels}, par opposition aux impôts de répartition (ou personnels) fondés sur la capacité contributive du contribuable. Il s’agit d’un impôt attaché à la chose elle-même (le bien foncier), qui en mesure la valeur indépendamment de la personne qui la possède.\\

\noindent L’impôt foncier est donc \textbf{objectif}, car il s’attache à la valeur de la propriété et non à la richesse globale du propriétaire. 
Ce principe garantit :
\begin{itemize}[font=\color{magenta} \normalsize, label=\ding{109}]
	\item une \textbf{égalité horizontale} entre les propriétaires de biens de même nature et de même valeur ;
	\item une \textbf{prévisibilité du rendement fiscal} pour l’État et les collectivités ;
	\item une \textbf{stabilité de l’assiette} sur le long terme.
\end{itemize}

\noindent Le cadastre, en identifiant et décrivant le bien foncier, transforme l’espace en matière imposable. L’impôt foncier repose sur la territorialisation de la richesse. 
Ainsi, la fiscalité foncière lie étroitement le \textbf{territoire} à la \textbf{fiscalité} : le bien devient à la fois un \textit{élément de richesse} et un \textit{repère administratif} dans l’assiette fiscale locale.

\subsection{La double nature de la taxe foncière : fiscale et économique}

\paragraph{1. Une nature fiscale : impôt local et permanent.}
La taxe foncière est avant tout un \textbf{impôt local}, inscrit dans le cadre de la décentralisation financière. 
Son produit est versé aux \textbf{communes}, qui en font une ressource essentielle pour financer leurs services publics (voirie, propreté, éclairage, écoles primaires, etc.). 
Son caractère \textbf{annuel et permanent} assure la stabilité des recettes communales et une meilleure prévisibilité budgétaire. 
L’\textbf{Office Togolais des Recettes (OTR)} gère l’ensemble du cycle fiscal (assiette, liquidation, recouvrement) conformément au \textbf{Livre des Procédures Fiscales (LPF, 2025)}.

\paragraph{2. Une nature économique : outil d’aménagement et de justice sociale.}
Au-delà de son rôle fiscal, la taxe foncière joue une fonction \textbf{économique et incitative} :
\begin{itemize}[font=\color{magenta} \normalsize, label=\ding{109}]
	\item elle \textbf{décourage la rétention foncière} et la spéculation sur les terrains non bâtis ;
	\item elle \textbf{incite à la construction et à la mise en valeur du patrimoine immobilier} ;
	\item elle favorise une \textbf{meilleure utilisation du sol urbain} et soutient les politiques d’aménagement du territoire.
\end{itemize}
Elle traduit une philosophie de la fiscalité locale : le propriétaire, bénéficiaire des services publics, doit contribuer à leur financement proportionnellement à la valeur de son bien. La taxe foncière est donc le prix de la stabilité territoriale et de la jouissance du capital immobilier.

\subsection{Caractéristiques juridiques principales}

Les taxes foncières possèdent plusieurs traits distinctifs qui conditionnent leur application :



\begin{enumerate}
	\item \textbf{Impôt réel (ou de quotité)} : 
	L’impôt foncier vise le bien, non la personne. 
	L’assiette dépend des \textit{caractéristiques physiques et économiques} du bien (superficie, affectation, valeur locative cadastrale). 
	Ce principe, inscrit à le CGI, fonde la neutralité de la taxe : elle s’applique indépendamment de la situation du contribuable.
	
	\item \textbf{Impôt direct} : 
	Il est directement mis à la charge du redevable désigné sur le rôle d’imposition. 
	Il ne peut être ni répercuté sur autrui ni récupéré indirectement, contrairement à la TVA.
	
	\item \textbf{Impôt annuel} : 
	L’imposition est due pour l’année entière, sur la base de la situation du bien au 1\textsuperscript{er} janvier. 
	Une mutation intervenue après cette date n’a d’effet qu’à compter de l’année suivante. 
	Ce principe, confirmé par la doctrine (\textsc{Schmidt \& Kornprobst}), assure la sécurité juridique des rôles d’imposition.
	
	\item \textbf{Impôt local} : 
	Le produit est affecté au financement des \textbf{collectivités territoriales}, conformément à la Loi n\textsuperscript{o} 2019-006 sur la décentralisation. 
	Les communes fixent le taux dans les limites prévues par le CGI, tandis que l’OTR assure le recouvrement et le reversement.
	
	\item \textbf{Impôt à assiette stable} : 
	Contrairement aux impôts sur les revenus, la base foncière varie peu dans le temps. 
	Sa stabilité en fait une ressource fiable pour les budgets locaux, à condition que les \textbf{valeurs locatives cadastrales} soient périodiquement révisées.
\end{enumerate}

\subsection{Effets juridiques et administratifs de la taxe foncière}

\begin{enumerate}
	\item \textbf{Création d’un lien fiscal durable entre le bien et la collectivité.}  
	Le bien immobilier devient une unité fiscale enregistrée dans la \textit{matrice cadastrale} et identifiée par une référence unique. 
	Ce lien se maintient indépendamment des mutations de propriété.
	
	\item \textbf{Transmission automatique de la charge fiscale.}  
	En cas de cession, la taxe due pour l’année en cours reste à la charge du propriétaire au 1\textsuperscript{er} janvier (art. 270 CGI). 
	Les clauses contractuelles entre vendeur et acheteur sont inopposables à l’administration.
	
	\item \textbf{Obligation déclarative du propriétaire.}  
	Tout propriétaire doit déclarer :
	\begin{itemize}
		\item toute \textbf{construction nouvelle} ou modification substantielle dans les 04 mois suivant l’achèvement ;
		\item tout \textbf{changement d’usage} ou de propriétaire. 
	\end{itemize}
	Ces obligations, prévues par le LPF (2025), garantissent la mise à jour de la base d’imposition.
	
	\item \textbf{Effet d’équité et de transparence.}  
	Une assiette bien définie renforce la confiance du contribuable et la légitimité de l’impôt. 
	À l’inverse, une évaluation inexacte crée un déséquilibre fiscal et mine la crédibilité de l’administration.
\end{enumerate}

\subsection{Synthèse pédagogique}

\begin{center}
	\begin{tabular}{|p{4cm}|p{4cm}|p{7cm}|}
		\hline
		\textbf{Caractéristiques} & \textbf{Nature} & \textbf{Conséquences pratiques} \\
		\hline
		Réelle & Porte sur le bien et non sur la personne & Assiette fondée sur les caractéristiques physiques et économiques du bien \\
		\hline
		Directe & À la charge du propriétaire ou de l’usufruitier & Nomination nominative dans le rôle d’imposition \\
		\hline
		Annuelle & Établie chaque année au 1\textsuperscript{er} janvier & Stabilité de l’assiette et sécurité juridique \\
		\hline
		Locale & Produit destiné aux communes & Ressource pour le financement local \\
		\hline
		Permanente & Repose sur un patrimoine durable & Assiette peu sensible aux aléas économiques \\
		\hline
	\end{tabular}
\end{center}


La taxe foncière, par sa nature d’\textbf{impôt réel, direct, local et annuel}, représente le fondement de la fiscalité immobilière togolaise. 
Elle repose sur un double socle : \textbf{le droit de propriété} et \textbf{la connaissance cadastrale}. 
Comme le souligne la doctrine, sans un cadastre fiable, il n’existe pas d’impôt foncier équitable.
C’est donc à travers la rigueur de la définition, la clarté de la base légale et la précision de la donnée cadastrale que se construit l’efficacité du système fiscal foncier.

\section{Le cadre légal et institutionnel}

Le régime de la taxe foncière au Togo repose sur un ensemble articulé de normes fiscales, de règles procédurales et d’organes administratifs spécialisés. 
Comprendre ce cadre est indispensable pour maîtriser l’assiette, la liquidation et le recouvrement des taxes foncières. 
Il s’agit d’un système global où le \textbf{cadastre}, le \textbf{CGI}, le \textbf{LPF}, les \textbf{lois de finances} et les \textbf{collectivités territoriales} interagissent pour produire un impôt cohérent, équitable et sécurisé.

\subsection{Les sources normatives de la fiscalité foncière}

Le fondement juridique de la taxe foncière repose sur trois catégories principales : 
(1) les textes fiscaux, 
(2) les textes sectoriels complémentaires, 
(3) les actes administratifs d’application.

\subsubsection{Le Code Général des Impôts (CGI, 2025)}

Le \textbf{CGI 2025} constitue la base législative principale. 
Les articles 257 à 287 définissent :
\begin{itemize} [font=\color{magenta} \normalsize, label=\ding{109}]
	\item le \textbf{champ d’application} des taxes foncières ;
	\item la \textbf{détermination de la base imposable} ;
	\item les \textbf{exonérations permanentes et temporaires} ;
	\item le \textbf{fait générateur et la période d’imposition} ;
	\item les \textbf{taux applicables} ;
	\item la \textbf{répartition du produit} au profit des collectivités territoriales.
\end{itemize}

\noindent La taxe foncière est un impôt réel fondé sur la détention du capital foncier, indépendamment du revenu qu'il procure.

\subsubsection{Le Livre des Procédures Fiscales (LPF)}

Le \textbf{LPF 2025} précise les modalités :
\begin{itemize} [font=\color{magenta} \normalsize, label=\ding{109}]
	\item d’assiette (déclarations, obligations des propriétaires) ;
	\item de liquidation (émission et validation des rôles) ;
	\item de recouvrement (paiement volontaire ou forcé) ;
	\item de sanction en cas de manquement ;
	\item de tenue et de mise à jour des fichiers cadastraux et fiscaux.
\end{itemize}

\noindent Le LPF garantit une application uniforme, transparente et légale de l’impôt foncier.

\subsubsection{Les lois de finances}

Les \textbf{Lois de finances annuelles} interviennent pour :
\begin{itemize} [font=\color{magenta} \normalsize, label=\ding{109}]
	\item fixer les coefficients ou mécanismes de \textbf{revalorisation des valeurs locatives} ;
	\item ajuster les \textbf{abattements et exonérations temporaires} ;
	\item préciser les \textbf{taux planchers et plafonds} des taxes locales ;
	\item introduire des mesures incitatives (logement social, construction neuve, etc.).
\end{itemize}

\noindent Elles constituent un instrument de politique économique permettant d’inciter ou de freiner certains comportements fonciers.

\subsubsection{Les textes sectoriels complémentaires}

Certains textes influencent indirectement la fiscalité foncière :
\begin{itemize}[font=\color{magenta} \normalsize, label=\ding{109}]
	\item \textbf{Loi 2019-006 sur la décentralisation} : rôle des communes et affectation du produit fiscal ;
	\item \textbf{Code de l’urbanisme et de la construction} : qualification des sols, zonage, permis de construire ;
	\item \textbf{Loi sur le foncier et les domaines} : définition de la propriété, titres fonciers, servitudes ;
	\item \textbf{Documents d’urbanisme (PDAU, PDU)} : vocation des sols (agricole, urbanisable, etc.).
\end{itemize}

Ces textes sont déterminants pour la \textbf{classification fiscale} des propriétés (habitation, commerce, industrie, agricole).

\subsubsection{Les actes administratifs de l’OTR}

L’OTR produit régulièrement :
\begin{itemize} [font=\color{magenta} \normalsize, label=\ding{109}]
	\item notes de service ;
	\item instructions générales ;
	\item guides méthodologiques ;
	\item barèmes techniques d’évaluation ;
	\item modèles de fiches cadastrales et fiscales.
\end{itemize}

\noindent Un impôt réel n’est équitable que si les procédures qui en garantissent l’assiette sont harmonisées.

\subsection{Le cadre institutionnel de la fiscalité foncière}

La gestion de la taxe foncière fait intervenir plusieurs institutions complémentaires : l’OTR, les communes, les services d’urbanisme, et la chaîne foncière notariale.

\subsubsection{L’Office Togolais des Recettes (OTR)}

L’OTR intervient à trois niveaux :

\begin{itemize}[font=\color{magenta} \large, label=\ding{43}]
	\item  L’assiette (Cadastre et Conservation Foncière)
	
\begin{itemize}[font=\color{magenta} \large, label=\ding{213}]
	\item Identification et délimitation du bien ;
	\item Mesure, classification et affectation cadastrale ;
	\item Détermination de la valeur cadastrale ;
	\item Mise à jour de la matrice cadastrale ;
	\item Vérification des déclarations (construction, démolition, changement d’usage).
\end{itemize}

\noindent Comme le souligne \textsc{Maurin}, le cadastre est la mémoire administrative du territoire ; sans lui aucune base imposable n’est fiable.

\item La liquidation
\begin{itemize}
	\item Vérification et calcul de la base nette imposable (BNI) ;
	\item Application des taux fixés par les textes ;
	\item Gestion des exonérations légales.
\end{itemize}

\item Le recouvrement
\begin{itemize}
	\item Emission des avis d’imposition ;
	\item Encaissement des paiements ;
	\item Suivi des arriérés ;
	\item Application des pénalités et majorations.
\end{itemize}

\end{itemize}


\subsubsection{Les collectivités territoriales (communes)}

Les communes sont les \textbf{bénéficiaires exclusives} du produit des taxes foncières.

Elles jouent un rôle :
\begin{itemize}[font=\color{magenta} \normalsize, label=\ding{109}]
	\item \textbf{normatif} : fixation des taux locaux dans le cadre légal ;
	\item \textbf{administratif} : coopération avec l’OTR pour la mise à jour des informations foncières ;
	\item \textbf{financier} : affectation des recettes aux services publics locaux.
\end{itemize}

La taxe foncière est donc un instrument essentiel de \textbf{décentralisation financière}.

\subsubsection{Les services d’urbanisme et de construction}

Ces services :
\begin{itemize}[font=\color{magenta} \normalsize, label=\ding{109}]
	\item délivrent les permis de construire ;
	\item contrôlent la conformité des constructions ;
	\item transmettent les informations sur les travaux ou changements d’usage ;
	\item participent à la détection des propriétés imposables.
\end{itemize}

Ils jouent un rôle déterminant dans la mise à jour des fichiers cadastraux et fiscaux.

\subsubsection{Les notaires et la chaîne foncière}

Les notaires interviennent dans toutes les mutations immobilières :
\begin{itemize}[font=\color{magenta} \normalsize, label=\ding{109}]
	\item rédaction des actes ;
	\item transmission à la conservation foncière ;
	\item enregistrement fiscal ;
	\item déclaration des mutations.
\end{itemize}

Toute mutation doit donner lieu à une \textbf{mise à jour du redevable} dans le rôle d’imposition.

\subsection{Logique d’ensemble du cadre légal et institutionnel}

La cohérence du système repose sur un principe central :

\begin{quote}
	\textbf{Le cadastre fournit la donnée ; le CGI fixe la règle ; le LPF encadre la procédure ; les communes bénéficient du produit.}
\end{quote}

Ainsi, une bonne application de la taxe foncière nécessite :
\begin{itemize}[font=\color{magenta} \normalsize, label=\ding{109}]
	\item une \textbf{donnée cadastrale fiable} ;
	\item une \textbf{règle fiscale claire et stable} ;
	\item une \textbf{procédure homogène et sécurisée} ;
	\item une \textbf{coordination institutionnelle efficace}.
\end{itemize}

Ce cadre garantit la \textbf{légalité}, la \textbf{sécurité juridique}, la \textbf{justice fiscale} et l’\textbf{efficacité du recouvrement}.

\section{Le champ d'application et le fait générateur}

Le champ d'application et le fait générateur constituent des éléments fondamentaux dans la détermination de l'assiette fiscale. 
Ils répondent aux questions suivantes : 
\textit{Quels biens sont imposables ? Qui doit payer la taxe ? À partir de quand la taxe est-elle due ?}
Une compréhension rigoureuse de ces notions est indispensable pour éviter les erreurs d'assiette, les pertes de recettes et les contestations.

\subsection{Champ d'application matériel}

Le champ d'application matériel désigne l'ensemble des \textbf{biens immobiliers soumis ou non à la taxe foncière}. 
Le \textbf{CGI 2025} distingue deux grandes catégories : les propriétés bâties et les propriétés non bâties.

\subsubsection{Les propriétés bâties (TFPB)}

Une propriété bâtie est toute construction :

\begin{itemize}[font=\color{magenta} \normalsize, label=\ding{109}]
	\item fixée au sol à perpétuelle demeure ;
	\item présentant un caractère de stabilité durable ;
	\item résultant d'un travail humain ;
	\item susceptible de procurer un revenu ou un avantage économique.\\
\end{itemize}

Relèvent de cette catégorie :

\begin{itemize}[font=\color{magenta} \normalsize, label=\ding{109}]
	\item les immeubles d’habitation (maisons, immeubles collectifs, dépendances) ;
	\item les immeubles non résidentiels (bureaux, commerces, ateliers, usines, entrepôts) ;
	\item les constructions assimilées : piscines maçonnées, hangars fixés au sol, plateformes industrielles ;
	\item les bâtiments en état d’être utilisés, même sans finitions esthétiques.\\
\end{itemize}

Certaines constructions sont exclues :

\begin{itemize}[font=\color{magenta} \normalsize, label=\ding{109}]
	\item installations démontables, provisoires ou mobiles ;
	\item structures non fixées au sol ou à vocation temporaire.
\end{itemize}

\subsubsection{Les propriétés non bâties (TFPNB)}

Une propriété non bâtie désigne tout bien immobilier non couvert par une construction imposable. 
Il s’agit notamment des :

\begin{itemize}[font=\color{magenta} \normalsize, label=\ding{109}]
	\item terrains nus situés en zones urbaines ou rurales ;
	\item terrains agricoles (cultivés ou cultivables), plantations, forêts privées ;
	\item terrains à bâtir dans des zones urbanisables  ;
	\item sols d'exploitation : carrières, mines, briqueteries, étangs ;
	\item dépendances non attenantes à un bâti (cours, jardins, parkings).
\end{itemize}

\subsubsection{Influence du zonage et de l’urbanisme}

La fiscalité foncière dépend fortement des documents d'urbanisme. Le zonage permet de déterminer :

\begin{itemize}[font=\color{magenta} \normalsize, label=\ding{109}]
	\item si un terrain est constructible ;
	\item s'il relève d'une vocation agricole ou naturelle ;
	\item ou s'il s’agit d'une zone industrielle ou spéciale.
\end{itemize}

L'inspecteur du cadastre doit vérifier la cohérence entre la vocation urbanistique et l’affectation fiscale du bien.

\subsection{Champ d'application personnel}

Selon l’article 260 du CGI, sont imposables :

\begin{itemize}[font=\color{magenta} \normalsize, label=\ding{109}]
	\item le \textbf{propriétaire} au 1\textsuperscript{er} janvier ;
	\item l’\textbf{usufruitier} en cas de démembrement de propriété ;
	\item le \textbf{preneur à bail emphytéotique} ou à construction ;
	\item le \textbf{concessionnaire} ou titulaire d’un droit réel ;
	\item les \textbf{coindivisaires ou copropriétaires}, solidairement.
\end{itemize}

Les occupants irréguliers ne sont pas redevables, mais leur présence peut modifier l’affectation réelle du bien et affecter les exonérations.

Les clauses de répartition de la charge de la taxe entre vendeur et acquéreur sont inopposables à l'administration fiscale.

\subsection{Le fait générateur et la période d'imposition}

Le fait générateur correspond au moment où naît l'obligation fiscale. 
Selon le CGI :

\begin{itemize}[font=\color{magenta} \normalsize, label=\ding{109}]
	\item la taxe est due pour \textbf{toute l’année} par le propriétaire au 1\textsuperscript{er} janvier ;
	\item la situation du bien à cette date détermine l'imposition.
\end{itemize}

\subsubsection{Conséquences pratiques}

\begin{itemize}[font=\color{magenta} \normalsize, label=\ding{236}]
	\item Une \textbf{vente en cours d’année} n’a pas d’effet sur la taxe due : le vendeur reste redevable pour toute l’année.
	\item Une \textbf{démolition} en février maintient la TFPB pour l’année en cours.
	\item Une \textbf{construction achevée en décembre} devient imposable dès le 1\textsuperscript{er} janvier suivant.
\end{itemize}

\subsubsection{Définition de l’achèvement}

Une construction est considérée comme achevée lorsqu’elle est en \textbf{état d’être utilisée}, même si les finitions (peinture, installations décoratives) ne sont pas terminées. Cette approche fonctionnelle est conforme à la doctrine, l’achèvement d’un immeuble s’apprécie selon sa fonctionnalité et non selon sa finition esthétique. 

\subsection{Cas pratiques}

\paragraph{Cas 1 : Bâti en achèvement.} 
Un immeuble habitable depuis octobre mais non achevé est imposable à la TFPB dès le 1\textsuperscript{er} janvier suivant.

\paragraph{Cas 2 : Terrain agricole.}
Un terrain de 5 ha exploité directement par son propriétaire bénéficie d’une exonération de TFPNB.

\paragraph{Cas 3 : Mutation non déclarée.}
Un immeuble vendu en février reste imposable au vendeur pour toute l’année.

\paragraph{Cas 4 : Copropriété.}
Un immeuble détenu en indivision par 4 héritiers implique une responsabilité solidaire des co-indivisaires.\\



\section{Les exonérations et exclusions}

Les exonérations constituent l'un des aspects les plus sensibles de la fiscalité foncière. 
Elles affectent directement le rendement fiscal, l'équité entre contribuables et la confiance du public dans l'administration. 
Une mauvaise gestion des exonérations peut entraîner des pertes de recettes, des erreurs de rôle et des incompréhensions prolongées. 
Il est donc essentiel que l'inspecteur du cadastre maîtrise parfaitement ces mécanismes.

\subsection{Aperçu des exonérations permanentes}

\noindent Le régime des taxes foncières prévoit un ensemble d’exonérations permanentes, accordées par la loi à certaines catégories de biens en raison de leur statut, de leur affectation ou de leur utilité sociale. 
Ces exonérations constituent des dérogations au principe général d’imposition et doivent, à ce titre, être strictement encadrées.\\

\noindent De manière générale, bénéficient d’exonérations permanentes les biens appartenant à l’État et aux collectivités territoriales lorsqu’ils sont affectés à un service public non lucratif, les édifices exclusivement dédiés à l’exercice du culte, ainsi que certains établissements d’enseignement ou de santé poursuivant une mission d’intérêt général sans but lucratif. 
Sont également concernés certains biens à vocation agricole ou patrimoniale, lorsque leur usage effectif ne relève pas d’une logique de rente foncière.\\

\noindent Dans tous les cas, le critère déterminant n’est pas uniquement la nature juridique du bien ou la qualité de son propriétaire, mais surtout son \textbf{usage réel}. 
Toute affectation lucrative ou toute utilisation étrangère à la mission justifiant l’exonération est susceptible d’entraîner la perte du bénéfice de celle-ci.\\

\noindent L’étude détaillée des différentes exonérations permanentes et de leurs conditions d’application fera l’objet d’analyses approfondies dans les chapitres consacrés respectivement aux propriétés bâties et aux propriétés non bâties.


\subsection{Aperçu des exonérations temporaires}

Les exonérations temporaires constituent des mécanismes incitatifs mis en place par le législateur afin d’encourager certains comportements jugés socialement ou économiquement souhaitables, tels que l’investissement immobilier, la construction de logements ou la réalisation de projets d’intérêt public. 
Elles se distinguent des exonérations permanentes par leur caractère limité dans le temps et par leur rattachement à une situation transitoire.\\

\noindent Ces exonérations concernent principalement les constructions nouvelles, certains logements à vocation sociale, ainsi que les terrains ou immeubles faisant l’objet de projets de viabilisation ou d’aménagement. 
Leur bénéfice est généralement subordonné au respect d’obligations déclaratives, à la conformité des travaux aux autorisations administratives et à l’usage effectif déclaré du bien.\\

\noindent Les conditions, la durée et les effets de ces exonérations feront l’objet d’un examen détaillé dans les chapitres consacrés aux propriétés bâties et non bâties.

\subsection{Exonérations partielles et proportionnelles}

Outre les exonérations totales, le régime des taxes foncières prévoit des mécanismes d’exonération partielle ou proportionnelle. 
Ces dispositifs ne soustraient pas intégralement le bien à l’impôt, mais en limitent l’assiette en fonction de critères liés à l’usage, à la consistance ou à la destination du bien.\\

\noindent Ils peuvent notamment s’appliquer :
\begin{itemize}
	\item à une fraction seulement de l’immeuble lorsque celui-ci présente des usages différenciés ;
	\item à une partie de la valeur imposable par le biais d’abattements ;
	\item à une période transitoire liée à un changement d’affectation ou de statut.
\end{itemize}

\noindent Ces mécanismes traduisent une volonté d’adaptation de l’impôt à la réalité économique et fonctionnelle des biens, tout en préservant le principe d’équité fiscale.

\subsection{Exclusions d’assiette}

Certaines propriétés ne relèvent pas du champ d’application des taxes foncières et sont, à ce titre, exclues de l’assiette fiscale. 
Il ne s’agit pas d’exonérations au sens strict, mais de biens qui, par leur nature même, ne peuvent être imposés.\\

\noindent Sont notamment concernés les éléments relevant du domaine public inaliénable, les infrastructures de voirie, les espaces naturels protégés ou les terrains grevés de servitudes empêchant toute valorisation économique. 


\subsection{Enjeux et risques liés à la gestion des exonérations}

La gestion des exonérations, qu’elles soient permanentes, temporaires ou partielles, constitue un enjeu majeur pour la fiscalité foncière locale. 
Une application imprécise ou excessive de ces régimes peut entraîner une contraction significative de l’assiette fiscale, des inégalités injustifiées entre contribuables et une fragilisation de la crédibilité de l’administration.\\

\noindent À l’inverse, une maîtrise rigoureuse des règles d’exonération, fondée sur l’analyse de l’usage réel des biens et le respect strict des conditions légales, permet de concilier objectifs incitatifs et impératifs de mobilisation durable des ressources locales. 







\section{Le rôle de l’Inspecteur du Cadastre}

Le rôle de l'inspecteur du cadastre est central dans l'ensemble du cycle de gestion de la taxe foncière. 
Il est l’acteur regroupant les compétences techniques, juridiques et administratives nécessaires pour garantir la fiabilité et l’équité de l’assiette foncière.

\subsection{Dimension technique : production et mise à jour de la donnée}

L’inspecteur du cadastre doit :
\begin{itemize}[font=\color{magenta} \normalsize, label=\ding{219}]
	\item identifier et délimiter les biens ;
	\item vérifier la cohérence des plans cadastraux ;
	\item mettre à jour les surfaces, affectations et usages ;
	\item réaliser ou contrôler les levés, mesures et géolocalisations ;
	\item entretenir la matrice cadastrale (base géométrique + base juridique).
\end{itemize}

Ces données constituent la \textbf{matière imposable} essentielle à la TFPB et à la TFPNB.

\subsection{Dimension juridique : application des règles fiscales}

L’inspecteur doit :
\begin{itemize}[font=\color{magenta} \normalsize, label=\ding{219}]
	\item qualifier le bien (bâti/non bâti, mixte) ;
	\item identifier le redevable légal (propriétaire, usufruitier, emphytéote) ;
	\item vérifier l’éligibilité aux exonérations ;
	\item contrôler la conformité des déclarations ;
	\item assurer la sécurité juridique des dossiers d’assiette.
\end{itemize}

Il applique le \textbf{CGI}, le \textbf{LPF}, les circulaires de l’OTR et la doctrine fiscale.

\subsection{Dimension administrative : gestion et traçabilité}

L’inspecteur joue un rôle clé dans :
\begin{itemize}[font=\color{magenta} \normalsize, label=\ding{219}]
	\item la mise à jour des fichiers d’assiette ;
	\item la synchronisation des informations entre services ;
	\item la préparation des éléments du rôle d’imposition ;
	\item le suivi des déclarations obligatoires.
\end{itemize}

Il veille à la \textbf{traçabilité}, à la \textbf{cohérence} et à la \textbf{qualité de la base de données cadastrales et fiscales}.

\subsection{Dimension relationnelle : interface avec les usagers et les partenaires}

L’inspecteur doit :
\begin{itemize}[font=\color{magenta} \normalsize, label=\ding{219}]
	\item informer le contribuable sur ses droits et obligations ;
	\item expliquer la fiscalité foncière et accompagner les démarches ;
	\item collaborer avec les notaires, les géomètres, les services domaniaux et d’urbanisme ;
	\item participer à la sensibilisation et à la communication fiscale locale.
\end{itemize}

Un bon inspecteur améliore la compréhension et l’acceptation de l’impôt.

\subsection{Dimension stratégique : modernisation du cadastre et innovations}

L’inspecteur contribue à :
\begin{itemize}[font=\color{magenta} \normalsize, label=\ding{219}]
	\item la digitalisation du cadastre (SIG, bases informatisées, géolocalisation) ;
	\item l’utilisation d’outils mobiles et de solutions numériques OTR ;
	\item aux opérations de recensement foncier ;
	\item l’amélioration continue des procédures d’assiette ;
	\item la veille sur les réformes foncières et fiscales.
\end{itemize}









