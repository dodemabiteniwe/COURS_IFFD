\documentclass[Cours_contentieuxTF2026.tex]{subfiles}

\begin{document}
	
	\chapter{Les Voies de Recours Juridictionnel en Matière de Taxes Foncières}
	
	La phase administrative du contentieux des taxes foncière s'achève par une décision de l'autorité fiscale. Lorsque cette décision ne donne pas entière satisfaction au redevable, la loi lui garantit le droit de porter le litige devant un juge indépendant. L'intervention du juge modifie l'équilibre des pouvoirs : l'Administration perd son privilège du préalable pour devenir une simple partie au procès, soumise au principe du contradictoire.
	
	\section{La compétence juridictionnelle : La dualité du contentieux des taxes foncière}
	
	En droit du contentieux, la détermination de la juridiction compétente (\textit{ratione materiae} et \textit{ratione loci}) est l'étape primordiale de l'action en justice. Saisir le mauvais tribunal entraîne une fin de non-recevoir d'ordre public : le juge doit soulever son incompétence d'office, ce qui peut s'avérer fatal pour le contribuable si les délais de recours sont entre-temps expirés.
	
	Le législateur togolais a organisé le contentieux fiscal autour d'une dualité juridictionnelle complexe. En matière immobilière, le juge compétent varie selon que l'on conteste la taxe annuelle (l'impôt direct) ou la taxe sur la transaction (le droit d'enregistrement).
	
	\subsection{Le contentieux de l'annuité (La Taxe Foncière) : La compétence administrative directe}
	
	La taxe foncière sur les propriétés bâties et non bâties appartient à la catégorie des impôts directs. À ce titre, elle obéit à un régime dérogatoire au droit commun de l'organisation judiciaire.
	
	\begin{enumerate}
		\item \textbf{L'éviction du Tribunal de Première Instance (TPI) :} \\
		En principe, tout litige naissant naît devant le juge de première instance. Or, l'\textbf{article 380 du LPF} crée un « saut juridictionnel ». Il dispose que les recours contre les décisions de l'Administration fiscale en matière d'impôts directs sont portés \textbf{directement} devant la Cour d'Appel siégeant en Chambre Administrative.
		\begin{itemize}
			\item \textit{Justification :} Cette concentration du contentieux de l'assiette au niveau de la Cour d'Appel vise à confier des matières techniques et à forts enjeux financiers à des magistrats expérimentés, tout en désengorgeant les tribunaux d'instance.
		\end{itemize}
		
		\item \textbf{La règle du « Premier et dernier ressort » :} \\
		C'est la spécificité la plus redoutable de cette procédure. La Cour d'Appel ne se prononce pas ici comme une juridiction du second degré, mais comme une juridiction d'exception statuant en premier et dernier ressort sur le fond (conformément à l'ordonnance n° 78-35).
		\begin{itemize}
			\item \textit{Conséquence stratégique :} Le contribuable et l'Administration n'ont droit qu'à un seul examen des faits (les surfaces, les évaluations, la catégorie de l'immeuble).
			\item \textit{L'unique voie de recours :} La décision de la Cour d'Appel n'est susceptible d'aucun appel régulier. La seule voie de réformation possible est le \textbf{pourvoi en cassation} devant la Chambre Administrative de la Cour Suprême. Or, le juge de cassation ne rejuge pas les faits ; il vérifie uniquement si la Cour d'Appel a fait une correcte application de la loi fiscale.
		\end{itemize}
	\end{enumerate}
	
	\subsection{Le contentieux de la mutation (L'enregistrement) : Le retour au juge civil}
	
	La pratique du Cadastre et de la Conservation foncière ne se limite pas aux impôts directs. Lorsqu'un terrain est vendu, donné ou hérité, l'opération donne lieu à la perception de \textbf{droits d'enregistrement}. Si le litige porte sur ces droits (notamment sur la contestation de la \textit{valeur vénale réelle} de l'immeuble retenue par l'inspecteur), la compétence juridictionnelle bascule radicalement.
	
	\begin{enumerate}
		\item \textbf{La compétence matérielle (Le juge civil) :} \\
		Par dérogation expresse à l'organisation de la chambre administrative, l'alinéa 2 de l'article 380 du LPF attribue ce contentieux spécifique aux \textbf{Tribunaux de Première Instance (TPI) qui siègent en chambres civiles}.
		\begin{itemize}
			\item \textit{Philosophie :} Le droit de l'enregistrement est intimement lié au droit civil (contrats de vente, successions, baux). Il est donc logique que le juge civil, protecteur naturel de la propriété privée et des contrats, soit le juge de ces impôts de mutation.
		\end{itemize}
		
		\item \textbf{La compétence territoriale (\textit{Lex rei sitae}) :} \\
		Pour les droits d'enregistrement liés à des biens immobiliers, l'\textbf{article 400 du LPF} consacre la compétence du tribunal du lieu de situation de l'immeuble. Si les biens d'une même exploitation s'étendent sur plusieurs juridictions, le TPI compétent est celui où se trouve le siège de l'exploitation ou la partie générant le plus grand revenu.
		
		\item \textbf{L'anomalie procédurale du TPI (Le jugement sans appel) :} \\
		Il est impératif pour l'étudiant en Master d'identifier le piège tendu par le LPF. Bien que statuant au niveau du TPI (qui normalement admet l'appel), le LPF précise que : \textit{« Les jugements rendus par ces tribunaux en cette matière sont sans appel sauf pourvoi en cassation »}.
		\begin{itemize}
			\item \textit{Analyse :} Même devant le juge civil de première instance, le législateur a supprimé le double degré de juridiction pour accélérer le traitement du contentieux fiscal et sécuriser plus rapidement les recettes de l'État.
		\end{itemize}
	\end{enumerate}
	
\section{Les conditions de recevabilité du recours juridictionnel}

L'accès au prétoire de la Cour d'Appel n'est pas un droit absolu et inconditionnel. Il est subordonné au respect d'un formalisme rigoureux et de délais préfix, édictés par le Livre des Procédures Fiscales (LPF). Ces règles, d'ordre public, s'imposent tant au contribuable qu'au juge, qui a l'obligation de soulever d'office toute cause d'irrecevabilité.

\subsection{L'encadrement strict des délais de saisine (Art. 381 du LPF)}

Le contentieux des taxes foncière est enfermé dans des délais brefs pour des raisons de sécurité juridique et de stabilité des budgets publics. L'État ne peut pas vivre sous la menace perpétuelle d'une annulation de ses recettes. Le délai de saisine varie selon le comportement de l'Administration fiscale lors de la phase préalable.

\begin{enumerate}
	\item \textbf{Le délai normal (En cas de décision expresse) :} \\
	Lorsque l'Administration répond formellement à la réclamation, l'action doit être introduite devant la Cour d'Appel dans un délai de \textbf{deux (02) mois}. Ce délai court à partir du jour de la réception (notification par avis de réception) de la décision de rejet total ou partiel. Une fois ce délai expiré, la forclusion est acquise et l'impôt devient inattaquable.
	
	\item \textbf{Le délai supplétif (En cas de décision implicite) :} \\
	Le législateur a prévu un mécanisme de sauvegarde pour éviter que l'inertie de l'Administration ne prive le contribuable de son droit au juge. Si l'Administration garde le silence, le contribuable peut saisir la Cour d'Appel dans un délai de \textbf{six (06) mois} suivant la date de présentation de sa réclamation initiale. Le silence prolongé vaut décision implicite de rejet, ouvrant ainsi la voie juridictionnelle.
\end{enumerate}

\subsection{Le formalisme de la requête introductive d'instance (Art. 383 du LPF)}

La saisine de la juridiction administrative obéit à une procédure essentiellement écrite. La requête n'est pas une simple lettre de doléances ; c'est un acte de procédure qui fixe le cadre du litige.

\begin{enumerate}
	\item \textbf{Le dépôt et les mentions obligatoires :} \\
	La requête doit être adressée au greffe de la Cour d'Appel (ou du Tribunal de Première Instance en matière de droits d'enregistrement). Elle doit être impérativement signée par son auteur ou son mandataire régulier. Elle doit explicitement contenir :
	\begin{itemize}
		\item \textbf{L'exposé sommaire des faits :} L'historique de l'imposition et de la contestation.
		\item \textbf{Les moyens de droit :} Les arguments juridiques (ex : violation d'une règle d'évaluation cadastrale, fausse application d'un tarif).
		\item \textbf{Les conclusions :} Ce que le demandeur attend précisément du juge (ex : « prononcer la décharge totale de la taxe » ou « ordonner la réduction de la base imposable à hauteur de X francs »).
	\end{itemize}
	
	\item \textbf{La production de la décision attaquée (La règle de la décision préalable) :} \\
	C'est une condition substantielle. La requête doit obligatoirement être accompagnée de l'avis de notification de la décision contestée de l'Administration. En droit administratif, on n'attaque pas un impôt « en l'air » ; on attaque l'acte par lequel l'Administration a refusé de l'annuler. Si la décision préalable n'est pas jointe, la requête est irrecevable.
\end{enumerate}

\subsection{L'intangibilité de l'objet du litige et la liberté des moyens}

Le passage de la phase administrative à la phase juridictionnelle est régi par la règle de l'entonnoir : le litige ne peut pas s'élargir devant le juge.

\begin{itemize}
	\item \textbf{La limitation aux impositions visées initialement :} \\
	Le demandeur ne peut contester devant la Cour d'Appel des impositions différentes de celles qu'il a visées dans sa réclamation préalable à l'Administration.
	\textit{Exemple : Si la réclamation administrative portait uniquement sur la taxe foncière de l'année N, le contribuable ne peut pas profiter de son recours en justice pour demander soudainement l'annulation de la taxe de l'année N-1. Il y a défaut de liaison du contentieux pour l'année N-1.}
	
	\item \textbf{L'ouverture aux conclusions et moyens nouveaux :} \\
	Toutefois, dans la stricte limite du montant du dégrèvement initialement sollicité, le contribuable bénéficie d'une liberté argumentaire. Il peut faire valoir devant le juge des « moyens nouveaux » (des arguments juridiques qu'il n'avait pas pensés à utiliser devant le Commissaire Général), à condition de les formuler explicitement dans sa demande introductive.
\end{itemize}

\subsection{La purge des vices de forme de la réclamation initiale}

L'article 383 du LPF introduit une disposition favorable au contribuable concernant la régularisation. Si l'Administration a rejeté la réclamation initiale pour un vice de forme (par exemple, défaut de production de l'avis d'imposition), ce vice peut être utilement « couvert » (corrigé) dans la requête adressée à la Cour d'Appel.

\begin{description}
	\item[L'exception majeure] \hfill \\
	Le défaut de signature de la réclamation initiale ne peut jamais être couvert. Si la réclamation n'était pas signée, elle était juridiquement inexistante, et le recours juridictionnel qui la suit sera irrecevable.
\end{description}
	
	\section{L'instruction et le jugement devant la Cour d'Appel}
	
	Une fois la Cour d'Appel valablement saisie, la phase purement administrative s'efface pour laisser place au procès juridictionnel. Le juge administratif dirige l'instruction. Dans le cadre du contentieux des taxes foncière, cette phase est profondément marquée par le caractère inquisitoire et contradictoire de la procédure, et recourt très fréquemment à l'expertise technique pour établir la vérité matérielle.
	
	\subsection{Le caractère écrit et contradictoire de l'instruction (Arts. 382, 385 et 386 du LPF)}
	
	La procédure devant le juge de l'impôt est essentiellement écrite. Le magistrat forge sa conviction sur la base des mémoires et des preuves documentaires échangés entre les parties.
	
	\begin{enumerate}
		\item \textbf{L'échange des mémoires :} \\
		La requête introductive d'instance est notifiée d'office par le greffe au Commissaire Général. L'Administration fiscale, qui peut être représentée par ses propres agents (les inspecteurs) ou par un avocat mandaté (Art. 382 du LPF), doit formuler ses arguments dans un « mémoire en défense ». Toutes les pièces produites par une partie sont obligatoirement communiquées à la partie adverse pour préserver l'égalité des armes.
		
		\item \textbf{La sanction de l'inertie (Le couperet du délai) :} \\
		L'article 386 du LPF institue une discipline procédurale redoutable. L'Administration fiscale dispose d'un délai strict de deux (02) mois pour produire ses observations. En cas de retard, le Président de la Cour peut lui accorder un ultime délai d'un mois.
		\begin{itemize}
			\item \textit{La présomption d'acquiescement :} Si l'Administration laisse expirer ces délais sans déposer son mémoire, elle est « réputée avoir acquiescé aux faits » exposés par le contribuable dans sa requête. C'est une sanction très lourde : le juge considérera les allégations factuelles du requérant comme établies.
			\item \textit{Le désistement du contribuable :} À l'inverse, si le juge impose un délai de réponse au contribuable et que celui-ci ne le respecte pas, il est juridiquement réputé s'être désisté de son action.
		\end{itemize}
	\end{enumerate}
	
	\subsection{L'expertise judiciaire foncière : L'éclairage technique du juge (Arts. 389 à 394 du LPF)}
	
	Le juge administratif est un technicien du droit, mais pas nécessairement un expert en évaluation immobilière. Face à un litige portant sur la consistance physique d'un immeuble (mesure de la surface, état d'achèvement d'un chantier, détermination de la catégorie de standing), le recours à l'expertise devient souvent incontournable.
	
	\begin{enumerate}
		\item \textbf{La nomination et l'exigence d'indépendance :} \\
		L'expertise est en principe confiée à un expert unique désigné par la Cour (ou à un collège de trois experts si l'une des parties le requiert). Pour garantir une impartialité absolue du rapport, l'article 391 du LPF pose une incompatibilité stricte : il est formellement interdit de désigner comme expert un fonctionnaire de l'Administration fiscale ou le mandataire du contribuable.
		
		\item \textbf{Le caractère contradictoire des opérations sur le terrain :} \\
		L'expert ne réalise pas ses constatations en secret. L'article 393 du LPF impose que le transport sur les lieux (la visite de l'immeuble litigieux) s'effectue en présence conjointe de l'agent de l'Administration et du contribuable (ou de son représentant). Cette visite contradictoire permet à chacun de formuler des remarques sur place.
		
		\item \textbf{La portée du rapport d'expertise :} \\
		Les constatations et l'avis technique de l'expert sont consignés dans un rapport notifié aux parties, qui peuvent y répondre par de nouveaux mémoires. Il est fondamental de rappeler que l'expert éclaire la religion du juge, mais ne le lie pas : le juge reste souverain et peut écarter les conclusions de l'expert s'il les estime infondées.
	\end{enumerate}
	
	\subsection{Le droit à l'information et la méthode par comparaison (Arts. 398 et 399 du LPF)}
	
	Dans le contentieux des évaluations foncières, le Cadastre utilise fréquemment la méthode d'évaluation par comparaison (déterminer la valeur d'une villa en se référant à la valeur retenue pour des villas similaires dans le même secteur).
	
	\begin{itemize}
		\item \textbf{La transparence face au secret professionnel :} \\
		L'article 399 du LPF garantit les droits de la défense du citoyen. Le contribuable a le droit absolu de prendre connaissance des documents que l'Administration a joints au dossier pour justifier son évaluation, y compris ceux contenant des indications relatives aux biens de tiers (les termes de comparaison). Le but est de permettre au requérant de vérifier si les points de comparaison choisis par l'Administration concernent des propriétés réellement similaires à la sienne.
		
		\item \textbf{La relativité de la force probante :} \\
		L'article 399 \textit{in fine} pose une limite importante à la technique administrative : les comparaisons établies par l'Administration ne constituent pas, à elles seules, des preuves irréfragables. Elles constituent un faisceau d'indices que le juge appréciera librement.
	\end{itemize}
	
	
\end{document}