
\documentclass[Cours_contentieuxTF2026.tex]{subfiles}

\begin{document}

\chapter{La procédure contentieuse (contentieux de droit)}

La procédure contentieuse constitue le cœur du contentieux fiscal. Elle organise,
dans un cadre juridique strict, les modalités selon lesquelles le contribuable peut
contester une imposition mise à sa charge et obtenir, le cas échéant, la réparation
d’une erreur commise par l’administration fiscale. En matière de taxes foncières,
cette procédure revêt une importance particulière en raison de la nature technique
de l’assiette et du rôle déterminant du cadastre dans la formation de l’impôt.

Le contentieux fiscal togolais est structuré autour du principe du préalable
administratif obligatoire, qui confère à l’administration fiscale un rôle central dans
la résolution des litiges. La phase juridictionnelle n’intervient qu’en cas d’échec de la
procédure administrative, laquelle demeure la voie normale et privilégiée de
règlement des différends fiscaux.

\section{Le principe du "Préalable Administratif Obligatoire" (RAPO)}

Le contentieux fiscal se caractérise par une architecture procédurale originale qui déroge au droit commun de la responsabilité administrative. La pierre angulaire de cet édifice est le principe du \textbf{Recours Administratif Préalable Obligatoire (RAPO)}. Cette règle, loin d'être une simple formalité bureaucratique, constitue une garantie substantielle tant pour l'Administration que pour le contribuable.

\subsection{Fondement juridique : Le privilège du préalable}
En droit fiscal togolais, comme dans la plupart des systèmes juridiques d'inspiration francophone, le contribuable ne peut jamais saisir directement le juge de l'impôt.

\begin{itemize}
	\item \textbf{Le texte de référence :} Le Livre des Procédures Fiscales (LPF), notamment par l’article 367, pose le principe selon lequel toute contestation relative à l'assiette ou au recouvrement des impôts, taxes, droits et redevances doit être \textit{« adressée à l'autorité administrative compétente »} avant toute instance juridictionnelle.
	
	\item \textbf{La nature juridique :} Il s'agit d'une \textbf{fin de non-recevoir d'ordre public}. Cela signifie que si un contribuable assigne directement l'OTR devant le Tribunal sans avoir déposé de réclamation préalable, le juge a l'obligation de rejeter la demande sans même l'examiner au fond (irrecevabilité \textit{in limine litis}), et ce, même si l'Administration ne soulève pas cette exception.
\end{itemize}

\subsection{Justification téléologique : La fonction de "filtre"}
La doctrine administrative justifie ce verrouillage de l'accès au juge par trois impératifs majeurs, particulièrement pertinents en matière de fiscalité foncière :

\begin{enumerate}
	\item \textbf{Le pouvoir de rectification de l'Administration (Le droit à l'erreur) :} \\
	L'administration fiscale gère des milliers de cotes foncières basées sur des données cadastrales complexes. L'erreur matérielle est inévitable (erreur de saisie, homonymie, confusion de parcelle). Le préalable administratif offre à l'Administration la possibilité de corriger elle-même ses propres erreurs, rapidement et sans formalisme excessif. C'est une phase d'\textit{audit interne provoqué}.
	
	\item \textbf{Le désengorgement des prétoires (Économie de procédure) :} \\
	Les statistiques démontrent que l'immense majorité des litiges fiscaux (plus de 90 \% dans les systèmes comparables) se résolvent au stade de la réclamation préalable. Sans ce filtre, les tribunaux administratifs seraient submergés par des contestations purement techniques (ex: métrage d'une terrasse) qui ne nécessitent pas l'intervention d'un magistrat.
	
	\item \textbf{La cristallisation du litige :} \\
	La réclamation préalable permet de figer le cadre du débat. Les moyens (arguments) que le contribuable pourra soulever devant le juge seront limités par la nature de ceux exposés dans sa réclamation initiale. Cela évite les demandes nouvelles imprévisibles devant le tribunal, garantissant ainsi la loyauté des débats (principe de l'\textit{immutabilité de l'instance}).
\end{enumerate}

\subsection{Spécificité du contentieux cadastral}
Pour le cadre du Cadastre, ce principe revêt une importance particulière. Contrairement à d'autres impôts (TVA, IS) qui reposent sur la comptabilité et l'interprétation du droit, la taxe foncière est un \textbf{impôt indiciaire}. Le litige porte souvent sur des faits matériels (la maison est-elle finie ? quelle est la surface réelle ?).

\begin{itemize}[label=\tiny$\blacksquare$]
	\item \textbf{Conséquence pratique :} Le préalable administratif est le moment privilégié pour l'expertise technique (visite des lieux, mesurage contradictoire). Il permet de confronter la réalité du terrain aux données du Système d'Information Foncier (SIF) avant toute judiciarisation. C'est à ce stade que l'expert géomètre de l'OTR joue son rôle de \textit{« juge technique »} de première instance.
\end{itemize}


\section{La réclamation contentieuse préalable}

L'acte introductif d'instance dans la phase administrative est la réclamation. Bien plus qu'une simple lettre de doléances, la réclamation contentieuse est un \textbf{acte juridique formel} qui obéit à un régime strict de recevabilité. Sa validité est conditionnée par le respect cumulatif de règles de délai, de forme et de capacité.

\subsection{Les conditions de délai : La forclusion}
En droit fiscal, les délais de recours sont des délais préfix, c'est-à-dire des délais de rigueur dont l'expiration entraîne l'extinction du droit d'agir (la forclusion). Cette déchéance est d'ordre public et peut être soulevée d'office par l'Administration ou le Juge.

\begin{itemize}
	\item \textbf{Le délai général (R+1) :} \\
	En matière d'impôts directs locaux (TFPB, TFPNB), le délai de réclamation expire le \textbf{31 décembre de l'année suivant} celle de la mise en recouvrement du rôle.
	\textit{Exemple : Pour un avis d'imposition reçu en mars 2025 (Rôle 2025), la réclamation est recevable jusqu'au 31 décembre 2026.}
	
	\item \textbf{Le délai spécial (L'événement motivant la réclamation) :} \\
	Lorsque le contribuable n'a pas reçu d'avis initial ou lorsqu'un événement postérieur vient modifier la situation juridique du bien (ex : une décision de justice reconnaissant qu'un terrain appartient à autrui), le délai court à compter de la date de cet événement.
	
	\item \textbf{La date faisant foi :} \\
	C'est la date de réception par l'Administration qui fait foi (cachet du service courrier de l'OTR) et non la date de rédaction.
\end{itemize}

\subsection{Les conditions de forme : Le formalisme ad validitatem}
Le contentieux fiscal n'admet pas de réclamation orale. Le formalisme demeure une condition de validité substantielle.

\begin{enumerate}
	\item \textbf{L'exigence de l'écrit signé :} \\
	La réclamation doit être manuscrite ou dactylographiée. L'élément crucial est la \textbf{signature manuscrite} de l'auteur. Une réclamation non signée est considérée comme inexistante juridiquement, bien qu'elle soit régularisable sur demande de l'Administration.
	
	\begin{itemize}[label=\tiny$\blacksquare$]
		\item \textbf{Précision :} La jurisprudence administrative considère que le défaut de signature est une irrégularité régularisable. L'administration doit inviter le contribuable à signer avant de rejeter.
	\end{itemize}
	
	\item \textbf{L'identification de l'imposition :} \\
	Le réclamant doit mentionner explicitement l'impôt contesté (Nature, Année d'imposition) et joindre obligatoirement l'\textbf{Avis de Mise en Recouvrement (AMR)} ou une copie de l'avis d'imposition. Cette pièce est capitale pour permettre au service de lier le contentieux à une cote précise.
	
	\item \textbf{L'exposé sommaire des moyens :} \\
	Le contribuable ne peut se contenter de dire « Je conteste ». Il doit articuler des moyens de fait ou de droit :
	\begin{itemize}
	\item Les moyens de fait : (ex : « La superficie bâtie n'est pas de 200m² mais de 150m² »).
	
	\item  Les moyens de droit : (ex : « Violation de l'article X du CGI sur l'exonération des bâtiments agricoles »).
\end{itemize}

\begin{itemize}[label=\tiny$\blacksquare$]
	
	\item  Note : En matière foncière, les moyens sont majoritairement factuels, nécessitant souvent l'intervention de l'expert géomètre.
	
\end{itemize}
	
\end{enumerate}

\subsection{La qualité et l'intérêt à agir}

La recevabilité de la réclamation est subordonnée à la légitimité de son auteur (Règle : \textit{Nul ne plaide par procureur}).

\begin{enumerate}

\item  \textbf{Le redevable légal :}\\
Seule la personne au nom de laquelle l'impôt est établi (inscrite au rôle) a qualité pour réclamer.

\begin{itemize}[label=\tiny$\blacksquare$]
\item \textbf{ Problématique de l'indivision :} En cas d'indivision successorale, un seul co-indivisaire peut valablement présenter une réclamation pour le compte de l'ensemble de l'indivision (mandat tacite de gestion).
\end{itemize}

\item  \textbf{Le mandat de représentation :} \\
Si le contribuable ne souhaite pas agir personnellement, il peut se faire représenter par un mandataire (avocat, expert-comptable, géomètre-expert, ou un tiers).

\begin{itemize}[label=\tiny$\blacksquare$]
\item  \textbf{Mandat spécial :} Pour les tiers non-avocats, la production d'un mandat spécial rédigé et signé par le contribuable est obligatoire à peine d'irrecevabilité. Pour les avocats, le mandat est présumé (ad litem).
\end{itemize}

\end{enumerate}

\subsection{Les effets juridiques de la réclamation}
Le dépôt d'une réclamation régulière produit deux effets majeurs en droit:

\begin{itemize}[label=\tiny$\blacksquare$]
	\item \textbf{L'effet interruptif de prescription :} La réclamation interrompt la prescription quadriennale. Elle ouvre une phase d'instruction administrative durant laquelle les délais de recouvrement sont suspendus.
	
	\item \textbf{L'absence d'effet suspensif de paiement (Principe du "Payer d'abord, discuter ensuite"): } 
   C'est une règle cardinale du droit fiscal : \textbf{la réclamation ne suspend pas l'exigibilité de l'impôt}. Le comptable public est en droit de poursuivre le recouvrement forcé (saisie) même si une réclamation est en cours d'instruction, sauf si le contribuable a expressément formulé une demande de\textbf{ sursis de paiement assortie de garanties} (Voir Section 6).
\end{itemize}


\section{L’instruction de la réclamation en la forme}
L'instruction de la réclamation constitue la phase centrale du contentieux administratif. Elle ne saurait être réduite à une simple vérification bureaucratique. Juridiquement, il s'agit d'une procédure administrative interne, \textbf{inquisitoire et contradictoire}, durant laquelle l'Administration a l'obligation d'examiner les griefs du contribuable.

Cette phase se décompose en deux temps procéduraux distincts : le contrôle de recevabilité (la forme) et l'examen du bien-fondé (le fond).

\subsection{Le contrôle liminaire de recevabilité (L'instruction en la forme)}

L'instruction d'une réclamation contentieuse commence systématiquement par une phase de « filtrage juridique ». Avant même d'examiner si le contribuable a raison sur le fond (ex: erreur de superficie), l'Administration doit vérifier s'il a le \textbf{droit d'agir}.

Ce contrôle de recevabilité est une étape d'ordre public : l'Administration a l'obligation de la mener, et le juge administratif la vérifiera d'office en cas de recours ultérieur.

\subsubsection{Les points de contrôle (La « Check-list » de recevabilité)}

L'agent chargé de l'instruction (souvent à la Division du Contentieux ou au Service Cadastre pour pré-instruction) doit passer la demande au crible de cinq critères cumulatifs. Le défaut d'un seul de ces critères peut entraîner le rejet.

\begin{enumerate}
	\item \textbf{Le contrôle du délai (La temporalité) :} \\
	C'est le premier point à vérifier. Le délai de réclamation (généralement jusqu'au 31 décembre de l'année N+1 suivant la mise en recouvrement) est un \textbf{délai préfix}.
	\begin{itemize}
		\item \textit{Règle :} Une réclamation déposée hors délai est frappée de \textbf{forclusion}. Aucune excuse (ignorance de la loi, absence du pays, maladie simple) ne permet de relever cette forclusion, sauf cas de force majeure avérée et très strictement interprétée.
		\item \textit{Pratique :} L'agent vérifie la date du cachet de la poste ou celle de l'accusé de réception du service courrier de l'OTR.
	\end{itemize}
	
	\item \textbf{Le contrôle de la qualité (La légitimité) :}
	\begin{itemize}
		\item \textit{Identité :} Le réclamant est-il bien celui qui est inscrit au rôle ? (Attention aux homonymies).
		\item \textit{Capacité :} Si le réclamant est une société, le signataire a-t-il le pouvoir de l'engager (Gérant, DG) ?
		\item \textit{Représentation :} Si la réclamation est signée par un tiers (géomètre privé, expert-comptable, neveu...), un \textbf{mandat régulier} est-il joint ? Seuls les avocats sont dispensés de présenter ce mandat.
	\end{itemize}
	
	\item \textbf{Le contrôle de la forme (L'écrit et la signature) :} \\
	Le contentieux fiscal est une procédure écrite.
	\begin{itemize}
		\item Une demande par téléphone ou une simple visite au bureau ne constitue pas une réclamation valable.
		\item \textbf{La signature manuscrite} est une condition substantielle. Une lettre dactylographiée non signée n'a pas d'existence juridique.
	\end{itemize}
	
	\item \textbf{Le contrôle de la motivation (L'exposé des moyens) :} \\
	La réclamation doit être \textbf{motivée}. Une lettre qui se bornerait à dire « \textit{Je refuse de payer cet impôt trop cher} » sans avancer d'argument factuel (erreur de base) ou juridique (mauvaise application de la loi) est irrecevable pour défaut de motivation.
	\begin{itemize}
		\item \textit{En matière foncière :} Le contribuable doit préciser ce qu'il conteste (la surface, le classement, l'affectation).
	\end{itemize}
	
	\item \textbf{La production de l'acte contesté (L'identification) :} \\
	La demande doit permettre d'identifier l'imposition. L'usage impose de joindre l'\textbf{Avis de Mise en Recouvrement (AMR)} ou l'avis d'imposition foncière. L'absence de référence à une cote précise rend la réclamation inopérante.
\end{enumerate}

\subsubsection{La gestion des irrégularités : Distinction fondamentale}

Face à une réclamation incomplète ou mal formulée, l'attitude de l'Administration diffère selon la nature du vice. C'est une distinction cruciale pour la protection des droits du contribuable.

\begin{description}
	\item[Les vices non régularisables (Rejet immédiat)] \hfill \\
	Certaines irrégularités sont définitives et ne peuvent être corrigées après coup.
	\begin{itemize}
		\item \textit{Cas type :} \textbf{La tardiveté}. Si le délai est expiré au moment du dépôt, l'Administration constate la forclusion et rejette la demande \textit{in limine litis} (dès le seuil). Aucune régularisation n'est possible.
	\end{itemize}
	
	\item[Les vices régularisables (Invitation à corriger)] \hfill \\
	Pour les autres vices de forme, l'Administration a un \textbf{devoir de loyauté}. Elle ne doit pas rejeter la demande immédiatement (ce qui serait un piège), mais inviter le contribuable à la régulariser.
	\begin{itemize}
		\item \textit{Exemples :} Oubli de signature, absence du mandat, oubli de la pièce jointe (avis d'imposition).
		\item \textit{Procédure :} Le service envoie une \textbf{Demande de régularisation} en accordant un délai (ex: 30 jours) au contribuable pour compléter son dossier.
		\item \textit{Sanction :} Ce n'est qu'à l'expiration de ce délai supplémentaire, si le contribuable reste silencieux, que le rejet pour vice de forme est prononcé.
	\end{itemize}
\end{description}



\subsubsection{Conséquence pratique pour le service du Cadastre}

Bien que l'instruction juridique relève souvent du service contentieux, le cadre du Cadastre est souvent le premier point de contact. Il doit être capable de réaliser un \textbf{pré-contrôle} :

\begin{itemize}[label=\tiny$\blacksquare$]
	\item Si un usager apporte une réclamation non signée, l'agent du Cadastre doit lui faire signer immédiatement au guichet.
	\item Si un usager conteste une taxe vieille de 5 ans, l'agent doit l'informer (avec pédagogie) du risque élevé de rejet pour forclusion, tout en acceptant le dépôt (car seul le chef de service a compétence pour signer le rejet).
\end{itemize}

\subsection{L'instruction au fond : L'expertise technique cadastrale}

Une fois la réclamation déclarée recevable en la forme, s'ouvre la phase d'instruction au fond. En matière de fiscalité foncière, contrairement à la fiscalité d'entreprise qui est comptable, l'instruction est essentiellement \textbf{physique et matérielle}. Elle consiste à confronter les énonciations de la matrice cadastrale à la réalité du terrain.

\subsubsection{La répartition de la charge de la preuve}

Dans le contentieux de l'assiette, la règle de principe est celle du droit commun : \textit{Actori incumbit probatio} (la charge de la preuve incombe au demandeur).

\begin{enumerate}
	\item \textbf{La présomption de régularité du rôle :} \\
	L'imposition émise par l'Administration est présumée exacte. C'est au contribuable qui la conteste d'apporter la preuve de son caractère erroné.
	
	\item \textbf{Le rôle actif de l'Administration (L'instruction inquisitoire) :} \\
	Cependant, cette charge de la preuve est aménagée. Le contribuable n'a souvent pas accès aux éléments techniques de calcul (fiches d'évaluation). Dès lors, l'Administration a le devoir de vérifier ses propres bases. Si le contribuable apporte un « commencement de preuve » (ex : un plan de géomètre privé contradictoire), l'Administration est tenue de procéder à une contre-expertise.
\end{enumerate}

\subsubsection{La vérification sur pièces (Le contrôle de bureau)}

Avant tout déplacement, le service du Cadastre saisi pour avis technique procède à un audit documentaire dans le Système d'Information Foncier (SIF) :

\begin{itemize}
	\item \textbf{Analyse de la Fiche d'Évaluation :} L'agent vérifie si les éléments déclarés ou recensés correspondent à la catégorie taxée.
	\textit{Exemple d'erreur détectable au bureau : Une parcelle taxée comme « Bâtie » alors que la mise à jour des images satellitaires montre un terrain vide.}
	
	\item \textbf{Vérification des calculs :} Contrôle de l'application correcte des tarifs au m² selon la zone (Zone administrative, résidentielle, périphérie) et vérification arithmétique de la surface pondérée.
\end{itemize}

\subsubsection{L'expertise terrain : Le constat contradictoire}

C'est l'acte central de l'instruction foncière. Lorsque le désaccord persiste sur la consistance du bien, une descente sur les lieux est indispensable.

\begin{enumerate}
	\item \textbf{Le fondement juridique (Droit de visite) :} \\
	Les agents assermentés du Cadastre disposent d'un droit d'entrée dans les propriétés pour les besoins de l'évaluation fiscale (sous réserve des heures légales).
	
	\item \textbf{La méthodologie du « constat contradictoire » :} \\
	Pour que le rapport d'expertise soit inattaquable devant le juge, l'opération doit être contradictoire.
	\begin{itemize}
		\item \textbf{Convocation :} Le contribuable est invité à être présent ou représenté.
		\item \textbf{Le mesurage :} Prise de mesures des surfaces hors œuvre et habitables.
		\item \textbf{Le recensement des éléments de confort (VLC) :} L'agent note la présence et l'état de fonctionnement des équipements (piscine, climatisation, ascenseur).
		\item \textit{Point critique :} L'appréciation de la \textbf{vétusté}. L'agent doit évaluer si le coefficient de vétusté appliqué correspond à la dégradation réelle.
	\end{itemize}
	
	\item \textbf{Le Rapport d'Expertise Cadastrale :} \\
	À l'issue de la visite, l'agent rédige un rapport technique qui lie le service du contentieux sur les faits. Ce rapport doit contenir : la description précise de l'immeuble, un croquis coté, des photographies datées, et une conclusion technique claire.
\end{enumerate}

\subsubsection{Les paramètres techniques spécifiques au litige}

L'instruction se concentre généralement sur trois variables d'ajustement de la Valeur Locative Cadastrale (VLC) :

\begin{description}
	\item[La classification (Le Standing)] \hfill \\
	Le litige porte sur le classement (Luxe, Moyen, Économique). L'instructeur doit objectiver le classement en utilisant la grille de cotation officielle (matériaux, finitions).
	
	\item[L'affectation (L'Usage)] \hfill \\
	Vérifier si l'usage réel correspond à l'usage déclaré (ex : villa d'habitation transformée en bureaux).
	
	\item[La pondération des surfaces] \hfill \\
	Vérifier si les coefficients sont justifiés (ex : une terrasse couverte ne doit pas être comptée à 100 \% comme une pièce de vie).
\end{description}

\begin{center}
	\fbox{\begin{minipage}{0.9\textwidth}
			\textbf{Note pédagogique :} Insister sur l'importance des \textbf{photos} dans le dossier d'instruction. Devant la Commission de Recours ou le Juge, une photo vaut mieux qu'une longue description pour prouver la vétusté ou l'inachèvement d'un immeuble.
	\end{minipage}}
\end{center}



\section{La décision administrative}

À l’issue de l’instruction, l’administration fiscale statue sur la réclamation par une
décision administrative.

\subsection{Admission totale}

L’admission totale intervient lorsque l’administration reconnaît le bien-fondé
complet de la réclamation. Elle entraîne le dégrèvement intégral de l’imposition
contestée.

\subsection{Admission partielle}

L’admission partielle correspond à la situation dans laquelle l’administration
reconnaît partiellement le bien-fondé de la réclamation. Elle conduit à un dégrèvement
partiel de la taxe foncière.

\subsection{Rejet}

Le rejet intervient lorsque l’administration estime que la réclamation n’est pas
fondée. Cette décision ouvre la voie à un éventuel recours juridictionnel.

\subsection{Obligation de motivation}

Toute décision administrative rendue en matière contentieuse doit être motivée en
droit et en fait. L’obligation de motivation constitue une garantie essentielle pour le
contribuable et une condition de la légalité de la décision.

\section{Le sursis de paiement}

\subsection{Principe}

Le sursis de paiement est une mesure par laquelle l’administration suspend
temporairement le recouvrement de l’imposition contestée pendant l’instruction de
la réclamation ou la phase juridictionnelle.

\subsection{Conditions}

Le sursis de paiement est subordonné à la présentation d’une réclamation régulière et
à une demande expresse du contribuable. Il peut être refusé lorsque la contestation
apparaît manifestement infondée.

\subsection{Garanties exigées}

Afin de préserver les intérêts du Trésor public, l’octroi du sursis de paiement peut
être subordonné à la constitution de garanties suffisantes, telles qu’une caution ou
une sûreté réelle, dans les conditions prévues par le Livre des procédures fiscales.

\section*{Conclusion du chapitre}
\addcontentsline{toc}{section}{Conclusion du chapitre}

La procédure contentieuse en matière de taxes foncières illustre l’équilibre recherché
par le législateur entre la protection des droits du contribuable et la sauvegarde des
intérêts financiers des collectivités territoriales. La maîtrise de cette procédure est
indispensable aux cadres du cadastre et de l’administration fiscale, appelés à jouer
un rôle central dans la prévention et le règlement des litiges fonciers.


\end{document}