
\documentclass[Cours_contentieuxTF2026.tex]{subfiles}

\begin{document}
\chapter{La typologie du contentieux des taxes foncières}

Le contentieux des taxes foncières ne constitue pas un ensemble uniforme de litiges.
Il recouvre une pluralité de contestations dont la nature, l’objet et le régime
juridique varient selon la phase du processus fiscal concernée. La typologie du
contentieux foncier permet ainsi de classer les litiges en fonction de leur finalité,
des règles procédurales applicables et des autorités compétentes pour en connaître.

Cette distinction est fondamentale tant pour le contribuable que pour
l’administration fiscale. Elle conditionne la recevabilité des réclamations, la
procédure à suivre, les délais applicables, les effets juridiques des décisions rendues
et, le cas échéant, la compétence du juge saisi. En matière de taxes foncières, cette
typologie revêt en outre une dimension technique particulière, liée au rôle central
du cadastre dans la détermination de l’assiette de l’impôt.

Conformément au Livre des procédures fiscales (LPF), le contentieux des taxes
foncières peut être regroupé autour de trois grandes formes : le contentieux
juridictionnel (ou contentieux de droit), les recours gracieux et les dégrèvements
d’office. À ces catégories s’ajoutent des sous-contentieux spécifiques, notamment le
contentieux du recouvrement et le contentieux technique cadastral.

\section{Le contentieux de l’assiette des taxes foncières}

Le contentieux de l’assiette constitue la forme la plus fréquente et la plus structurante
du contentieux des taxes foncières. Il porte sur la légalité et le bien-fondé de
l’imposition, c’est-à-dire sur les éléments servant de base à la liquidation de la taxe.

\subsection{Notion et fondement juridique}

Le contentieux de l’assiette vise à contester l’existence, le montant ou les modalités
de calcul de la taxe foncière mise à la charge du contribuable. Il relève du contentieux
de droit au sens strict et s’inscrit dans le cadre du préalable administratif obligatoire
prévu à l’article 367 du Livre des procédures fiscales.

Le contribuable soutient, dans ce cadre, que l’administration a méconnu les
dispositions du Code général des impôts lors de l’établissement de l’imposition.
La réclamation peut porter notamment sur :
\begin{itemize}
	\item l’inexactitude de la base imposable ;
	\item l’erreur dans l’évaluation de la valeur locative cadastrale ;
	\item la mauvaise qualification du bien (propriété bâtie ou non bâtie) ;
	\item l’identification erronée du redevable légal ;
	\item le refus injustifié d’une exonération prévue par la loi fiscale.
\end{itemize}

Lorsque la réclamation n’aboutit pas à une solution satisfaisante et qu’il n’est pas fait
recours à l’arbitrage de la Commission administrative des recours prévue à l’article
356 du LPF, le contribuable peut saisir la juridiction compétente.

\subsection{Le rôle déterminant du cadastre dans le contentieux de l’assiette}

Dans le domaine des taxes foncières, le contentieux de l’assiette est étroitement lié
aux données cadastrales. Les erreurs susceptibles d’affecter l’assiette de l’impôt
concernent le plus souvent :
\begin{itemize}
	\item la superficie réelle de l’immeuble ;
	\item son usage effectif ;
	\item sa catégorie cadastrale ;
	\item son état d’achèvement ;
	\item les droits réels exercés sur le bien.
\end{itemize}

Les services du cadastre, intégrés à la Direction du Cadastre et de la Conservation
Foncière de l’Office Togolais des Recettes, interviennent comme experts techniques
dans l’instruction des réclamations. Leur contribution est essentielle à la
vérification des faits allégués et à la motivation des décisions de dégrèvement ou de
rejet.

\section{Le contentieux du recouvrement des taxes foncières}

Le contentieux du recouvrement se distingue du contentieux de l’assiette par son
objet. Il ne remet pas en cause le bien-fondé de l’imposition, mais porte sur les
conditions dans lesquelles l’administration exige le paiement de la taxe.

\subsection{Champ du contentieux du recouvrement}

Le contentieux du recouvrement concerne notamment :
\begin{itemize}
	\item l’obligation de payer la dette fiscale ;
	\item la régularité des actes de poursuite ;
	\item l’extinction de la dette par paiement, compensation ou prescription ;
	\item la prise en compte d’un sursis de paiement régulièrement accordé.
\end{itemize}

Il relève principalement de la compétence du comptable public et peut être porté
devant le juge compétent dans les conditions prévues par le LPF.

\subsection{Autonomie du contentieux du recouvrement}

Le contentieux du recouvrement est juridiquement autonome par rapport au
contentieux de l’assiette. Une imposition peut être régulière dans son principe tout en
donnant lieu à un contentieux du recouvrement en raison d’irrégularités dans les
poursuites engagées.

Cette autonomie impose une coordination étroite entre les services chargés de
l’assiette, du contentieux et du recouvrement, afin d’éviter des mesures de poursuite
contraires aux droits du contribuable, notamment en présence d’une réclamation
assortie d’une demande de sursis de paiement.

\section{Les recours gracieux en matière de taxes foncières}

Les recours gracieux relèvent du contentieux fiscal au sens large. Ils se distinguent du
contentieux de droit en ce que le contribuable ne conteste pas formellement la
légalité de l’imposition, mais sollicite une mesure de bienveillance de
l’administration.

\subsection{Fondement juridique des recours gracieux}

Le Livre des procédures fiscales prévoit plusieurs mesures gracieuses, notamment :
\begin{itemize}
	\item les remises gracieuses de pénalités et majorations prévues à l’article 419 du
	LPF ;
	\item les transactions fiscales prévues à l’article 421 du LPF ;
	\item les modérations ou remises conditionnelles ;
	\item la décharge de responsabilité prévue à l’article 540 du LPF.
\end{itemize}

Ces mesures relèvent du pouvoir discrétionnaire de l’administration fiscale et ne
constituent pas un droit pour le contribuable.

\subsection{Application aux taxes foncières}

En matière de taxes foncières, les recours gracieux sont fréquemment fondés sur la
situation économique du redevable, l’indigence, ou des circonstances
exceptionnelles affectant l’immeuble. Ils participent à la conciliation entre les droits
du contribuable et les intérêts financiers des collectivités territoriales.

\section{Les dégrèvements d’office et admissions en non-valeur}

Les dégrèvements d’office constituent une forme particulière de contentieux fiscal,
prévue à l’article 536 du Livre des procédures fiscales.

\subsection{Conditions et portée des dégrèvements d’office}

Lorsqu’une erreur d’imposition est constatée au préjudice du contribuable, le
dégrèvement ou la restitution peut être prononcé d’office, sans que l’intéressé soit
tenu de présenter une réclamation régulière, et ce même lorsque les délais de
réclamation sont expirés.

Ces dégrèvements peuvent concerner des erreurs imputables aussi bien à
l’administration qu’au contribuable, notamment lorsque l’imposition a été établie
sur la base de déclarations ultérieurement reconnues inexactes.

\subsection{Intérêt particulier en matière foncière}

En matière de taxes foncières, les dégrèvements d’office jouent un rôle essentiel,
compte tenu de la fréquence des erreurs liées à l’actualisation des données
cadastrales. Ils traduisent le souci d’équité et de bonne administration fiscale et
contribuent à la fiabilité du système d’imposition foncière.

\section*{Conclusion du chapitre}
\addcontentsline{toc}{section}{Conclusion du chapitre}

La typologie du contentieux des taxes foncières met en évidence la diversité des
litiges susceptibles de naître de l’établissement et du recouvrement de cet impôt
local. Elle révèle le caractère largement administratif et technique du contentieux
foncier, dans lequel les dispositions du Livre des procédures fiscales et le rôle du
cadastre occupent une place centrale.

La maîtrise de cette typologie constitue un préalable indispensable à l’étude des
procédures contentieuses proprement dites, qui feront l’objet des développements
consacrés à la réclamation préalable, aux procédures gracieuses et aux voies de
recours juridictionnelles.


\end{document}