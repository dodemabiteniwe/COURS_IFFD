\subsection{Les formes de contentieux fiscal applicables aux taxes foncières}

Le contentieux des taxes foncières s’inscrit dans le cadre général du contentieux
fiscal, lequel constitue une branche particulière du contentieux administratif.
Il regroupe l’ensemble des litiges nés de l’action de l’administration fiscale dans
l’exercice de ses missions d’assiette, de contrôle et de recouvrement de l’impôt.

En droit fiscal togolais, et conformément aux dispositions du Livre des procédures
fiscales (LPF), il est possible de distinguer, en fonction de la nature de la
contestation, trois grandes formes de contentieux fiscal, applicables aux taxes
foncières : la juridiction contentieuse, les recours gracieux et les dégrèvements
d’office.

\paragraph{1. La juridiction contentieuse (contentieux au sens strict)}

La juridiction contentieuse constitue le contentieux fiscal au sens strict. Elle a pour
objet la réparation d’une erreur de droit ou de fait commise par l’administration
fiscale dans l’établissement ou le recouvrement de l’impôt, en vue de la reconnaissance
d’un droit au profit du contribuable.

Cette forme de contentieux est caractérisée par l’existence d’une phase
administrative préalable obligatoire. En vertu de l’article 367 du Livre des procédures
fiscales, toute contestation portant sur une imposition doit, à peine
d’irrecevabilité, être précédée d’une réclamation adressée à l’administration fiscale
compétente.

Lorsque le contribuable ne trouve pas satisfaction à l’issue de cette phase
administrative, et qu’il n’est pas fait recours à l’arbitrage d’une commission
administrative, notamment la Commission administrative des recours prévue à
l’article 356 du LPF, le litige peut être porté devant la juridiction compétente.

Dans le domaine des taxes foncières, la juridiction contentieuse concerne
principalement :
\begin{itemize}
	\item les contestations relatives à l’assiette de l’impôt (valeur locative,
	consistance de l’immeuble, identification du redevable) ;
	\item les contestations portant sur la liquidation de la taxe ;
	\item certaines contestations relatives au recouvrement, lorsque la régularité
	des poursuites est mise en cause.
\end{itemize}

\paragraph{2. Les recours gracieux ou contentieux au sens large}

Les recours gracieux relèvent d’une logique fondamentalement différente de celle du
contentieux de droit. Dans ce cadre, le contribuable ne conteste pas formellement le
bien-fondé de l’imposition mise à sa charge, mais sollicite une mesure de bienveillance
de l’administration fiscale, en raison de difficultés financières ou de circonstances
exceptionnelles.

Le recours gracieux repose ainsi sur des considérations d’équité et non sur une
appréciation de la légalité de l’impôt. Il s’agit d’une faculté offerte au contribuable,
sans que celui-ci ne puisse se prévaloir d’un droit à l’obtention de la mesure
sollicitée.

Le Livre des procédures fiscales prévoit plusieurs formes de mesures gracieuses,
notamment :
\begin{itemize}
	\item les remises gracieuses de majorations et de pénalités prévues à l’article 419
	du LPF ;
	\item les transactions fiscales prévues à l’article 421 du LPF ;
	\item les modérations ou remises conditionnelles ;
	\item la décharge de responsabilité prévue à l’article 540 du LPF.
\end{itemize}

Ces mesures ne peuvent être accordées qu’à la demande expresse du contribuable ou
de son représentant légal et relèvent du pouvoir discrétionnaire de l’administration
fiscale. En matière de taxes foncières, elles sont fréquemment sollicitées en cas
d’indigence, de sinistre affectant l’immeuble ou de difficultés économiques
durables.

\paragraph{3. Les dégrèvements d’office ou admissions en non-valeur}

Les dégrèvements d’office constituent une troisième forme de contentieux fiscal,
distincte tant de la juridiction contentieuse que des recours gracieux. Ils trouvent leur
fondement dans l’article 536 du Livre des procédures fiscales.

Le dégrèvement ou la restitution d’office peut être prononcé par l’administration
fiscale lorsqu’une erreur d’imposition est constatée au préjudice du contribuable,
sans que celui-ci soit tenu de présenter une réclamation régulière, et ce, même lorsque
les délais de réclamation sont expirés.

Cette procédure vise à corriger les impositions erronées ou exagérées, qu’elles
résultent :
\begin{itemize}
	\item d’erreurs matérielles ou comptables ;
	\item d’erreurs d’évaluation cadastrale ;
	\item ou de bases d’imposition conformes aux déclarations du contribuable mais
	révélées ultérieurement inexactes.
\end{itemize}

Les dégrèvements d’office ne se limitent pas aux erreurs imputables à
l’administration. Ils peuvent également être prononcés lorsque l’erreur est imputable
au contribuable lui-même, dans un souci d’équité et de bonne administration fiscale.

Dans le domaine des taxes foncières, cette procédure revêt une importance
particulière, compte tenu du rôle technique du cadastre dans la détermination de
l’assiette de l’impôt et de la fréquence des erreurs liées à l’actualisation des données
foncières.

\paragraph{Observation générale}

L’ensemble de ces formes de contentieux met en évidence le caractère largement
administratif du contentieux fiscal togolais. La majorité des litiges relatifs aux taxes
foncières sont réglés au stade administratif, la phase juridictionnelle n’intervenant
qu’en cas d’échec de la procédure préalable.

Cette organisation confère aux services fiscaux et cadastraux, intégrés au sein de
l’Office Togolais des Recettes, un rôle déterminant dans la garantie de la légalité de
l’impôt, la protection des droits des contribuables et la sécurisation des ressources
des collectivités territoriales.
