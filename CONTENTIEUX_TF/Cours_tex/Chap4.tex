
\documentclass[Cours_contentieuxTF2026.tex]{subfiles}

\begin{document}
	
	\chapter{La Procédure Gracieuse en matière de taxes foncières}
	
	
	Si le contentieux de l'assiette est le domaine du droit strict, la procédure gracieuse est le domaine de l'équité. Elle permet à l'Administration de tempérer la rigueur de la loi fiscale lorsque son application stricte conduirait à des situations socialement ou économiquement insupportables pour le contribuable.
	
	Pour le cadre du Cadastre, comprendre cette distinction est capital : on ne traite pas une demande de remise pour « indigence » avec les mêmes outils techniques (mesurage, plan) qu'une contestation de surface.
	
	\section{Notion et fondement de la procédure gracieuse}
	
	Alors que le contentieux de l'assiette est le règne de la légalité stricte (\textit{dura lex, sed lex}), la procédure gracieuse constitue une « soupape de sécurité » indispensable au système fiscal. Elle introduit une dimension humaine et sociale dans une mécanique administrative par nature rigide. Pour le cadre supérieur de l'OTR, il est impératif de ne pas confondre ce pouvoir de faveur avec une quelconque arbitrariété : la grâce obéit à sa propre logique administrative.
	
	\subsection{Nature juridique : Du pouvoir de disposition à l'équité}
	
	La juridiction gracieuse ne trouve pas son fondement dans le droit de l'impôt \textit{stricto sensu}, mais dans le pouvoir hiérarchique et financier de l'État sur ses propres créances.
	
	\begin{enumerate}
		\item \textbf{Le passage de la compétence liée au pouvoir discrétionnaire :}
		\begin{itemize}
			\item Dans le contentieux de l'assiette, l'Administration exerce une \textbf{compétence liée}. Si le contribuable prouve une violation de la loi, l'Administration a l'obligation juridique de dégrever.
			\item Dans la procédure gracieuse, l'Administration exerce un \textbf{pouvoir discrétionnaire}. Face à un contribuable indigent qui demande une remise, l'Administration n'est jamais tenue d'accepter. Elle apprécie l'opportunité de la décision en fonction de deux paramètres : la situation du redevable et les intérêts du Trésor.
		\end{itemize}
		
		\item \textbf{Le fondement de l'Équité :} \\
		La procédure gracieuse repose sur le principe selon lequel l'application aveugle de la loi peut parfois créer des injustices individuelles.
		\begin{itemize}
			\item \textit{Fondement :} L'\textbf{article 419} du LPF autorise l'autorité compétente à accorder des remises lorsque le contribuable se trouve dans l'impossibilité matérielle de s'acquitter de sa dette.
			\item \textit{Objet :} Il ne s'agit plus de contester la dette, mais de solliciter la bienveillance du créancier (l'État) pour en obtenir l'abandon total ou partiel.
		\end{itemize}
	\end{enumerate}
	
	\subsection{La distinction fondamentale avec le contentieux de droit}
	
	La confusion entre « réclamation contentieuse » et « demande gracieuse » est l'erreur la plus fréquente chez les contribuables. Cette distinction est pourtant structurelle.
	
	\begin{description}
		\item[Distinction par l'objet de la demande (Légalité vs Solvabilité)] \hfill
		\begin{itemize}
			\item \textbf{Le Contentieux (Légalité) :} Le contribuable conteste le principe même de l'impôt ou son calcul. Le débat est objectif.
			\textit{Phrase clé :} « Je ne dois pas payer cet impôt car il est faux. »
			\item \textbf{Le Gracieux (Solvabilité) :} Le contribuable ne conteste pas la régularité de l'imposition. Il admet implicitement que l'impôt est légalement dû. Il déplace le débat sur le terrain subjectif.
			\textit{Phrase clé :} « Je reconnais devoir cet impôt, mais je ne peux pas le payer. »
		\end{itemize}
		
		\item[Distinction par l'étendue du contrôle juridictionnel] \hfill
		\begin{itemize}
			\item Une décision contentieuse est soumise au \textbf{plein contrôle} du juge de l'impôt.
			\item Une décision gracieuse bénéficie d'une large immunité (contrôle restreint à l'erreur manifeste d'appréciation). En aucun cas, le juge ne peut se substituer à l'Administration pour accorder une remise.
		\end{itemize}
		
		\item[Conséquence procédurale majeure] \hfill \\
		Une demande gracieuse ne peut pas servir à contourner les délais du contentieux. Si un contribuable est forclos pour réclamer, il ne peut pas utiliser la voie gracieuse pour contester la légalité de l'impôt.
	\end{description}
	
	
\section{Typologie des mesures gracieuses}

Le pouvoir de bienveillance de l'Administration ne s'exerce pas de manière monolithique. Le Livre des Procédures Fiscales offre une palette d'instruments juridiques permettant d'adapter la réponse administrative à la gravité de la situation du contribuable. Pour le praticien, il est essentiel de maîtriser la nomenclature exacte de ces mesures, car chacune obéit à des conditions d'octroi spécifiques.

\subsection{La remise gracieuse (L'effacement de la dette)}

La remise est l'acte par lequel l'Administration renonce, unilatéralement, à recouvrer tout ou partie de sa créance. Elle constitue la mesure la plus radicale. Il convient d'opérer une distinction fondamentale (\textit{summa divisio}) selon l'assiette de la remise :

\begin{description}
	\item[La remise des pénalités (Le levier de la conformité)] \hfill \\
	\begin{itemize}
		\item \textit{Objet :} Elle porte sur les majorations de retard, les intérêts moratoires et les amendes fiscales.
		\item \textit{Philosophie :} Les pénalités ont une fonction punitive et dissuasive. Si le contribuable prouve sa bonne foi (ex : retard dû à une erreur de la banque ou à une hospitalisation), la sanction perd sa raison d'être.
		\item \textit{Pratique administrative :} C'est la mesure la plus couramment accordée. Elle est souvent utilisée comme un instrument de « négociation » : l'Administration accepte d'effacer les pénalités (l'accessoire) si le contribuable s'engage à régler immédiatement les droits simples (le principal). C'est la \textbf{remise conditionnelle}.
	\end{itemize}
	
	\item[La remise des droits (L'exception d'indigence)] \hfill \\
	\begin{itemize}
		\item \textit{Objet :} Elle porte sur l'impôt lui-même (la taxe foncière proprement dite).
		\item \textit{Gravité :} Accorder une remise de droits revient à priver le Trésor Public (et les collectivités locales) d'une recette budgétaire légitime. C'est une atteinte au principe d'égalité devant l'impôt.
		\item \textit{Condition stricte :} Elle n'est envisageable qu'en cas de \textbf{détresse financière absolue} (indigence, faillite personnelle irrémédiable, calamité nationale). Contrairement à la remise des pénalités qui sanctionne un comportement, la remise de droits sanctionne une impossibilité matérielle définitive de payer.
	\end{itemize}
\end{description}

\subsection{La modération fiscale (L'ajustement équitable)}

Souvent confondue avec la remise partielle, la modération est une notion plus subtile.

\begin{itemize}
	\item \textbf{Définition :} Elle consiste à ramener la cote d'impôt à un niveau jugé « équitable » ou « supportable » pour le contribuable, sans l'effacer totalement.
	\item \textbf{Champ d'application :} Elle est particulièrement adaptée aux impôts indiciaires comme la taxe foncière.
	\textit{Exemple : Un propriétaire âgé possède une vaste demeure familiale dont la Valeur Locative est très élevée, mais ses revenus de retraite sont très faibles. La modération permet de réduire la taxe pour qu'elle n'absorbe pas l'intégralité du revenu disponible.}
\end{itemize}

\subsection{La transaction fiscale (Le contrat judiciaire)}

Bien que distincte de la procédure gracieuse \textit{stricto sensu}, la transaction est un mode alternatif de règlement des litiges qui s'en rapproche par sa finalité d'apaisement.

\begin{enumerate}
	\item \textbf{Nature contractuelle :} \\
	Contrairement à la remise (acte unilatéral), la transaction est un \textbf{contrat synallagmatique} (bilatéral) par lequel les parties terminent une contestation née ou préviennent une contestation à naître.
	
	\item \textbf{Le mécanisme des concessions réciproques :} \\
	\begin{itemize}
		\item \textit{L'Administration} consent à abandonner une partie des rappels d'impôts ou, plus souvent, la totalité des pénalités graves.
		\item \textit{Le Contribuable} s'engage, en contrepartie, à payer immédiatement le solde convenu et, surtout, à renoncer à tout contentieux présent ou futur sur cette affaire.
	\end{itemize}
	
	\item \textbf{Limite d'ordre public :} \\
	En droit fiscal, \textbf{on ne transige pas sur le principe de l'impôt}, mais uniquement sur les conséquences financières et les pénalités. La transaction ne peut jamais avoir pour effet d'accorder une exemption illégale.
\end{enumerate}	
	
	
 \section{Les critères d'octroi des mesures gracieuses}
 
 L'octroi d'une mesure gracieuse constitue une dérogation au principe constitutionnel d'égalité devant l'impôt. Par conséquent, cette décision ne peut reposer sur le simple bon vouloir de l'agent instructeur. Elle doit résulter d'une analyse \textit{in concreto} et objective de la situation du requérant. L'Administration fiscale fonde sa doctrine sur deux piliers cumulatifs : l'incapacité matérielle de payer (critère économique) et le civisme du contribuable (critère moral).
 
 \subsection{L'appréciation de la situation de fortune (La capacité contributive réelle)}
 
 L'instruction gracieuse vise à déterminer si le paiement intégral de la dette fiscale compromettrait les besoins vitaux du contribuable. Cette analyse repose sur la distinction fondamentale entre la « gêne passagère » et l'« indigence ».
 
 \begin{description}
 	\item[L'analyse du bilan financier du ménage] \hfill \\
 	L'instructeur dresse un état comparatif des ressources et des charges.
 	\begin{itemize}
 		\item \textbf{Le « Reste à Vivre » :} C'est l'indicateur clé. Une fois déduites les charges incompressibles (alimentation, santé, écolage, loyer), le solde disponible permet-il d'apurer la dette, même de manière échelonnée ?
 		\item \textbf{La disproportion manifeste :} La remise est envisagée lorsqu'il existe une disproportion flagrante entre la dette fiscale et le revenu annuel (ex : une taxe foncière représentant 6 mois de pension de retraite).
 	\end{itemize}
 	
 	\item[L'analyse du patrimoine (La solvabilité apparente vs réelle)] \hfill \\
 	Le rôle du Cadastre est ici prépondérant pour déjouer les fausses déclarations d'insolvabilité.
 	\begin{itemize}
 		\item \textbf{Le patrimoine réalisable :} Un contribuable qui plaide la pauvreté mais possède plusieurs parcelles non bâties ou des immeubles de rapport (locatifs) se verra opposer l'argument de la « réalisation d'actifs ». L'État n'a pas vocation à accorder une subvention déguisée pour la conservation d'un patrimoine privé.
 		\item \textbf{La résidence principale :} En revanche, l'Administration évite généralement de contraindre un redevable à vendre sa résidence principale, surtout s'il s'agit d'une personne vulnérable.
 	\end{itemize}
 	
 	\item[Les circonstances atténuantes exogènes] \hfill \\
 	L'examen prend en compte les accidents de la vie qui ont altéré la capacité contributive : décès du conjoint, maladie invalidante, perte d'emploi brutale, ou sinistre ayant affecté l'immeuble non couvert par une assurance.
 \end{description}
 
 \subsection{Le comportement fiscal du redevable (Le critère moral)}
 
 La juridiction gracieuse est une juridiction de faveur. En vertu de l'adage \textit{« Nemo auditur propriam turpitudinem allegans »} (Nul ne peut se prévaloir de sa propre turpitude), la bienveillance de l'État ne profite qu'aux contribuables de bonne foi.
 
 \begin{enumerate}
 	\item \textbf{L'historique fiscal (Le civisme) :} \\
 	L'instructeur consulte le « casier fiscal » du requérant.
 	\begin{itemize}
 		\item Un contribuable qui a toujours respecté ses obligations déclaratives et de paiement par le passé bénéficie d'une présomption de bonne foi.
 		\item À l'inverse, un contribuable coutumier des retards systématiques verra sa demande rejetée ou limitée.
 	\end{itemize}
 	
 	\item \textbf{L'origine de la dette (Bonne foi vs Manœuvres frauduleuses) :} \\
 	C'est le critère dirimant.
 	\begin{itemize}
 		\item Si la dette provient d'une erreur involontaire, la voie gracieuse est ouverte.
 		\item Si la dette (et surtout les pénalités) résulte d'un \textbf{contrôle fiscal ayant révélé une fraude} (ex : construction illégale non déclarée, minoration de loyers), la remise est, par principe, exclue. L'Administration ne saurait rendre d'une main ce qu'elle a repris de l'autre pour sanctionner une violation de la loi.
 	\end{itemize}
 	
 	\item \textbf{La cohérence des déclarations :} \\
 	L'instruction gracieuse est souvent l'occasion d'un recoupement ultime. Si l'examen du train de vie (véhicules, voyages, scolarité) contredit l'état d'indigence allégué, la remise est refusée et le dossier peut être transmis au contrôle fiscal.
 \end{enumerate}
 
 \vspace{0.5cm}
 
 \begin{center}
 	\fbox{\begin{minipage}{0.9\textwidth}
 			\textbf{Synthèse :} L'agent du Cadastre, lorsqu'il est consulté pour avis sur une demande gracieuse, ne doit pas se limiter à vérifier la surface. Il doit éclairer le décideur sur le patrimoine réel du demandeur (autres propriétés dans la base de données) pour éviter que l'État n'accorde une remise à un « faux indigent ».
 	\end{minipage}}
 \end{center}
 
 \section{Instruction et décision en matière gracieuse}
 
 L'instruction de la demande gracieuse diffère fondamentalement de l'instruction contentieuse. Elle n'est pas dirigée vers la recherche de la vérité juridique (qui a tort ? qui a raison ?), mais vers l'appréciation de l'opportunité d'une décision de faveur. C'est une procédure administrative interne qui aboutit à une décision unilatérale, soumise à un régime de contrôle juridictionnel spécifique.
 
 \subsection{Le pouvoir discrétionnaire de l'Administration (L'opportunité)}
 
 Le principe directeur de la matière gracieuse est celui du \textbf{pouvoir discrétionnaire}. Ce concept juridique, central en droit administratif, mérite une définition précise pour les praticiens.
 
 \begin{enumerate}
 	\item \textbf{Définition et portée :} \\
 	Contrairement à la compétence liée (où l'Administration \textit{doit} agir dans un sens déterminé par la loi), le pouvoir discrétionnaire confère à l'autorité décidante une \textbf{liberté de choix}.
 	\begin{itemize}
 		\item Face à une demande de remise, même si le contribuable est objectivement dans la gêne, le Commissaire Général n'est jamais \textit{tenu} d'accorder la faveur sollicitée. Il apprécie souverainement l'opportunité de la mesure au regard de l'intérêt général et des impératifs budgétaires.
 	\end{itemize}
 	
 	\item \textbf{La délégation de compétence :} \\
 	En pratique, pour éviter l'engorgement du sommet de la hiérarchie, le pouvoir de décision est délégué selon des seuils financiers (fixés par arrêté) :
 	\begin{itemize}
 		\item Les Chefs de Division ou Directeurs des Impôts (DGE/DME) pour les petits montants.
 		\item Le Commissaire des Impôts pour les montants intermédiaires.
 		\item Le Commissaire Général pour les montants élevés ou les remises de droits en principal.
 	\end{itemize}
 \end{enumerate}
 
 \subsection{Les limites juridiques (Le rempart contre l'arbitraire)}
 
 Discrétionnaire ne signifie pas arbitraire. L'État de droit impose des bornes à cette liberté d'appréciation pour garantir l'équité du système fiscal.
 
 \begin{itemize}
 	\item \textbf{Le respect du Principe d'Égalité :} \\
 	C'est la limite constitutionnelle majeure. L'Administration ne peut traiter différemment deux contribuables placés dans une situation identique.
 	\textit{Exemple d'illégalité : Accorder une remise à un contribuable A pour motif politique et la refuser au contribuable B voisin est une violation du principe d'égalité.}
 	
 	\item \textbf{L'interdiction du Détournement de Pouvoir :} \\
 	La décision gracieuse doit servir exclusivement l'intérêt général. Si la remise est accordée pour des motifs étrangers au service (favoritisme personnel, corruption), la décision est entachée d'illégalité.
 	
 	\item \textbf{La décision conditionnelle (La transaction) :} \\
 	L'Administration peut assortir sa décision de conditions strictes.
 	\textit{Mécanisme : « J'accorde la remise des pénalités à la condition expresse que le principal soit payé sous 8 jours ». Si le contribuable ne paie pas, la décision devient caduque et la dette revit.}
 \end{itemize}
 
 \subsection{L'absence de droit au recours sur le fond (L'immunité juridictionnelle)}
 
 La décision prise en matière gracieuse bénéficie d'un régime contentieux particulier.
 
 \begin{description}
 	\item[L'incompétence du juge sur le fond] \hfill \\
 	Le juge de l'impôt n'est pas le supérieur hiérarchique de l'Administration. Il ne peut pas se substituer à elle pour accorder une grâce.
 	\textit{Règle : Un contribuable ne peut pas demander au Tribunal d'ordonner à l'OTR de lui accorder une remise.}
 	
 	\item[L'exception : Le Recours pour Excès de Pouvoir (REP)] \hfill \\
 	Si le contribuable ne peut attaquer le refus sur son opportunité, il peut l'attaquer sur sa \textbf{légalité externe}. Il peut saisir le juge administratif si la décision est entachée :
 	\begin{itemize}
 		\item D'un \textbf{vice de forme} ou de procédure (ex : incompétence du signataire).
 		\item D'une \textbf{erreur manifeste d'appréciation} (ex : refus opposé à un sinistré alors que l'Administration soutient qu'il est riche).
 		\item D'une \textbf{erreur de droit}.
 	\end{itemize}
 	\textit{Effet de l'annulation : Si le juge annule le refus, il ne donne pas la remise lui-même. Il renvoie l'affaire devant l'Administration qui a l'obligation de réexaminer la demande.}
 \end{description}
 
 \vspace{0.5cm}
 
 \begin{center}
 	\fbox{\begin{minipage}{0.9\textwidth}
 			\textbf{Synthèse finale :} Le contentieux gracieux est un outil de gestion sociale de l'impôt. Pour l'inspecteur, c'est le levier qui permet de distinguer le mauvais payeur (qui subira la rigueur de la loi) du contribuable malchanceux (qui bénéficiera de la solidarité nationale).
 	\end{minipage}}
 \end{center}
 
 
 \section{Instruction et décision en matière gracieuse}
 
 L'instruction de la demande gracieuse est une procédure administrative encadrée par les articles 419 et suivants du Livre des Procédures Fiscales (LPF). Si le pouvoir est discrétionnaire, la procédure, elle, est réglementée pour garantir la transparence et protéger les intérêts du Trésor.
 
 \subsection{La compétence décisionnelle et les seuils (Ratione materiae)}
 
 La compétence pour accorder une remise, une modération ou une transaction est strictement hiérarchisée en fonction du montant des sommes en jeu (droits + pénalités) par cote ou par affaire.
 
 \begin{enumerate}
 	\item \textbf{Le partage de compétence (Arts. 423, 426, 428 LPF) :}
 	\begin{itemize}
 		\item \textbf{Le Commissaire des Impôts :} Est compétent lorsque les sommes faisant l'objet de la demande n'excèdent pas \textbf{deux cents millions (200 000 000) de francs CFA}.
 		\item \textbf{Le Commissaire Général :} Est seul compétent au-delà de cette limite de 200 millions de FCFA.
 		\item \textit{Délégation (Art. 427 LPF) :} Le Commissaire des Impôts peut déléguer sa signature aux Directeurs compétents (ex : Directeur des Grandes Entreprises), dans des limites qu'il fixe par décision interne.
 	\end{itemize}
 	
 	\item \textbf{L'exclusion de compétence (Art. 419 in fine) :} \\
 	Il est capital pour l'agent du Cadastre de noter qu'aucune autorité publique (pas même le Commissaire Général) ne peut accorder de remise totale ou partielle sur :
 	\begin{itemize}
 		\item Les \textbf{droits d'enregistrement} (très fréquent en matière foncière) ;
 		\item Les droits de timbre ;
 		\item Les taxes sur le chiffre d'affaires (TVA).
 	\end{itemize}
 	\textit{Conséquence : Une demande gracieuse portant sur des droits d'enregistrement d'un titre foncier est irrecevable de plano.}
 \end{enumerate}
 
 \subsection{La procédure d'instruction et de notification}
 
 L'article 420 du LPF impose que la demande soit adressée au service territorial du lieu d'imposition, accompagnée de l'avis d'imposition ou de l'AMR.
 
 \begin{description}
 	\item[Le pouvoir de rejet immédiat (Art. 421 LPF)] \hfill \\
 	Le Commissaire des Impôts dispose d'un pouvoir de « filtrage ». Il peut se prononcer \textbf{sans instruction préalable} si, au vu des procédures en cours, la demande ne peut manifestement pas être accueillie favorablement (ex : demande dilatoire).
 	
 	\item[La formalisation de la proposition (Art. 422 LPF)] \hfill \\
 	Si l'Administration accepte le principe d'une mesure gracieuse (notamment une transaction), elle ne l'impose pas, elle la « propose ».
 	\begin{itemize}
 		\item \textbf{Forme :} La proposition est notifiée par lettre recommandée avec avis de réception (LRAR).
 		\item \textbf{Contenu :} Le document chiffre précisément le montant résiduel de l'impôt et des pénalités que le contribuable devra payer s'il accepte l'offre.
 		\item \textbf{Délai d'option :} Le contribuable dispose d'un \textbf{délai strict de quinze (15) jours} à compter de la réception pour accepter ou refuser. Le silence ou le refus entraîne la caducité de l'offre et le retour à l'exigibilité totale.
 	\end{itemize}
 \end{description}
 
 \subsection{Les limites légales au pouvoir de remise (Les verrous)}
 
 Le pouvoir d'appréciation de l'Administration se heurte à des interdictions légales visant à moraliser la vie fiscale.
 
 \begin{enumerate}
 	\item \textbf{Le plafond de la mauvaise foi (Art. 431 LPF) :} \\
 	C'est une disposition fondamentale pour lutter contre la fraude. Si la mauvaise foi du contribuable est établie (manœuvres frauduleuses, dissimulation), ou s'il a omis de déclarer des revenus encaissés hors du Togo :
 	\begin{itemize}
 		\item Le pouvoir de remise est \textbf{plafonné à 50 \%}.
 		\item L'Administration a l'interdiction légale d'aller au-delà de ce taux. Une remise de 80 \% accordée à un fraudeur serait illégale.
 	\end{itemize}
 	
 	\item \textbf{Le verrou juridictionnel (Art. 430 LPF) :} \\
 	Une fois qu'un jugement définitif a été rendu par un tribunal, l'Administration perd sa liberté totale.
 	\begin{itemize}
 		\item Toute demande de remise de sanctions fiscales après jugement doit être soumise au \textbf{Président de la juridiction} qui a prononcé la condamnation.
 		\item La remise ne peut être accordée qu'après \textbf{avis conforme} de ce magistrat.
 	\end{itemize}
 \end{enumerate}
 
 \subsection{La Transaction : Effets et Portée (Art. 425 LPF)}
 
 La transaction est un mécanisme spécifique prévu pour les impositions \textit{« encore susceptibles de faire l'objet d'un recours contentieux »} (non définitives).
 
 \begin{itemize}
 	\item \textbf{L'effet extinctif :} La conclusion d'une transaction a l'autorité de la chose jugée entre les parties. Dès que la transaction est conclue (signée et payée), \textbf{aucune procédure contentieuse ne peut être engagée ou reprise}. Le contribuable renonce définitivement à contester le bien-fondé de l'impôt devant le juge.
 	
 	\item \textbf{Le risque du refus :} Si le contribuable refuse la transaction proposée et porte le litige devant le tribunal, il prend un risque : le juge fixera alors le taux des pénalités en même temps que la base de l'impôt.
 	
 	\item \textbf{Les voies de recours (Art. 424 LPF) :} La décision de rejet est en principe définitive. Un recours n'est possible devant la même autorité que si, et seulement si, des \textbf{faits nouveaux} sont invoqués (ex : aggravation subite de la situation financière postérieurement à la première demande).
 \end{itemize}
 
	
\end{document}