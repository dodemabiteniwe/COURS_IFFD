\documentclass[Cours_contentieuxTF2026.tex]{subfiles}

\begin{document}
\chapter{Ordonnancement, Restitution et Exécution des Décisions}

La clôture de la phase administrative du contentieux des taxes foncière par une décision favorable (dégrèvement contentieux ou remise gracieuse) ne met pas immédiatement fin au litige dans les faits. La décision de l'autorité fiscale, acte purement juridique, doit être traduite en écritures comptables. Ce chapitre aborde la mécanique interne de l'Administration qui permet de purger la dette du contribuable et, le cas échéant, de le rembourser, tout en protégeant les deniers publics.

\section{L'ordonnancement des dégrèvements (Le formalisme comptable)}

L'ordonnancement est l'acte administratif par lequel l'autorité compétente (l'ordonnateur) donne l'ordre au comptable public (le Receveur) de modifier ses prises en charge.

\subsection{La procédure administrative (L'Article 534 du LPF)}

En droit fiscal, une simple lettre notifiant un dégrèvement à un contribuable n'a aucune valeur comptable pour le Trésor. Le mécanisme est rigoureusement encadré par l'\textbf{article 534 du LPF}.

\begin{itemize}
	\item \textbf{Le Certificat de dégrèvement :} Le montant de la décharge, réduction, remise ou modération doit obligatoirement faire l'objet d'un « certificat ». Ce document officiel est établi et signé par l'ordonnateur (le Commissaire Général ou son délégataire).
	\item \textbf{La transmission au Receveur :} Ce certificat est ensuite adressé au receveur compétent. Le receveur ne participe pas à la décision de dégrèvement ; il en est l'exécutant.
	\item \textbf{L'ajustement des prises en charge :} Au vu de ce certificat, le receveur procède par voie de diminution du montant de ses « prises en charge » (il réduit le montant global des créances qu'il doit recouvrer).
	\item \textbf{La justification :} Le receveur a l'obligation de conserver précieusement ces certificats. Ils constituent les pièces justificatives qui lui permettront de prouver à la Cour des Comptes pourquoi il n'a pas recouvré l'intégralité du rôle qui lui avait été confié.
\end{itemize}

\subsection{Le circuit de validation et de sécurité}

Pour éviter les fraudes (par exemple, un agent créant un faux dégrèvement pour effacer la dette d'un proche), le circuit d'ordonnancement est informatisé et verrouillé. Le service d'assiette (le Cadastre) initie le projet de dégrèvement, mais ne peut le valider définitivement dans le système. Seule la validation électronique de l'ordonnateur génère le certificat que le receveur pourra imputer dans sa comptabilité.

\section{La restitution des sommes indûment perçues}

En vertu du principe du « privilège du préalable », le contribuable est souvent contraint de payer sa taxe foncière même s'il la conteste. Si le dégrèvement survient \textit{après} le paiement, l'État devient débiteur du citoyen.

\subsection{Conditions et modalités de restitution (L'Article 535 du LPF)}

L'\textbf{article 535 du LPF} encadre très précisément ce droit au remboursement.

\begin{enumerate}
	\item \textbf{La condition de l'excédent :} Le remboursement n'intervient que si les sommes versées par le contribuable (avant le dégrèvement) excèdent le montant de la nouvelle cote révisée.
	\item \textbf{Le mécanisme de l'ordre de paiement :} Le comptable ne pioche pas simplement dans sa caisse pour rembourser. La restitution s'opère au vu d'un « ordre de paiement » formel.
	\item \textbf{La libération contre reçu :} Le remboursement est effectué contre un reçu (ou une décharge électronique en cas de virement bancaire), permettant au comptable de justifier la sortie de fonds de sa caisse.
\end{enumerate}

\subsection{Délais et compensation}

L'Administration s'efforce de procéder aux restitutions dans les meilleurs délais pour éviter le versement d'intérêts moratoires (intérêts de retard payés par l'État au contribuable). Toutefois, avant de procéder au remboursement effectif par virement, le Receveur vérifiera si le contribuable n'a pas d'autres dettes fiscales en souffrance (par exemple, des impayés sur une autre parcelle). Si c'est le cas, il procédera d'abord à une \textbf{compensation légale} d'office.

\section{La responsabilité du comptable public et l'apurement des restes à recouvrer}

Le receveur des impôts n'est pas un simple caissier. En droit public, il est personnellement et pécuniairement responsable du recouvrement des impôts figurant sur les rôles qui lui sont transmis. S'il ne les recouvre pas, il doit les payer sur ses propres deniers, sauf à mettre en œuvre les procédures protectrices prévues par le LPF.

\subsection{L'admission en non-valeurs des cotes irrécouvrables (Art. 536 LPF)}

Il arrive qu'une taxe foncière soit juridiquement parfaitement valable, mais matériellement impossible à recouvrer (ex : faillite totale du propriétaire, disparition physique de l'immeuble non signalée, insolvabilité absolue).

\begin{description}
	\item[La demande du receveur :] Selon l'\textbf{article 536 du LPF}, le receveur peut demander l'admission en « non-valeurs » (ANV) des cotes devenues irrécouvrables suite à une modification de la fortune ou de la situation du contribuable.
	\item[L'instruction par l'Assiette :] Le receveur ne peut pas déclarer lui-même la dette irrécouvrable. L'instruction du dossier est faite par le service d'assiette (le Cadastre), qui doit confirmer, après enquête, l'impossibilité de poursuivre le recouvrement.
	\item[La décision et le recours :] Le pouvoir de statuer appartient à l'ordonnateur. Si l'ordonnateur refuse l'ANV (estimant que le receveur n'a pas fait assez d'efforts pour recouvrer), le comptable peut porter ce litige interne devant la \textbf{Cour des comptes}.
\end{description}

\subsection{Le Sursis de versement (Arts. 537 à 539 LPF)}

L'\textbf{article 537 du LPF} est le couperet du comptable : il ne peut être dispensé de verser \textit{sur ses deniers personnels} les impôts non recouvrés que s'il obtient un sursis, une décharge ou une admission en non-valeur.

\begin{itemize}
	\item \textbf{Le mécanisme du sursis (Art. 538 et 539) :} Même si les taxes ne sont pas encore formellement admises comme irrécouvrables, le receveur peut demander à sa hiérarchie un « sursis de versement ». Ce sursis est accordé par l'ordonnateur pour une durée d'\textbf{une année} (renouvelable une seule fois). Il donne un répit au comptable pour suspendre temporairement sa mise en débet pendant qu'il épuise les dernières voies d'exécution.
\end{itemize}

\subsection{La demande en décharge ou atténuation de responsabilité (Art. 540 LPF)}

C'est l'ultime rempart pour le comptable public.

\begin{enumerate}
	\item \textbf{La procédure :} Si la demande d'admission en non-valeur (Art. 536) a été rejetée par l'ordonnateur, le receveur est en principe tenu de combler le trou dans la caisse. Toutefois, l'\textbf{article 540 du LPF} lui permet de formuler une demande motivée en « décharge ou atténuation de responsabilité ».
	\item \textbf{L'arbitrage :} L'ordonnateur statuera définitivement sur cette demande, mais seulement après avoir recueilli l'avis préalable du Commissaire des impôts. S'il est prouvé que l'échec du recouvrement n'est pas lié à une négligence du receveur (ex : force majeure, défaillance structurelle de l'assiette), ce dernier sera déchargé de sa responsabilité pécuniaire.
\end{enumerate}	
	
\end{document}