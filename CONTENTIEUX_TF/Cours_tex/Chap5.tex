
\documentclass[Cours_contentieuxTF2026.tex]{subfiles}

\begin{document}

\chapter{Les Dégrèvements Et Remises D'office}

Jusqu'à présent, nous avons étudié les procédures initiées par le contribuable. Or, l'Administration fiscale n'est pas une machine aveugle. Elle dispose, en vertu de l'\textbf{article 418 du Livre des Procédures Fiscales (LPF)}, du pouvoir de rectifier spontanément les erreurs qu'elle constate.
Cette procédure, dite « d'office », témoigne de la volonté de l'OTR de garantir une justice fiscale proactive, sans attendre que le citoyen ne se manifeste.

\section{Notion et fondement juridique }

\subsection{Le fondement textuel (L'Article 418 du LPF)}

Le pouvoir de régularisation de l'Administration n'est pas une faveur arbitraire, c'est une compétence légale définie par l'article 418 :

\begin{quote}
	\textit{« L'Administration fiscale peut prononcer d'office le dégrèvement ou la restitution d'impositions qui n'étaient pas dues, jusqu'au 31 décembre de la troisième année suivant celle au cours de laquelle le délai de réclamation a pris fin [...] »}
\end{quote}

Cette disposition pose trois principes directeurs :

\begin{enumerate}
	\item \textbf{Le pouvoir d'initiative (\textit{Sponte Sua}) :} L'expression « prononcer d'office » signifie que l'Administration agit de son propre mouvement, sans besoin d'une requête formelle du contribuable.
	
	\item \textbf{L'objet de la mesure :} Elle vise les « impositions qui n'étaient pas dues ». Il s'agit donc de réparer une \textbf{illégalité} ou une \textbf{erreur matérielle} (impôt mal calculé, doublon, erreur de personne). Cela distingue le dégrèvement d'office (réparation juridique) de la remise gracieuse (faveur sociale).
	
	\item \textbf{L'effet restitutionnel :} Si le contribuable a déjà payé l'impôt indûment réclamé, l'article 418 ouvre droit à « restitution ». L'État doit rembourser le trop-perçu.
\end{enumerate}

\subsection{Le cadre temporel : Le délai spécial de prescription}

L'apport majeur de l'article 418 réside dans l'extension des délais. Alors que le contribuable est enfermé dans un délai court (généralement le 31 décembre de l'année N+1), l'Administration dispose d'un temps beaucoup plus long pour corriger ses fautes.

\subsubsection*{Analyse du délai de l'Art. 418}

Le texte fixe la limite au \textit{« 31 décembre de la troisième année suivant celle au cours de laquelle le délai de réclamation a pris fin »}.

\begin{center}
	\fbox{\begin{minipage}{0.8\textwidth}
			\textbf{Exemple pratique de calcul (Cas d'école) :}
			\begin{itemize}
				\item \textbf{Année d'imposition (N) :} Taxe Foncière 2025.
				\item \textbf{Fin du délai de réclamation du contribuable (N+1) :} 31 décembre 2026.
				\item \textbf{Délai de l'Administration (N+4) :} Le délai court pendant 3 ans à partir de la fin 2026.
				\begin{itemize}
					\item Année 1 : 2027
					\item Année 2 : 2028
					\item Année 3 : 2029
				\end{itemize}
				\item \textbf{Date limite pour agir d'office :} \textbf{31 décembre 2029.}
			\end{itemize}
	\end{minipage}}
\end{center}

\textbf{Conséquence pour l'Inspecteur :} Si vous découvrez en 2028 une erreur sur une taxe de 2025, le contribuable est forclos (il ne peut plus réclamer), mais vous, Administration, avez encore la compétence juridique pour dégrever d'office sur la base de l'article 418. C'est un filet de sécurité puissant.


\section{Cas d'application en matière foncière}

L'application de l'article 418 du LPF trouve son terrain de prédilection dans la fiscalité foncière. Contrairement aux impôts déclaratifs (TVA, IS) qui reposent sur la comptabilité du contribuable, la taxe foncière repose sur le constat administratif (recensement). L'Administration est donc la première responsable de la qualité des données.
Les cas d'ouverture du dégrèvement d'office peuvent être classés en trois catégories distinctes.

\subsection{Les erreurs purement matérielles (Le « Faux matériel »)}

Il s'agit des dysfonctionnements techniques du système d'information ou des erreurs de saisie manuelles. Ces erreurs sont objectives : elles ne nécessitent aucune interprétation juridique, leur simple constat suffit à déclencher la procédure.

\begin{enumerate}
	\item \textbf{Les doubles emplois (Doublons de cotes) :} \\
	C'est le cas le plus fréquent dans un cadastre en cours de modernisation. Un même immeuble est imposé deux fois au titre de la même année.
	\begin{itemize}
		\item \textit{Cas typique :} Une parcelle est taxée sous son ancien numéro de réquisition et simultanément sous son nouveau numéro de Titre Foncier.
		\item \textit{Action :} L'Administration doit prononcer le dégrèvement d'office de la cote la plus ancienne ou la moins précise pour ne conserver que la cote valide.
	\end{itemize}
	
	\item \textbf{Les erreurs de saisie (Sur-évaluation manifeste) :} \\
	Elles résultent d'une « faute de plume » ou de clavier lors de l'intégration des données dans le logiciel fiscal.
	\begin{itemize}
		\item \textit{Exemple :} Une Valeur Locative (VL) saisie avec un zéro supplémentaire (ex : 10 000 000 F au lieu de 1 000 000 F). Cette erreur entraîne une imposition confiscatoire qui « n'est pas due » au sens de l'art. 418.
	\end{itemize}
	
	\item \textbf{L'erreur sur la personne du débiteur (Faux contribuable) :} \\
	L'imposition est établie au nom d'une personne qui n'a aucun lien de droit avec l'immeuble (homonymie, ancien locataire taxé comme propriétaire, propriétaire décédé depuis plus de 3 ans dont la succession est close).
\end{enumerate}

\subsection{Les erreurs juridiques et cadastrales (Le « Faux juridique »)}

Ici, l'erreur porte sur la qualification de l'immeuble ou sur son existence légale au regard de la loi fiscale.

\begin{description}
	\item[La perte de la matière imposable (Disparition de l'assiette)] \hfill \\
	La taxe foncière est un impôt annuel dû sur les immeubles existant au 1\textsuperscript{er} janvier.
	\begin{itemize}
		\item \textit{Cas d'application :} Si un immeuble a été démoli, incendié ou est tombé en ruine courant de l'année N, il ne doit plus être imposé en N+1. Si le système informatique reconduit l'imposition par automatisme, cette cote est illégale.
		\item \textit{Action :} Dès que l'inspecteur constate la vacance structurelle ou la ruine (par PV de constat ou imagerie), il doit dégrever d'office les années postérieures à la perte de l'immeuble.
	\end{itemize}
	
	\item[Le non-respect des exonérations légales de plein droit] \hfill \\
	Certains immeubles sont exonérés de par la loi (CGI) en raison de leur affectation. Si l'Administration les taxe par méconnaissance de leur statut réel, elle commet une erreur de droit.
	\textit{Exemples : Taxation d'un édifice cultuel (église, mosquée), d'un bâtiment agricole servant exclusivement à l'exploitation, ou d'une propriété diplomatique.}
	
	\item[L'erreur de consistance (Surface ou Catégorie)] \hfill \\
	Il s'agit d'une erreur sur les éléments physiques servant au calcul de la Valeur Locative.
	\textit{Exemple : Un terrain nu taxé par erreur comme terrain bâti (TFPB au lieu de TFPNB), ou une maison de standing « économique » taxée au tarif « luxe ».}
\end{description}

\subsection{Les révisions générales (Mesures de portée collective)}

Le dégrèvement d'office peut dépasser le cadre individuel pour devenir un outil de gestion des crises ou de la base de données.

\begin{itemize}
	\item \textbf{Les calamités publiques (Force majeure) :} \\
	Lorsqu'un événement exceptionnel (inondations majeures, glissement de terrain) rend inhabitable toute une zone géographique, l'Administration peut par décision du Commissaire Général ou décret, procéder au dégrèvement d'office collectif de toutes les taxes foncières de la zone sinistrée. C'est une mesure de solidarité nationale.
	
	\item \textbf{L'épuration des cotes irrécouvrables (Gestion comptable) :} \\
	Dans une logique de performance, l'Administration peut décider de dégrever d'office des milliers de « micro-cotes » ou de cotes « orphelines » (propriétaire inconnu, terrain introuvable) qui polluent le Reste à Recouvrer (RAR) du Receveur. Le coût de gestion de ces impôts étant supérieur à leur montant, le dégrèvement agit comme un assainissement comptable.
\end{itemize}

\section{Le rôle proactif du service du Cadastre (La fiabilisation)}

L'application de l'article 418 du LPF ne doit pas être vue comme une simple procédure de nettoyage, mais comme une fonction stratégique de \textbf{« Contrôle Qualité »}. L'Inspecteur du Cadastre n'est pas seulement un producteur de bases taxables ; il est le garant de la sincérité du fichier immobilier. Son rôle proactif est indispensable pour transformer une « donnée brute » (un plan) en une « recette nette » (de l'argent dans les caisses de l'État).

\subsection{La détection des erreurs : De l'arpentage à l'audit de données}

L'initiative du dégrèvement d'office repose sur la capacité du service à capter les signaux faibles indiquant une anomalie d'imposition.

\begin{enumerate}
	\item \textbf{Le traitement des retours de courrier (NPAI) :} \\
	Le premier indicateur d'une base fiscale corrompue est le taux de Plis Non Distribués (PND) ou « N'habite Pas à l'Adresse Indiquée » (NPAI).
	\begin{itemize}
		\item \textit{Analyse :} Un avis d'imposition qui revient systématiquement chaque année signifie souvent que le contribuable est inconnu ou que l'immeuble est fictif.
		\item \textit{Action proactive :} Au lieu de renvoyer ces plis au stock, le service du Cadastre doit auditer ces cotes. Si l'inexistence est confirmée, la procédure de dégrèvement d'office doit être enclenchée pour arrêter l'hémorragie de frais de gestion.
	\end{itemize}
	
	\item \textbf{Le recoupement avec les données externes (L'interopérabilité) :} \\
	L'administration moderne fonctionne en réseau. La détection des « impositions qui ne sont plus dues » (Art. 418) se fait par croisement de fichiers :
	\begin{itemize}
		\item \textit{Avec les Services Techniques Municipaux :} La réception des permis de démolir doit déclencher automatiquement une alerte dans le Système d'Information Foncier (SIF).
		\item \textit{Avec les concessionnaires (CEET/TDE) :} Une coupure définitive de compteur électrique indique souvent une vacance ou une ruine.
	\end{itemize}
	
	\item \textbf{Les tournées de conservation cadastrale :} \\
	Lors des mises à jour annuelles, l'inspecteur ne doit pas seulement chercher les \textit{nouvelles} constructions (matière nouvelle), mais aussi identifier la \textit{perte} de matière (immeubles écroulés, zones érodées).
\end{enumerate}

\subsection{La sécurisation des bases fiscales : L'impact sur le recouvrement}

Le dégrèvement d'office, loin d'être une « perte » pour l'État, est un outil d'assainissement budgétaire. Il existe une corrélation directe entre la qualité de l'assiette (Cadastre) et la performance du recouvrement (Receveur).

\begin{description}
	\item[L'assainissement du Reste à Recouvrer (RAR)] \hfill \\
	Le taux de recouvrement se calcule par la formule suivante :
	\[
	\text{Taux de Recouvrement} = \frac{\text{Recettes Encaissées}}{\text{Prises en charge (Rôle émis)}}
	\]
	\begin{itemize}
		\item \textit{Problématique :} Si le dénominateur (le Rôle) est gonflé artificiellement par des cotes fictives ou des doublons, le taux de recouvrement s'effondre mécaniquement.
		\item \textit{Stratégie :} En dégrevant d'office les \textbf{« non-valeurs »} (créances irrécouvrables car sans objet), le Cadastre « dégraisse » le rôle. Il permet au Receveur de concentrer ses moyens de coercition sur les créances réelles.
	\end{itemize}
	
	\item[La responsabilité administrative et comptable] \hfill \\
	Proposer un dégrèvement d'office est un acte de responsabilité.
	\begin{itemize}
		\item \textit{La preuve :} Pour éviter tout soupçon de complaisance, toute proposition de dégrèvement d'office doit être étayée par un dossier technique irréfutable (photos géolocalisées, PV de constat).
		\item \textit{L'obligation de résultat :} Laisser perdurer des cotes erronées constitue une faute de gestion, car cela fausse les prévisions budgétaires des Collectivités Locales.
	\end{itemize}
\end{description}


\end{document}